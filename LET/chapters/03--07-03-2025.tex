\chapter{07/03/2025}

\section{Pulizia context free}

\paragraph{Simboli non definiti}

Simboli che non hanno produzioni associate.
Esempio:

\begin{itemize}
  \item $S \rightarrow ABC$
  \item $A \rightarrow a$
  \item $B \rightarrow b$
\end{itemize}

\paragraph{Simboli irraggiungibili}

Ci possono essere dei simboli non terminali non raggiungibili dall'espansione della grammatica.
Esempio:

\begin{itemize}
  \item $S \rightarrow AB$
  \item $A \rightarrow a$
  \item $B \rightarrow b$
  \item $C \rightarrow c$
\end{itemize}

\paragraph{Simboli improduttivi}

Ci possono essere dei simboli non terminali che aggiungono un sacco di simboli ma non arrivano mai ad esaurirsi. (attenzione alla ricorsione)
Esempi:

\begin{itemize}
  \item $S \rightarrow AB$
  \item $A \rightarrow a$
  \item $B \rightarrow bB$
\end{itemize}

\paragraph{Cicli infiniti}

\begin{itemize}
  \item $S \rightarrow AB$
  \item $A \rightarrow a$
  \item $B \rightarrow bB|b|C$
  \item $C \rightarrow D$
  \item $D \rightarrow E$
  \item $E \rightarrow B$
\end{itemize}

\paragraph{Chiusura (punti fissi)}

Come con i punti fissi: $o, f(o), f(f(o)), f(f(f(o))) .... S \Rightarrow F(o) = S$.
Possiamo usare le chiusure per calcolare le proprieta' soprastanti.
