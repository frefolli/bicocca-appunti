\chapter{Algoritmo KMP}

\section{Definizioni}

\paragraph{Prefix Function} del pattern P di lunghezza m: $\phi : [0,m] \rightarrow [-1,m]$. $\phi(j) = |B(P[1,j])|$ se $j \geq 1$ altrimenti $\phi(j) = -1$. \`E la lunghezza del bordo del prefisso del pattern di lunghezza j.

\section{L'Algoritmo}

L'algoritmo ricalca quello banale a finestra per la ricerca esatta, con la differenza che la funzione $\phi$ viene utilizzata per fare un salto ottimizzato quando si ottiene un mismatch. Infatti in caso di mismatch:

\begin{itemize}
    \item $i$ \`e la posizione della finestra W
    \item $j$ \`e la posizione di mismatch sul pattern P
    \item $P[1,j-1]$ \`e il prefisso di match
    \item $i+j-1$ \`e la posizione del mismatch su T
    \item La nuova posizione della finestra \`e $i \equiv i + j - phi(j-1) - 1$
    \item Il confronto riparte dalla posizione $j \equiv phi(j-1) + 1$ sul pattern
    \item Il confronto riparte dalla posizione $k = i + j - 1$ sul testo
\end{itemize}

Un caso particolare \`e quando $j = m+1$ ovvero il confronto arriva oltre l'ultima posizione del pattern. Si restituisce $i$ come occorrenza del pattern e il confronto riparte da $i' = i + m - phi(m)$ sul testo e da $j' = phi(m) + 1$ sul pattern.

\section{Confronto}

\begin{center}
    \begin{tabular}{|c c c|}
        \hline
        Classe & ASF & KMP \\
        \hline
        Spazio & $O(m|\Sigma|)$ & $O(m)$ \\
        \hline
        Tempo & $O(m|Sigma| + n)$ & $O(n+m)$ \\
        \hline
        Preprocessing di P & $O(m|\Sigma|)$ & $O(m)$ \\
        \hline
        Scansione di T & $O(n)$ & $O(n)$ \\
        \hline
    \end{tabular}
\end{center}

\begin{itemize}
    \item Automa
    \begin{itemize}
        \item efficiente per pattern piccoli
        \item richiede più tempo e memoria per pattern grandi
        \item ricerca di P in testi diversi
    \end{itemize}
    \item KMP
    \begin{itemize}
        \item efficiente per pattern grandi
        \item richiede più tempo per pattern piccoli
    \end{itemize}
\end{itemize}
