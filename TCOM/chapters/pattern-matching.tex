\chapter{Pattern Matching}

\section{Programma}

\begin{itemize}
    \item Ricerca con Automa a Stati Finiti
    \begin{itemize}
        \item Ricerca esatta
        \item Confronto di simboli
        \item Preprocessing: pattern $\rightarrow \theta(m|S|)$
        \item Ricerca: scansione del testo $\rightarrow \theta(n)$
        \item Tempo totale $\rightarrow \theta(m|S|+n)$
    \end{itemize}

    \item Algoritmo di Knuth-Morris-Pratt (KMP)
    \begin{itemize}
        \item Ricerca esatta
        \item Confronto di simboli
        \item Preprocessing: pattern $\rightarrow \theta(m)$
        \item Ricerca: scansione del testo $\rightarrow \theta(n)$
        \item Tempo totale $\rightarrow \theta(m+n)$
    \end{itemize}

    \item Algoritmo di Baeza-Yates e Gonnet (BYG)
    \begin{itemize}
        \item Ricerca esatta
        \item Paradigma SHIFT-AND
        \item Preprocessing: pattern $\rightarrow \theta(|S|+m)$
        \item Ricerca: scansione del testo $\rightarrow \theta(n)$
        \item Tempo totale $\rightarrow \theta(|S|+m+n)$
    \end{itemize}

    \item Algoritmo di Wu-Manber (WM)
    \begin{itemize}
        \item Ricerca approssimata
        \item Paradigma SHIFT-AND
        \item Preprocessing: pattern $\rightarrow \theta(|S|+m)$
        \item Ricerca: scansione del testo $\rightarrow \theta(kn)$
        \item Tempo totale $\rightarrow \theta(|S|+m+kn)$
    \end{itemize}

    \item Ricerca con Suffix Array (SA)
    \begin{itemize}
        \item Ricerca esatta
        \item Confronto di simboli
        \item Preprocessing: testo (costruzione di SA)
        \item Ricerca $\rightarrow O(m log n)$
    \end{itemize}

    \item Ricerca con Burrows-Wheeler Transform/FM-index
    \begin{itemize}
        \item Ricerca esatta
        \item Preprocessing: testo (costruzione di BWT/FM-index)
        \item Ricerca $\rightarrow \theta(m)$
    \end{itemize}
\end{itemize}

\section{Definizioni}

\paragraph{Pattern Matching} cercare un motivo all'interno di un oggetto pi\`u o meno complesso.
\paragraph{Pattern Matching su Stringhe} cercare all'interno di un testo $T$ le occorrenze di un pattern $P$.
\paragraph{String Matching \textbf{Esatto}} cercare occorrenze esatte di P in T
\paragraph{String Matching \textbf{Approssimato}} cercare occorrenze approssimate di P in T

\paragraph{Stringa} giustapposizione (operatore $\ast$) di simboli su un alfabeto $\Sigma$ (lunghezza di $X = |X|$).
\paragraph{Stringa nulla $\epsilon$} stringa composta da zero simboli.
\paragraph{Sottostringa} $X[i,j] = X[i.j] = X[i:j] = X[i]X[i+1]...X[j]$, gli indici partono da $1$.
\paragraph{Sottostringa propria} $i \neq 1 \land j \neq |X|$.
\paragraph{Prefisso} $X[1,j]$
\paragraph{Suffisso} $X[i,|X|]$

\paragraph{Occorrenza Esatta} Una posizione $i$ del testo $T$ tale che $T[i, i+m-1] = P$ \`e un'occorrenza esatta di $P$ in $T$.
\paragraph{Occorrenza Approssimata} Una posizione $i$ del testo $T$ tale che esista almeno una sottostringa $S = T[i-L+1, i]$ tale che $ED(P, S) \leq k$, con $k$ soglia di errore, \`e un'occorrenza Approssimata di $P$ in $T$.
Nota bene: se $ED(P, S) \geq |P| - L$ allora $i$ non \`e mai un'occorrenza approssimata.

\paragraph{Match} Due simboli $\alpha,\beta$ sono un \textbf{match} sse $\alpha = \beta$
\paragraph{Mismatch} Due simboli $\alpha,\beta$ sono un \textbf{match} sse $\alpha \neq \beta$

\section{Algoritmo Banale per R.E.}

Uso una finestra $W$ di lunghezza $m = |P|$ che scorre su T da sinistra a destra. Il cursore scorre lungo la finestra e il pattern P verificando un match. In caso di mismatch la finistra si sposta a destra di $1$ e il cursore riparte dalla prima posizione della finestra $W$ e dal primo carattere del pattern $P$.

\begin{algorithm}
    \begin{algorithmic}
        \Procedure{TRIVIAL-EXACT-OCCURRENCES}{P, T}
            \State $n \gets |T|$
            \State $m \gets |P|$
            \State $i \gets 1$
            \While{$i \leq n-m+1$}
              \State $j \gets 1$
              \While{$P[i] = P[i+j-1] \land j \leq m$}
                \State $j \gets j+1$
              \EndWhile
              \If {j = m+1}
                \State output $i$
              \EndIf
            \EndWhile
        \EndProcedure
    \end{algorithmic}
\end{algorithm}

\section{Algoritmo Banale per R.A.}

Uso una finestra $W$ di lunghezza variabile $m \in [m-k, m+k]$ che scorre su T da sinistra a destra. La posizione iniziale della finestra \`e $i = m-k$, la lunghezza iniziale pure $m - k$. Se la finestra non evidenzia un'occorrenza approssimata di P in T allora scorro a destra di una posizione.

Fondamentalmente la differenza concettuale rispetto all'algoritmo banale della ricerca esatta \`e che la finestra \`e trascinata da destra invece che da sinistra, e allungata a sinistra ogni volta fino a che si pu\`o.

\begin{algorithm}
    \begin{algorithmic}
        \Procedure{TRIVIAL-APPROX-OCCURRENCES}{T, P, k}
            \State $n \gets |T|$
            \State $m \gets |P|$
            \State $i \gets m-k$
            \While{$i \leq n$}
              \State $L \gets m-k$
              \While{$L \leq m+k \land i-L+1 \geq 1$}
                \If {$ED(T[i-L+1, i], P) \leq k$}
                  \State output $i$
                \EndIf
                \State $L \gets L+1$
              \EndWhile
              \State $i \gets i+1$
            \EndWhile
        \EndProcedure
    \end{algorithmic}
\end{algorithm}
