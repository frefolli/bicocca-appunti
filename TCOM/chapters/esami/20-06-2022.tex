\chapter{20 Giugno 2022}

\section{Esercizio 5}

Dare la definizione di funzione di transizione per ricercare un pattern P in un testo T tramite automa a stati finiti. In particolare, specificare il dominio e il codominio (punti: 4). Siano poi P=ccaca e T=bccacaec un pattern e un testo definiti su alfabeto {a, b, c, d, e}.

Si chiede di:
\begin{itemize}
    \item specificare la funzione di transizione di P (punti: 2)
    \item mostrare l'automa a stati finiti per cercare P in T, evidenziando gli stati che identificano un'occorrenza di P in T; mostrare inoltre il calcolo delle posizioni di inizio di ciascuna occorrenza sulla base dello stato trovato (punti: 3)
\end{itemize}

NOTA BENE: tutte le notazioni usate nel rispondere alla domanda devono essere spiegate altrimenti la risposta non viene valutata.

\paragraph{soluzione} TODO:

\section{Esercizio 6}

Dare la definizione di parola Di dell'algoritmo di ricerca esatta basato su paradigma SHIFT-AND.

\paragraph{soluzione} TODO:

\section{Esercizio 7}

Definire il Suffix Array di un testo.

\paragraph{soluzione} TODO:
