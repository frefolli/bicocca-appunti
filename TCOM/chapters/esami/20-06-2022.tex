\chapter{20 Giugno 2022}

\section{Esercizio 5}

Dare la definizione di funzione di transizione per ricercare un pattern P in un testo T tramite automa a stati finiti. In particolare, specificare il dominio e il codominio.
Siano poi $P=ccaca$ e $T=bccacaec$ un pattern e un testo definiti su alfabeto $\{a, b, c, d, e\}$.

Si chiede di:
\begin{itemize}
    \item specificare la funzione di transizione di P
    \item mostrare l'automa a stati finiti per cercare P in T, evidenziando gli stati che identificano un'occorrenza di P in T; mostrare inoltre il calcolo delle posizioni di inizio di ciascuna occorrenza sulla base dello stato trovato
\end{itemize}

NOTA BENE: tutte le notazioni usate nel rispondere alla domanda devono essere spiegate altrimenti la risposta non viene valutata.

\paragraph{soluzione}

\begin{center}
  \begin{tabular}{|c c c c c c|}
    $\delta$ & a & b & c & d & e \\ \hline
    0 & 0 & 0 & 1 & 0 & 0 \\ \hline
    1 & 0 & 0 & 2 & 0 & 0 \\ \hline
    2 & 3 & 0 & 0 & 0 & 0 \\ \hline
    3 & 0 & 0 & 4 & 0 & 0 \\ \hline
    4 & 5 & 0 & 2 & 0 & 0 \\ \hline
    5 & 0 & 0 & 1 & 0 & 0 \\ \hline
  \end{tabular}
\end{center}

Lo stato che identifica un'occorrenza \`e lo stato $5$. L'occorrenza si trova in posizione $i - m + 1 = i - 4$, con $i$ ultimo carattere letto del testo che ha portato allo stato $5$.

\section{Esercizio 6}

Dare la definizione di parola Di dell'algoritmo di ricerca esatta basato su paradigma SHIFT-AND.

\paragraph{soluzione} L'ho gi\`a fatto in un altro esercizio: vattelo a cercare con CTR+F.

\section{Esercizio 7}

Definire il Suffix Array di un testo.

\paragraph{soluzione} L'ho gi\`a fatto in un altro esercizio: vattelo a cercare con CTR+F.
