\chapter{16 Febbraio 2022}

\section{Esercizio 8}

Si dia la definizione di parola $D^h_i$ dell'algoritmo di Wu e Manber, specificando il significato dell'apice h e del pedice i (NB: tutte le notazioni utilizzate devono essere adeguatamente spiegate). Fare un esempio per un testo e un pattern a piacere. Specificare inoltre (motivando adeguatamente la risposta) la parola $D^4_0$ per un generico pattern lungo 8 simboli.

\paragraph{soluzione} TODO:

\section{Esercizio 9}

Durante l'esecuzione dell'algoritmo KMP la finestra si trova in posizione 20 e il primo mismatch tra pattern e testo si trova in posizione 24 sul testo.
Specificare la posizione sul testo da cui riparte il confronto dopo lo spostamento della finestra e perche' non si puo' derivare (con i dati a disposizione) la corrispondente posizione sul pattern (da cui riparte il confronto).

\paragraph{soluzione} TODO:

\section{Esercizio 10}

Definire l'FM-index di un testo (NB: tutte le notazioni utilizzate devono essere adeguatamente spiegate) specificando dominio e codominio delle funzioni.

\paragraph{soluzione} TODO:
