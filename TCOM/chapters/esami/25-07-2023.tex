\chapter{25 Luglio 2023}

\section{Esercizio 1}

Descrivere l'algoritmo di ricerca esatta di un pattern in un testo tramite il paradigma SHIFT-AND, cio\`e descrivere la fase di preprocessing e la fase di scansione.

\paragraph{soluzione} TODO:

\section{Esercizio 2}

L'esecuzione della ricerca esatta di P = abcaac in testo T con automa a stati finiti passa dallo stato 4 allo stato 2 dopo avere letto un certo simbolo $\sigma$ di T.
Specificare, motivando la risposta, che simbolo \`e $\sigma$.

\paragraph{soluzione} TODO:

\section{Esercizio 3}

Sia data la BWT $B = cc\$aaab$ di un testo. Indicare, motivando la risposta, qual \`e il valore j restituito per $i = 4$ dalla LF-function $j = LF(i)$.
(si ricorda che $j = LF(i)$ \`e la posizione nel Suffix Array del suffisso del testo che inizia con il simbolo $B[i]$)

\paragraph{soluzione} TODO:
