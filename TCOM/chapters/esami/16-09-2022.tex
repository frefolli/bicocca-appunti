\chapter{16 Settembre 2022}

\section{Esercizio 5}

Dare la definizione di funzione di transizione per la ricerca esatta di un pattern P in un testo T tramite automa a stati finiti. In particolare, specificare il dominio e il codominio.
Si richiede inoltre di spiegare come funziona l'algoritmo di ricerca di P in T utilizzando un esempio di pattern e testo.

\paragraph{soluzione} L'ho gi\`a fatto in un altro esercizio: vattelo a cercare con CTR+F.

\section{Esercizio 6}

Dare la definizione di parola $Di$ dell'algoritmo di Baeza-Yates-Gonnet.

\paragraph{soluzione} L'ho gi\`a fatto in un altro esercizio: vattelo a cercare con CTR+F.

\section{Esercizio 7}

Definire l'operazione di backward extension di un Q-intervallo.

\paragraph{soluzione}

Detto $Q$ una sottostringa del pattern P, il Q-intervallo \`e l'intervallo di indici sul Suffix Array e sulla BWT i quali suffissi corrispondenti hanno come prefisso Q. L'operazione di backward extension consiste nel calcolare l'intervallo per $\sigma Q$, con $\sigma \in \Sigma$.
Questa operazione pu\`o essere fatta in tempo lineare individuando due indici $i_1, i_k$ rispettivamente la prima e l'ultima posizione in cui $BWT[i] = \sigma$ (ricordardo che il simbolo nella BWT \`e il carattere che precede il primo carattere del suffisso corrispondente all'indice del Suffix Array).
Quindi si utilizza il Last-First Mapping per individuare il nuovo intervallo $[b', e') = [LF(i_1), LF(i_k)+1)$.

In alternativa utilizzando la FM-Index \`e possibile effettuare questa operazione in tempo costante con $[b', e') = [Occ(b, \sigma) + C(\sigma) + 1, Occ(e, \sigma) + C(\sigma) + 1)$.
