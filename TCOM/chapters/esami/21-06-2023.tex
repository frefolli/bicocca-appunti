\chapter{21 Giugno 2023}

\section{Esercizio 1}

Spiegare come avviene la ricerca esatta di P in T con automa a stati finiti, supponendo di avere già definito la funzione di transizione $\delta$.
Specificare in particolar modo come vengono identificate le occorrenze.

\paragraph{soluzione} TODO:

\section{Esercizio 2}

L'esecuzione della ricerca esatta di $P = bbbbbacca$ in un testo T con l'algoritmo KMP si trova con la finestra W in posizione 32.
Il più lungo prefisso di P che occorre in posizione 32 \`e composto da 4 simboli. Calcolare la successiva posizione della finestra e le posizioni dei simboli su T e su P da cui riparte il confronto.
Spiegare il procedimento usato.

\paragraph{soluzione} TODO:

\section{Esercizio 3}

Dare la definizione di Last-First Function $j = LF(i)$, che realizza la proprietà di Last-First mapping della BWT, spiegando il significato di $i$ e $j$.

\paragraph{soluzione} TODO:
