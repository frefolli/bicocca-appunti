\chapter{27 gennaio 2023}

\section{Esercizio 1}

Dare la definizione di funzione di transizione per la ricerca esatta tramite automa a stati finiti specificando in particolare dominio e codominio.

\paragraph{soluzione}

La funzione di transizione dell'ASF \`e $\delta \; : \; [0, |P|] \times \Sigma \rightarrow [0, |P|]$. Il suo valore \`e $\delta(q, P[q]) = q+1$ per i match e $\forall \sigma \neq P[q] \; \Rightarrow \; \delta(|B(P[q]\sigma)|)$ per i mismatch.

\section{Esercizio 2}

Scegliere un testo e un pattern e fornire su di essi un esempio di parola $D^1_5$ di Wu e Manber e spiegare cosa rappresenta.

\paragraph{soluzione} Prendo $T = abcdabcdaa$ e il pattern $abcdd$.

$D^1_5$ \`e la word dove $D^1_5[j] = 1$ sse $P[1,j] = suff_1(T[1,5])$, ovvero se il j-esimo prefisso di P \`e uguale ad un suffisso di T[1,5] con al pi\`u un errore.

In questo caso $T[1,5] = abcda$ e $D^1_5 = 11001$, ovvero $abcda$ ha un match con $abcdd$ con $ED(p,t) \leq 1$.
In particolare $D^1_5[1] = 1$ perch\`e $P[1,1] = T[5,5]$ e $D^1_5[2] = 1$ perch\`e $ED(P[1,2], T[5,5]) = 1 \leq 1$.

\section{Esercizio 3}

Scegliere un testo e fornire la funzione Occ del suo FM-index, spiegando poi cosa rappresenta.

\paragraph{soluzione} Scelgo il testo $T = aabac$ con alfbeto $\Sigma = \{\$, \; a, \; b, \; c\}$.

Il SA \`e $\{6 \; 1 \; 2 \; 4 \; 3 \; 5\}$ e quindi la BWT \`e $\{a \; a \; b \; c \; a \; \$\}$ e la F \`e $\{\$ \; a \; a \; a \; b \; c\}$.

Dalla BWT calcolo la funzione Occ:

\begin{center}
  \begin{tabular}{|c | c | c | c | c|}
    \hline
    $i$ & \$ & a & b & c \\ \hline
    1 & 0 & 0 & 0 & 0 \\ \hline
    2 & 0 & 1 & 0 & 0 \\ \hline
    3 & 0 & 2 & 0 & 0 \\ \hline
    4 & 0 & 2 & 1 & 0 \\ \hline
    5 & 0 & 2 & 1 & 1 \\ \hline
    6 & 0 & 2 & 1 & 1 \\ \hline
    7 & 1 & 3 & 1 & 1 \\ \hline
  \end{tabular}
\end{center}

La funzione $Occ \; : \; [1,n+1] \times \Sigma \rightarrow [0,n]$ nella sua applicazione $Occ(i, \sigma)$ ha come valore la quantit\`a di simboli $\sigma$ nell'intervallo $BWT[1,i-1]$ della BWT.
Dall'applicazione parziale di questa funzione $Occ(7, \sigma)$ si ottiene la funzione $C$:

\begin{itemize}
  \item $C(\$) = 0$
  \item $C(a) = 1$
  \item $C(b) = 4$
  \item $C(c) = 5$
\end{itemize}

Dove $C(\sigma)$ indica rispettivamente la quanti\`a di simboli $\beta < \sigma$ presenti nella BWT.
