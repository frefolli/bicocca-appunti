\chapter{BWT}

\section{Burrows-Wheeler Transform}

\`E una struttura dati che rappresenta una permutazione reversibile di un T in $\theta(n\;log\Sigma)$ spazio.

La costruzione della BWT coincide con l'ordinamento di tutte le permutazioni del testo T in ordine lessicografico.

\paragraph{definizione} una permutazione $T_q$ \`e uguale alla concatenazione di $T[1,q-1] \ast T[q,|n|]$.
Di conseguenza $T[q-1]$ \`e il primo simbolo di $T_q$, e $T[q-1 ,|T|]$ contiene $T[q, |T|]$.

Formalmente \`e l'array tale che $\forall i \in [1,|T|]$, $BWT[i] = \sigma$ sse $T_q$ \`e la i-esima rotazione in ordine lessicografico di T e $\sigma$ \`e l'ultimo carattere di $T_q$, ovvero $T[q-1] = \sigma$ (con q = 1, $T[n] = \sigma$).

\paragraph{definizione} $F$ \`e l'array con l'ordine lessicografico dei simboli di T. Servir\`a dopo.

\subsection{Esempio}

\putimagebig{images/8.png}{Esempio di BWT}{png:8}

\section{Reversibile}

\`E possibile riottenere il testo T a partire dalla BWT tramite l'array F. Visto che la BTW gli ultimi simboli delle permutazioni di T, allora $F[i] = T[q] \Rightarrow BWT[j] = T[q-1]$.

\begin{itemize}
  \item Si inizializza il puntatore $i = 1$ sulle prime posizioni dei due array
  \item Siccome $\$ = min \Sigma$ allora $BWT[1] = T[n]$, in ogni caso
  \item ad ogni passo scrivo all'indietro $BWT[i]$ sul testo da riempire T e $i$ diventa la posizione su F della q-esima occorrenza di $BWT[i]$ ($\sigma = BWT[i]$ \`e la q-esima occorrenza di $\sigma$ in BTW).
  \item si continua cos\`i fino a che $BWT[i] = \$$, in quel caso il testo \`e terminato.
  \item come metodo di verifica della correttezza di una BWT semplicemente non si deve ottenere $\$$ prima di aver riempito $|T|$ caselle di T.
\end{itemize}

Come nota: la corrispondenza della q-esima occorrenza di $\sigma$ su F con la q-esima occorrenza di $\sigma$ su BWT si chiaman \textbf{Last-First Mapping}.

\section{Calcolo di BWT da SA}

Se conosco il testo T posso passare da BWT e SA agilmente.

\paragraph{definizione} Visto che SA lista gli indici dei simboli in F, si pu\`o costruire la BWT come $\forall i \in [1, |T|]$, $BWT[i] = T[SA[i] - 1]$.

\section{Q-Intervallo}

\paragraph{Q} \`e una sottostringa di P (nell'algoritmo generalmente un suffisso)
\paragraph{Q-Intervallo su BWT} \`e l'intervallo di posizioni $[b,e)$ sulla BWT tali che i suffissi successivi ai simboli nella BWT condividano strettamente il prefisso Q
\paragraph{Q-Intervallo su SA} \`e l'intervallo di posizioni $[b,e)$ sul SA tali che i suffissi corrispondenti condividano strettamente il prefisso Q
\paragraph{il numero di occorrenze} di Q in T \`e uguale a $(e-b)$.

\subsection{Esempio}

\putimagebig{images/9.png}{Esempio di Q-intervalli}{png:9}

\section{Backward Extension}

La \textbf{backward extension} di un Q-intervallo $[b, e)$ con un simbolo $\sigma$ \`e il $\sigma Q$-intervallo. Ricavarlo \`e semplice perch\`e la BWT contiene l'informazione del simbolo precedente al suffisso i-esimo. Quindi per ogni posizione dell'intervallo $[b,e)$ dove $BWT[i] = \sigma$ si usa il \textbf{Last First Mapping} per risalire all'indice nel SA del suffisso che inizia con quel simbolo. Il nuovo intervallo $[b',e')$ e' formato dagli estremi indici ricavati in questo modo.

\begin{itemize}
    \item $b' = LF($ pi\`u piccola posizione in $[b, e]$ tale che $B[i] = \sigma$)
    \item $e' = LF($ pi\`u grande posizione in $[b, e]$ tale che $B[i] = \sigma$) + 1
\end{itemize}

\subsection{Esempio}

\putimagebig{images/10.png}{Esempio di Backward Extension}{png:10}

\section{Algoritmo di Ricerca}

\begin{itemize}
    \item Si parte con il suffisso vuoto del pattern P, $Q \equiv \epsilon$, il Q-intervallo \`e $[1,n+1)$.
    \item Quindi si inizializza $i = |P|$
    \item ad ogni passo si, $P[i] = \sigma$ viene concatenato a Q e si effettua la backward extension calcolando il Q-intervallo di $\sigma Q$.
    \item se il Q-intervallo diventa l'intervallo vuoto allora P non ha occorrenze in T.
    \item una volta che $Q = P$ se l'intervallo \`e non vuoto allora ci sono $(e-b)$ occorrenze in T, e gli indici sul SA indicano le posizioni dei suffissi su T, ovvero delle occorrenze su T.
\end{itemize}

Il tutto in tempo $O(n\,m)$, ma si pu\`o ottimizzare ulteriormente.

\subsection{Esempio}

\putimagebig{images/11.png}{Esempio di Ricerca con BWT e SA}{png:11}
