\chapter{Introduzione alla crittografia}

\section{Introduzione}

La \textbf{crittografia} \`e lo studio di quelle tecniche atte a immagazzinare, processare, trasmettere e in generale proteggere un'informazione su un canale non sicuro. Comprende sia tecniche per nascondere la natura del messaggio che tecniche per proteggerlo da modifiche.

Non va confusa con la \textit{steganografia}, che invece \`e un insieme di tecniche per nascondere il messaggio in altri media.

La crittografia studia sia le implementazione di crittosistemi a fini di proteggere le comunicazioni, sia le loro vulnerabilit\`a per sfruttarle o per rafforzare i suddetti sistemi.

\putimage{images/crittografia/01-01.png}{Modello di Comunicazione}{png:01-01}

In Figura \ref{png:01-01} \`e riportato una rappresentazione del dominio. Alice vuole inviare un messaggio segreto a Bob e per evitare che Eve lo intercetti lo cifra (\`e necessario che Bob conosca un algoritmo per decifrare il messaggio). Eve pu\`o essere semanticamente sia un malintenzionato che un agente della Polizia Postale, quindi \`e uno schema molto generale.

Le capacit\`a di Eve possono variare in base alle assunzioni che si fanno: pu\`o scrivere e/o leggere, magari nessuno dei due, e tendenzialmente dispone di capacit\`a di computazione arbitraria (generalmente si assume che disponga di una Macchina di Turing Probabilistica).

\paragraph{Kerkhoff's principles}

Alice e Bob utilizzano due funzioni $E, D$ per cifrare, decifrare ma la segretezza del sistema non dovrebbe risiedere nelle funzioni in se ma in una quantit\`a di informazione chiamata \textbf{chiave}. Ovvero visto che cambiare chiave \`e pi\`u semplice che cambiare algoritmi la variabilit\`a della cifratura deve risiedere nella variabilit\`a della chiave.

Definiamo quindi un crittosistema come una tupla $CS = (PT, CT, K, E, D)$.

\begin{itemize}
  \item PT \`e lo spazio dei messaggi in chiaro
  \item CT \`e lo spazio dei messaggi cifrati
  \item K \`e lo spazio delle chiavi
  \item $E : PT \times K \rightarrow CT$ \`e la funzione di cifratura
  \item $D : CT \times K \rightarrow PT$ \`e la funzione di decifratura
\end{itemize}

\paragraph{Alcune considerazioni}

Lo spazio delle chiavi deve essere abbastanza grande da impedire in pratica un attacco bruteforce. Deve essere \textbf{semplice} (tempo polinomiale da una DTM) costruire un messaggio cifrato e \textbf{estremamente difficile} (tempo non polinomiale da una DTM) effettuare l'operazione inversa senza conoscere la chiave.

\paragraph{Tipi di Attacchi}

In generale distinguiamo tra queste categorie in base alle capacit\`a di Eve:

\begin{itemize}
  \item \textbf{ciphertext only}: Eve conosce solo $c$ e vuole computare il messaggio $m$.
  \item \textbf{known plaintext}: Eve anche alcune coppie $(m, c)$.
  \item \textbf{chosen plaintext}: Eve ha accesso alla cifratura di messaggi.
  \item \textbf{chosen ciphertext}: Eve ha accesso alla decifratura di messaggi.
\end{itemize}

Inoltre un crittosistema pu\`o essere considerato \textbf{parzialmente rotto} se possiamo porre alcuni limiti ai possibili valori della chiave riducendo lo spazio su cui fare il bruteforce; \textbf{completamente rotto} se possiamo ricavare direttamente $m$ e/o $k$ in modo arbitrario.

\paragraph{Simmetria e Asimmetria}

Nel contesto della crittografia in essere consideriamo \textbf{simmetrico} un crittosistema in cui la chiave di cifratura \`e uguale a quella di decifratura, \textbf{asimmetrico} il contrario. Una definizione pi\`u precisa vuole che la chiave di decifratura sia facile da ottenere a partire dalla chiave di cifratura.

Nei crittosistemi che non dipendono da una chiave esplicita di solito si considera la \textit{chiave} come la funzione di cifratura/decifratura e vice versa.

\section{Cifrari a Sostituzione}

Un cifrario a sostituzione prevedere la sostituzione (sorpreso?) di una o pi\`u lettere con un altro set di lettere sulla base di una chiave $K$.

Sono detti \textbf{monoalfabetici} i sistemi che cifrano una lettera del messaggio sempre con la stessa lettera del messaggio cifrato.

\subsection{Cifrario di Cesare}

\`E un cifrario simmetrico a sostituzione monoalfabetica dove presa una chiave $K \in [0, |\Sigma|]$ si shifta a sinistra (o a destra) ogni simbolo del messaggio. \`E celebre la versione con $K = 3$.

% TODO: aggiungi esempio.

\paragraph{Problemi}

Come si \`e intuito il problema principale \`e lo spazio delle chiavi troppo piccolo che rendere possibile un attacco bruteforse.

\paragraph{Osservazioni}

Essendo un cifrario lineare monadico, le operazioni sono commutative e simmetriche.

\subsection{Crittanalisi dei Sistemi Monoalfabetici}

In generale questi sistemi sono molto vulnerabili, specie se il messaggio che si deve de/cifrare appartiene al linguaggio naturale (e in generale a qualche linguaggio strutturato). Nel primo caso spesso si utilizzano le frequenze delle lettere che variano di lingua in lingua e che permettono di ricavare facilmente spesso la chiave tramite l'assunzione che nei sistemi monoalfabetici i simboli associati alle lettere con maggior frequenza si ripetono a loro volta con maggior frequenza.

\paragraph{Contrattacco}

A volte si inseriscono delle lettere a bassa frequenza nel messaggio in modo strutturato per creare rumore. Pi\`u efficace \`e invece il fatto di associare alle lettere pi\`u simboli e scegliere randomicamente quale usare a runtime. Esempio: la lettera E ha 12\% di frequenza? bene, allora scelgo 12 simboli dall'alfabeto CT e li associo ad E. La scelta del simbolo da usare pu\`o anche essere non casuale per interferire con le frequenze dei simboli.

\subsection{Cifrario di Hill}

\`E un crittosistema polialfabetico che sfrutta l'algebra lineare per cifrare un messaggio. La chiave \`e una matrice $M$ che viene moltiplicata ad un messaggio disposto in forma matriciale. Chiaramente la chiave di decifrazione \`e $M^{-1}$
\`E il primo crittosistema lineare a blocchi che vedremo.

\putimage{images/crittografia/01-02.png}{Esempio}{png:01-02}

\paragraph{Problemi}

Per cominciare creare una chiave non \`e semplicissimo perch\`e ci si deve limitare all'uso di matrici invertibili (per ovvie ragioni).

Inoltre essendo un sistema lineare potrei creare una permutazione che invalida il messaggio o lo altera in qualche modo.

\subsection{Cifrario di Playfair}

La chiave \`e costruita prendendo le lettere dell'alfabeto e disponendole in una matrice 5x5 in modo casuale.

\putimage{images/crittografia/01-03.png}{Esempio}{png:01-03}

Quindi la cifratura avviene dividendo il messaggio in blocchi da due lettere(senza blocchi con due lettere uguali) e se le due lettere sono nella stessa riga nella chiave allora esse vengono shiftate verso destra (sempre nella chiave). Se sono nella stessa colonna vengono mosse verso il basso.
Altrimenti visto che formano un rettangono di cui sono gli angoli, si considerano gli altri due angoli e sostituiscono con quelli.

La decifratura avviene eseguendo le operazioni senso opposto.

\paragraph{Problemi}

A parte il fatto che la chiave ha una struttura regolare ed \`e corta, quindi \`e piuttosto semplice da indovinare, il problema principale consiste nel fatto che ogni blocco viene cifrato sempre nello stesso modo nel messaggio. Lasciando quindi spazio ad attacchi sulle frequenze su digrammi.

\subsection{Cifrario di Vigenere}

\`E un cifrario di Cesare con una chiave di lunghezza $|K| > 1$, eventualmente lunga quanto il plaintext (ma pu\`o essere ripetuta ciclicamente per poter cifrare tutto).

\paragraph{Problemi}

\`E gi\`a pi\`u robusto ma si possono fare delle assunzioni grazie al fatto che se la chiave \`e pi\`u corta del messaggio allora pi\`u lettere saranno cifrate con la stessa chiave. Una volta indovinata la lunghezza della chiave si pu\`o sfruttare questa assunzione per fare un attacco basato su frequenze.

Per trovare la lunghezza della chiave si pu\`o usare il metodo di Kasisky che consiste nel notare ripetizioni di sequenze di lettere nel ciphertext e usare le distanze relative per identificare i possibili candidati per la lunghezza della chiave.

\putimage{images/crittografia/01-04.png}{Tabella di Appoggio}{png:01-04}

\subsection{Macchina Enigma}

\`E una macchina dotata di rotori che applicano sequenzialmente una sostituzione monoalfabetica. Dopo aver cifrato ogni lettera del messaggio i rotori compiono uno scatto di una posizione come un orologio, rendendolo complessivamente polialfabetico. La chiave \`e la posizione iniziale dei rotori.

La decifrazione avviene ponendo la stessa posizione iniziale e grazie ad un riflettore posto alla fine dell'ultimo rotore.

\putimage{images/crittografia/01-05.png}{Struttura}{png:01-05}

\paragraph{Problemi}

Il sistema \`e molto robusto e la tecnica di Kasisky non pu\`o essere applicata essendo il ciphertext un prodotto di sostituzioni polialfabetiche.

Prima dei rotori c'\`e anche una plugboard che effettua una ulteriore sostituzione, ma l'effetto pu\`o essere annullato usando due macchine enigma in serie.

Alla fine il sistema \`e stato rotto costruendo pi\`u copie della macchina enigma in parallelo che effettuano un attacco bruteforce su un sottospazio delle chiavi.

Un ulteriore limite alle chiavi \`e il fatto che $c \ne m$ by design.
