\chapter{AES e DES}

\section{DES}

DES \`e il primo crittosistema standard. Utilizza una chiave da 56 bit, interleaved con un bit di non parit\`a in terminazione ad ogni byte, ma noi consideriamo solo i 56 bits.

\paragraph{Generazione delle Chiavi}

Alla chiave originale viene applicata una permutazione e si divide il blocco di bit in due blocchi da 28 bit ciascuno $C_0, D_0$.

A partire da questi si ricavano ad ogni iterazione $C_n, D_n$ da $C_{n-1}, D_{n-1}$ tramite uno shift a sinistra seguendo una tabella che indica 1 o 2 shift in base all'iterazione.

La chiave dell'i-esima iterazione $K_i$ \`e ottenuta tramite una permutazione che seleziona 48 bit a partire dalla concatenazione $C_iD_i$

\paragraph{Cifratura}

Il messaggio \`e diviso in blocchi da 64 bits ciascuno, e ogni blocco viene permutato e splittato in due blocchi da 32 bits $L_0, R_0$.
Ad ogni iterazione si calcolano $L_n = R_{n-1}$ e $R_{n} - L_{n-1} \xor f(R_{n-1}, K_n)$. Alla fine si applica l'inverso della permutazione iniziale.

La decifrazione segue i passi al contrario.

\paragraph{Funzione $f$}

Il blocco da 32 bits in ingresso \`e espanso a 48 bit secondo una matrice di permutazione e viene xorato con la chiave. Il risultato viene diviso in blocchi da 6 bit e ogni blocco viene trasformato in un blocco di 4 bit tramite delle S-boxes. Il valore da sostituire viene triangolato tramite $x, y$ che sono rispettivamente \textit{primo e ultimo bit}, e i 4 bit interni.

\pragragraph{Pregi e Difetti}

\`E molto veloce (implementato con hardware dedicato per via delle operazioni sui singoli bit) e opera con permutazioni per introdurre non linearit\`a. La versione corrente resiste alla crittoanalisi lineare e differenziale.

Ma ha una sicurezza di soli 56 bits e negli anni 90 gi\`a era stata costruita una macchina per romperlo. Non \`e pi\`u considerato sicuro.

Poi ci sono altri problemi, come chiavi omomorfiche o complementari che permettono di ricostruire il messaggio originale o lo mantengono intatto. (vedi 3DES).

\paragraph{3DES}

Per aumentare la sicurezza del DES si \`e pensato di concatenarlo 3 volte cifrando/decifrando (alternatamente) 3 volte il messaggio con 1, 2 o 3 chiavi diverse in base allo schema. Ma la sicurezza in termini di bit non supera i 112.

\section{AES}

AES \`e la seconda generazione di crittosistemi standard pensato per essere pi\`u robusto e versatile (utilizza chiavi da 128 fino a 256 bits) e per essere pi\`u efficiente, specie quando implementato come software (opera su byte invece che su bits).

Noi analizziamo l'AES-128.

\paragraph{State}

L'algoritmo mantiene uno state di 128 bits (16 bytes) che inizialmente contiene il messaggio originale e alla fine contiene il messaggio cifrato. Questo stato \`e rappresentato come matrice e si fanno operazioni su questa rappresentazione come rotazioni e shifting.

\paragraph{Operazioni}

\begin{itemize}
  \item State = m
  \item AddRoundKey
  \begin{itemize}
  \item SubBytes
  \item ShiftRows
  \item MixColumns
  \item AddRoundKey
  \end{itemize}
  \item SubBytes
  \item ShiftRows
  \item MixColumns
\end{itemize}

\subparagraph{AddRoundKey}

Computa lo xor tra lo state e la chiave derivata.

\subparagraph{SubBytes}

Applica una trasformazione non lineare invertibile, che corrisponde all'inverso nel campo polinomiale $GF(2^8)$ sul polinomio irriducibile $x^8+x^4+x^3+x+1$.

Questo \`e equivalente ad utilizzare ancora una volta delle S-boxes che precalcolano il risultato.

\subparagraph{ShiftRows}

\`E una rotazione verso sinistra delle righe di 0, 1, 2 o 3 posizioni in base all'indice della riga.

\subparagraph{MixColumns}

Moltiplica lo state per una matrice invertibile nel campo $GF(2^8)$.

\paragraph{Chiavi Derivate}

Si necessita di 11 chiavi. Si inizializza un array di 10 word a 4 bytes con alcune costanti. Quindi si applicano rotazioni a sinistra all'interno della word e sostituzioni dei byte nella word con SubBytes.

\paragraph{Sicurezza}

AES adotta la strategia \textit{wide trail} che rende le crittoanalisi lineari e differenziali difficili da applicare. L'unica parte non lineare sono le S-box.
