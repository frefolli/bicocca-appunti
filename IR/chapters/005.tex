\chapter{IR Models}

The role of an IR system is to assess \textbf{relevance} (topical).
There isn't a single model used in practice, but a mixture of more models.

\section{Boolean Model}

It is based on set theory.
A document is formally represented as a set of index terms and binary weights are associated with those index terms: $R(d_j) = { t_j | w_{ij} == 1 | \forall wij \in {0,1}}$.

A query is formally represented as a Boolean expression on terms.
\begin{itemize}
  \item AND is Set Intersection
  \item OR is Set Union
  \item NOT is Set Difference
  \item ADJ/NEAR is a Context search
\end{itemize}
The matching mechanism applies set operations.
Relevance is modeled as a binary property of documents (Retrieval Status Value either 0 or 1).

When performing exact matching, the representation is inverted ($R(d_1) = {t_1, t_2, t_3} \rightarrow R(t_1) = {d_1, d_2}$) so that i can apply set operations on $R(t_j)$ values.

Queries are performed with Lazy Evaluation (TODO: recursive and dynamic bottom-up matching algorithm).

\section{Vector Space Model}

It's based on n-dimensional vector space (where n is the size of the  vocabulary).
I assume linear independence of terms to simplify computation.
A document is a linear combination of the terms which appear: $d_j = \Sigma ^{N} _{i = 1} w_{ij} * t_i$.
Without the assumption of term independence, the cosine similarity of terms should be computed.
Queries that want to elaborate OR, AND and so on, must use distinct queries and then join the results.

\section{Binary Independence Model}

The Odds of an event A is the ratio $\frac {P(A)} {1 - P(A)}$.
Search engines can rely on \textit{relevance feedback} (the user marks the found documents with whether they were relevant or not).
Some assumptions: documents and queries are represented as binary vectors, and query terms / documents are assumed to be Bayes independent.

The probability that the document $x$ is relevant with this query $q$: $P(R = 1|x) = \frac {P(x|R = 1, q)P(R = 1, q)} {P(x, q)}$.
Then using the concept of odds we compute the Retrieval Status Value: $RSV(x, q) = \frac {P(R = 1|x)} {P(R = 0|x)}$.
The initial rank uses $P(x|R=1)=0.5$ and $P(x=R=0)=\frac {n}{N}$ (the rate of appearance for that term in documents of the collection).
Then the probabilities are improved with $P(x|R=1)=\frac {V_t} {V}$ (the number of truly relevant documents over the number of documents) and $P(x=R=0)=\frac {n - V_t}{N - V}$.
