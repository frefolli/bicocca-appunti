\chapter{Scrum}

\section{I principi}

Si basa su tre principi:
\begin{itemize}
  \item \textbf{Visibilit\`a}
  \begin{itemize}
    \item gli aspetti che condizionano il risultato (quali feature sono implementate, quali sono testate, quali sono da modificare) devono essere chiaramente visibili al \textit{Direttorato} cosicch\`e possa prendere decisioni consapevolmente.
    \item \`E altres\`i importante dare definizioni chiare, per sapere per esempio cosa si intende con il termine 'eseguit\`o. Alcuni esempi di definizioni:
    \begin{itemize}
      \item I tests sono stati scritti e sono tutti passati.
      \item I test di analisi statica sono tutti passati.
      \item Il codice \`e stato riorganizzato e rivisto con successo.
      \item Il software \`e stato dispiegato nell'ambiente di Messa in Scena (o Produzione).
      \item La documentazione \`e disponibile in Italiano e in Inglese.
      \item ... etc
    \end{itemize}
  \end{itemize}
  \item \textbf{Ispezione}
  \begin{itemize}
    \item Tutti gli aspetti del processo di sviluppo devono essere ispezionati per identificare i problemi
    \item Le frequenze di ispezione possono dipendere dalla frequenza di aggiornamento/pubblicazione/generazione di artefatti, rapporti, oggettistica.
  \end{itemize}
  \item \textbf{Adattamento}
  \begin{itemize}
    \item Se c'\`e qualche problema, il \textit{Direttorato} deve prendere immediati provvedimenti per ricondurre alla socialista via sia il Software che il Processo e risolvere i problemi nel pi\`u breve tempo possibile.
  \end{itemize}
\end{itemize}

\section{Il processo}

\putimage{images/processo-scrum.png}{Il processo Scrum}{img:processo-scrum}

I Punti cardine del processo sono:
\begin{itemize}
  \item Iterazioni brevi di 1/4 settimane
  \item Mantenimento di una lista degli Arretrati.
  \item Un incontro Scrum giornaliero.
  \item Al termine di ogni iterazione si ha un prodotto incrementalmente costruito e immediatamente dispiegabile.
\end{itemize}

\section{I ruoli}

I componenti del Direttorio assumono ruoli ben definiti:
\begin{itemize}
  \item \textbf{Rappresentante del Popolo} (RP)
  \begin{itemize}
    \item \`E responsabile della massimizzazione del valore fornito dalla \textit{squadra} assicurando che gli arretrati della squadra siano allineati con le esigenze del Popolo.
  \end{itemize}
  \item \textbf{Squadra} (SQ)
  \begin{itemize}
    \item sviluppa in modo incrementale e autonomo la funzionalit\`a in base alle direzioni del \textit{Rappresentante del Popolo}.
  \end{itemize}
  \item \textbf{Gran Maestro del Socialismo Internazionale} (GMSI)
  \begin{itemize}
    \item \`E responsabile del processo di sviluppo, insegna e adatta Scrum alle esigenze della Sezione, supervisiona il comportamento di tutti i partecipanti al processo Scrum.
  \end{itemize}
  \item Preferibilmente sono interpretati da persone diverse
  \item I ruoli possono essere riassegnati tra le iterazioni
\end{itemize}

Nel presente documento si fa anche riferimento a ulteriori ruoli che sono \textit{esterni} al Direttorio:
\begin{itemize}
\item \textbf{Popolo}
  \begin{itemize}
    \item \`E il gruppo dei portatori di interesse e dei proprietari dei mezzi di produzione, rappresentato dal \textit{RP}.
  \end{itemize}
\end{itemize}

\section{Gli arretrati}

\putimage{images/arretrati.png}{Gli arretrati}{img:arretrati}

\section{Gli incontri}

\subsection{Gli incontri di pianificazione}

Ogni incontro dovrebbe essere diviso in parti consecutive:

\begin{itemize}
  \item Stabilire gli obiettivi dell'iterazione (4 ore)
  \begin{itemize}
    \item Partecipano il \textit{Rappresentante del Popolo}, la \textit{Squadra} e il \textit{Gran Maestro del Socialismo Internazionale}.
    \begin{itemize}
      \item Possono essere invitati esperti della Direzione Politica di Stato.
    \end{itemize}
    \item il \textit{RP} prepara l'elenco degli Arretrati di Produzione prima della riunione.
    \item il \textit{RP} presenta le voci degli Arretrati con la priorit\`a pi\`u alta.
    \item la \textit{Squadra} pone domande su contenuto, scopo, significato, …
    \item Prima che trascorrano le 4 ore, la \textit{Squadra} seleziona la funzionalit\`a del prodotto che pu\`o essere impegnata entro la fine della prossima iterazione.
    \item Le attivit\`a possono essere un mix di epiche e storie, a seconda delle priorit\`a e della pertinenza delle attivit\`a.
  \end{itemize}
  \item Perfezionare l'evasione degli obiettivi (4 ore)
  \begin{itemize}
    \item Partecipano il \textit{Rappresentante del Popolo}, la \textit{Squadra} e il \textit{Gran Maestro del Socialismo Internazionale}.
    \begin{itemize}
      \item Possono essere invitati esperti del dominio tecnologico.
    \end{itemize}
  \end{itemize}
  \item le decisioni spettano SOLO alla \textit{Squadra} (gli Arretrati possono essere modificati solo dalla \textit{Squadra}).
  \item Le attivit\`a devono essere perfezionate e valutate.
\end{itemize}

\subsection{Gli incontri giornalieri}

\begin{itemize}
  \item Incontro faccia a faccia.
  \item Partecipano la \textit{Squadra} e il \textit{Gran Maestro del Socialismo Internazionale}.
  \item La durata massima \`e fissata a circa 15 minuti.
  \item Sempre nello stesso posto alla stessa ora ogni giorno lavorativo.
  \item Il \textit{GMSI} inizia l'incontro partendo dalla sua sinistra e procedendo in senso antiorario.
  \item Ogni membro della \textit{Squadra} risponde a 3 domande:
  \begin{itemize}
    \item Cosa ho fatto dall'ultimo incontro giornaliero?
    \item Cosa far\`o da ora al prossimo incontro giornaliero?
    \item Cosa mi impedisce di svolgere il mio lavoro nel modo pi\`u efficace possibile?
  \end{itemize}
  \item Nessuna digressione, solo risposte brevi alle domande.
  \item Parla solo una persona alla volta, le altre persone ascoltano, nessuna interruzione, nessuna discussione.
  \item Dopo che un membro della \textit{Squadra} ha riferito, gli altri membri della \textit{Squadra} possono chiedere di organizzare un incontro dopo l'incontro giornaliero.
\end{itemize}

\subsection{Gli incontri di revisione}

\begin{itemize}
  \item La durata massima \`e fissata a circa 4 ore.
  \item Partecipano il \textit{Rappresentante del Popolo}, la \textit{Squadra} e il \textit{Gran Maestro del Socialismo Internazionale}.
  \item La \textit{Squadra} non dovrebbe impiegare pi\`u di 1 ora per preparare questa riunione.
  \item Obiettivi:
  \begin{itemize}
    \item Presentare al \textit{RP} e al \textit{Popolo} una funzionalit\`a che \`e ESEGUITA.
    \item Una funzionalit\`a che non \`e eseguita NON PU\`O essere presentata.
    \item Tutto ciò che non \`e una funzionalità NON PU\`O essere presentato, a meno che non sia per supportare la comprensione di una funzionalit\`a.
    \item La funzionalit\`a \`e dimostrata dalle postazioni di lavoro della \textit{Squadra} (solitamente il server di garanzia della qualit\`a).
  \end{itemize}
  \item Inizia con un membro della \textit{Squadra} che presenta: obiettivo dell'iterazione, arretrati impegnato, arretrati completato.
  \item La maggior parte della riunione riguarda:
  \begin{itemize}
    \item I membri della \textit{Squadra} dimostrano le funzionalit\`a.
    \item Il \textit{RP} e il \textit{Popolo} pongono alcune domande.
    \item I membri della \textit{Squadra} annotano le modifiche da apportare.
  \end{itemize}
  \item La riunione termina con un sondaggio del \textit{Popolo}.
  \item Il \textit{PR}, il \textit{Popolo} e la \textit{Squadra} discutono di una possibile riorganizzazione degli arretrati.
\end{itemize}

\subsection{Gli incontri di retrospettiva}

\begin{itemize}
  \item La durata massima \`e fissata a circa 3 ore.
  \item Il \textit{RP} (facoltativo), la \textit{Squadra} e il \textit{GMSI} discutono di cosa \`e andato bene e cosa \`e andato male e di conseguenza modificano i piani per la prossima iterazione.
  \item I membri della \textit{Squadra} rispondono a 2 domande:
  \begin{itemize}
    \item Cosa \`e andato bene durante l'ultima iterazione?
    \item Cosa potrebbe essere migliorato nella prossima iterazione?
  \end{itemize}
  \item Il \textit{GMSI} riassume tutte le risposte in un modulo.
  \item La \textit{Squadra} assegna la priorità agli elementi nel modulo.
  \item La \textit{Squadra} decide quali elementi vengono trasformati in elementi di azione ad alta priorit\`a nella prossima iterazione, come requisiti non funzionali.
\end{itemize}

\subsection{Gli incontri di toelettatura}

Si possono prevedere o richiedere degli incontri finalizzati alla scrittura o al perfezionamento di attivit\`a e storie. In genere una per iterazione.

\section{Attivit\`a e Storie}

Le attivit\`a sono descrizioni di cose da fare o da ottenere in modo concreto.
Le Storie Utente sono descrizioni di funzionalit\`a singole (o situazioni dal punto di vista dell'utente). \`E necessario che siano:
\begin{itemize}
  \item Indipendenti
  \item Negoziabili
  \item Preziose per gli utenti
  \item Valutabili
  \item Piccole
  \item Testabili
\end{itemize}

In base al tipo di applicazione pu\`o risultare difficile o impossibile realizzare delle Storie Utente (caso di applicazioni dove il ruolo degli utenti \`e assente o molto limitato).
In questi casi si avranno prevalentemente Attivit\`a (o "Carte"). Le Storie Utente sono anch'esse attivit\`a.
Per scrivere le Storie Utente \`e necessario seguire uno stile consistenze ed esplicito per consentire di chiarire le richieste, le funzionalit\`a, i limiti e le negoziazioni.
\`E buona cosa descrivere nelle Storie Utente il diritto di Conferma e di Refutazione se \`e importante.

Altro tipo di attivit\`a sono le Attivit\`a Tecniche, che richiedono di solito una descrizione di azioni che non riguardano il prodotto ma lo sviluppo di esso o altri dettagli tecnici.
Queste possono essere rappresentate anche come Storie con al centro gli sviluppatori.

Per valutare l'impegno necessario per eseguire una storia (o un'attivit\`a) si utilizza un Punteggio espresso in numeri interi (per esempio seguendo la sequenza di Fibonacci: 1, 2, 3, 5, 8, 13 ... etc).
Grazie a questi punteggi \`e possibile confrontare l'impegno delle storie. Il valore assoluto assegnato come punteggio non \`e molto rilevante, ma \`e altres\`i importante rispettare il valore relativo tra i punteggi delle storie: una storia da 5 punti deve richiedere più del doppio dello sforzo di una storia da 2 punti.

\subsection{Stima dei Punteggi (o Briscola delle Storie)}

\putimage{images/briscola-delle-storie.jpg}{La "Briscola delle Storie"}{img:briscola-delle-storie}

Per ogni Storia (o attivit\`a):
\begin{itemize}
  \item si discute brevemente cosa prevede.
  \item ogni membro della \textit{Squadra} seleziona una carta il cui punteggio equivale alla sua stima di quanto impegno \`e richiesto secondo lui (ovvero esprime una stima personale).
  \item se le stime sono concordi allora il punteggio della storia \`e deciso.
  \item altrimenti si discute per definire un punteggio su cui siano tutti d'accordo.
  \item \`E fortemente sconsigliato utilizzare approcci puramente matematici, come la media o la mediana, per sintetizzare in breve tempo le stime emesse dai membri.
\end{itemize}

\subsection{Velocit\`a}

La velocit\`a istantanea del processo di sviluppo \`e espressa come la somma dei punti delle storie eseguite in ogni singola iterazione.
La velocit\`a approssimata del processo si esprime come la media delle ultime N iterazioni.
