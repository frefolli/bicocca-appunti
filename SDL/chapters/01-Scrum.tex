\chapter{Scrum}

\section{I principi}

Si basa su tre principi:
\begin{itemize}
  \item **Visibilita'**
  \begin{itemize}
    \item gli aspetti che condizionano il risultato (quali feature sono implementate, quali sono testate, quali sono da modificare) devono essere chiaramente visibili al *Direttorato* cosicche' possa prendere decisioni consapevolmente.
    \item E' altresi' importante dare definizioni chiare, per sapere per esempio cosa si intende con il termine 'eseguito'. Alcuni esempi di definizioni:
    \begin{itemize}
      \item I tests sono stati scritti e sono tutti passati.
      \item I test di analisi statica sono tutti passati.
      \item Il codice e' stato riorganizzato e rivisto con successo.
      \item Il software e' stato dispiegato nell'ambiente di Messa in Scena (o Produzione).
      \item La documentazione e' disponibile in Italiano e in Inglese.
      \item ... etc
    \end{itemize}
  \end{itemize}
  \item **Ispezione**
  \begin{itemize}
    \item Tutti gli aspetti del processo di sviluppo devono essere ispezionati per identificare i problemi
    \item Le frequenze di ispezione possono dipendere dalla frequenza di aggiornamento/pubblicazione/generazione di artefatti, rapporti, oggettistica.
  \end{itemize}
  \item **Adattamento**
  \begin{itemize}
    \item Se c'e' qualche problema, il *Direttorato* deve prendere immediati provvedimenti per ricondurre alla socialista via sia il Software che il Processo e risolvere i problemi nel piu' breve tempo possibile.
  \end{itemize}
\end{itemize}

\section{Il processo}

\putimage{images/processo-scrum.png}{Il processo Scrum}{img:processo-scrum}

I Punti cardine del processo sono:
\begin{itemize}
  \item Iterazioni brevi di 1/4 settimane
  \item Mantenimento di una lista degli Arretrati.
  \item Un incontro Scrum giornaliero.
  \item Al termine di ogni iterazione si ha un prodotto incrementalmente costruito e immediatamente dispiegabile.
\end{itemize}

\section{I ruoli}

I componenti del Direttorio assumono ruoli ben definiti:
\begin{itemize}
  \item **Rappresentante del Popolo** (RP)
  \begin{itemize}
    \item E' responsabile della massimizzazione del valore fornito dalla *squadra* assicurando che gli arretrati della squadra siano allineati con le esigenze del Popolo.
  \end{itemize}
  \item **Squadra** (SQ)
  \begin{itemize}
    \item sviluppa in modo incrementale e autonomo la funzionalita' in base alle direzioni del *Rappresentante del Popolo*.
  \end{itemize}
  \item **Gran Maestro del Socialismo Internazionale** (GMSI)
  \begin{itemize}
    \item E' responsabile del processo di sviluppo, insegna e adatta Scrum alle esigenze della Sezione, supervisiona il comportamento di tutti i partecipanti al processo Scrum.
  \end{itemize}
  \item Preferibilmente sono interpretati da persone diverse
  \item I ruoli possono essere riassegnati tra le iterazioni
\end{itemize}

Nel presente documento si fa anche riferimento a ulteriori ruoli che sono *esterni* al Direttorio:
\begin{itemize}
\item **Popolo**
  \begin{itemize}
    \item E' il gruppo dei portatori di interesse e dei proprietari dei mezzi di produzione, rappresentato dal *RP*.
  \end{itemize}
\end{itemize}

\section{Gli arretrati}

\putimage{images/arretrati.png}{Gli arretrati}{img:arretrati}

\section{Gli incontri}

\subsection{Gli incontri di pianificazione}

Ogni incontro dovrebbe essere diviso in parti consecutive:

\begin{itemize}
  \item Stabilire gli obiettivi dell'iterazione (4 ore)
  \begin{itemize}
    \item Partecipano il *Rappresentante del Popolo*, la *Squadra* e il *Gran Maestro del Socialismo Internazionale*.
    \begin{itemize}
      \item Possono essere invitati esperti della Direzione Politica di Stato.
    \end{itemize}
    \item il *RP* prepara l'elenco degli Arretrati di Produzione prima della riunione.
    \item il *RP* presenta le voci degli Arretrati con la priorita' piu' alta.
    \item la *Squadra* pone domande su contenuto, scopo, significato, …
    \item Prima che trascorrano le 4 ore, la *Squadra* seleziona la funzionalita' del prodotto che puo' essere impegnata entro la fine della prossima iterazione.
    \item Le attivita' possono essere un mix di epiche e storie, a seconda delle priorita' e della pertinenza delle attivita'.
  \end{itemize}
  \item Perfezionare l'evasione degli obiettivi (4 ore)
  \begin{itemize}
    \item Partecipano il *Rappresentante del Popolo*, la *Squadra* e il *Gran Maestro del Socialismo Internazionale*.
    \begin{itemize}
      \item Possono essere invitati esperti del dominio tecnologico.
    \end{itemize}
  \end{itemize}
  \item le decisioni spettano SOLO alla *Squadra* (gli Arretrati possono essere modificati solo dalla *Squadra*).
  \item Le attivita' devono essere perfezionate e valutate.
\end{itemize}

\subsection{Gli incontri giornalieri}

\begin{itemize}
  \item Incontro faccia a faccia.
  \item Partecipano la *Squadra* e il *Gran Maestro del Socialismo Internazionale*.
  \item La durata massima e' fissata a circa 15 minuti.
  \item Sempre nello stesso posto alla stessa ora ogni giorno lavorativo.
  \item Il *GMSI* inizia l'incontro partendo dalla sua sinistra e procedendo in senso antiorario.
  \item Ogni membro della *Squadra* risponde a 3 domande:
  \begin{itemize}
    \item Cosa ho fatto dall'ultimo incontro giornaliero?
    \item Cosa faro' da ora al prossimo incontro giornaliero?
    \item Cosa mi impedisce di svolgere il mio lavoro nel modo piu' efficace possibile?
  \end{itemize}
  \item Nessuna digressione, solo risposte brevi alle domande.
  \item Parla solo una persona alla volta, le altre persone ascoltano, nessuna interruzione, nessuna discussione.
  \item Dopo che un membro della *Squadra* ha riferito, gli altri membri della *Squadra* possono chiedere di organizzare un incontro dopo l'incontro giornaliero.
\end{itemize}

\subsection{Gli incontri di revisione}

\begin{itemize}
  \item La durata massima e' fissata a circa 4 ore.
  \item Partecipano il *Rappresentante del Popolo*, la *Squadra* e il *Gran Maestro del Socialismo Internazionale*.
  \item La *Squadra* non dovrebbe impiegare piu' di 1 ora per preparare questa riunione.
  \item Obiettivi:
  \begin{itemize}
    \item Presentare al *RP* e al *Popolo* una funzionalita' che e' ESEGUITA.
    \item Una funzionalita' che non e' eseguita NON PUO' essere presentata.
    \item Tutto ciò che non e' una funzionalità NON PUO' essere presentato, a meno che non sia per supportare la comprensione di una funzionalita'.
    \item La funzionalita' e' dimostrata dalle postazioni di lavoro della *Squadra* (solitamente il server di garanzia della qualita').
  \end{itemize}
  \item Inizia con un membro della *Squadra* che presenta: obiettivo dell'iterazione, arretrati impegnato, arretrati completato.
  \item La maggior parte della riunione riguarda:
  \begin{itemize}
    \item I membri della *Squadra* dimostrano le funzionalita'.
    \item Il *RP* e il *Popolo* pongono alcune domande.
    \item I membri della *Squadra* annotano le modifiche da apportare.
  \end{itemize}
  \item La riunione termina con un sondaggio del *Popolo*.
  \item Il *PR*, il *Popolo* e la *Squadra* discutono di una possibile riorganizzazione degli arretrati.
\end{itemize}

\subsection{Gli incontri di retrospettiva}

\begin{itemize}
  \item La durata massima e' fissata a circa 3 ore.
  \item Il *RP* (facoltativo), la *Squadra* e il *GMSI* discutono di cosa e' andato bene e cosa e' andato male e di conseguenza modificano i piani per la prossima iterazione.
  \item I membri della *Squadra* rispondono a 2 domande:
  \begin{itemize}
    \item Cosa e' andato bene durante l'ultima iterazione?
    \item Cosa potrebbe essere migliorato nella prossima iterazione?
  \end{itemize}
  \item Il *GMSI* riassume tutte le risposte in un modulo.
  \item La *Squadra* assegna la priorità agli elementi nel modulo.
  \item La *Squadra* decide quali elementi vengono trasformati in elementi di azione ad alta priorita' nella prossima iterazione, come requisiti non funzionali.
\end{itemize}

\subsection{Gli incontri di toelettatura}

Si possono prevedere o richiedere degli incontri finalizzati alla scrittura o al perfezionamento di attivita' e storie. In genere una per iterazione.

\section{Attivita' e Storie}

Le attivita' sono descrizioni di cose da fare o da ottenere in modo concreto.
Le Storie Utente sono descrizioni di funzionalita' singole (o situazioni dal punto di vista dell'utente). E' necessario che siano:
\begin{itemize}
  \item Indipendenti
  \item Negoziabili
  \item Preziose per gli utenti
  \item Valutabili
  \item Piccole
  \item Testabili
\end{itemize}

In base al tipo di applicazione puo' risultare difficile o impossibile realizzare delle Storie Utente (caso di applicazioni dove il ruolo degli utenti e' assente o molto limitato).
In questi casi si avranno prevalentemente Attivita' (o "Carte"). Le Storie Utente sono anch'esse attivita'.
Per scrivere le Storie Utente e' necessario seguire uno stile consistenze ed esplicito per consentire di chiarire le richieste, le funzionalita', i limiti e le negoziazioni.
E' buona cosa descrivere nelle Storie Utente il diritto di Conferma e di Refutazione se e' importante.

Altro tipo di attivita' sono le Attivita' Tecniche, che richiedono di solito una descrizione di azioni che non riguardano il prodotto ma lo sviluppo di esso o altri dettagli tecnici.
Queste possono essere rappresentate anche come Storie con al centro gli sviluppatori.

Per valutare l'impegno necessario per eseguire una storia (o un'attivita') si utilizza un Punteggio espresso in numeri interi (per esempio seguendo la sequenza di Fibonacci: 1, 2, 3, 5, 8, 13 ... etc).
Grazie a questi punteggi e' possibile confrontare l'impegno delle storie. Il valore assoluto assegnato come punteggio non e' molto rilevante, ma e' altresi' importante rispettare il valore relativo tra i punteggi delle storie: una storia da 5 punti deve richiedere più del doppio dello sforzo di una storia da 2 punti.

\subsection{Stima dei Punteggi (o Briscola delle Storie)}

\putimage{images/briscola-delle-storie.jpg}{La "Briscola delle Storie"}{img:briscola-delle-storie}

Per ogni Storia (o attivita'):
\begin{itemize}
  \item si discute brevemente cosa prevede.
  \item ogni membro della *Squadra* seleziona una carta il cui punteggio equivale alla sua stima di quanto impegno e' richiesto secondo lui (ovvero esprime una stima personale).
  \item se le stime sono concordi allora il punteggio della storia e' deciso.
  \item altrimenti si discute per definire un punteggio su cui siano tutti d'accordo.
  \item E' fortemente sconsigliato utilizzare approcci puramente matematici, come la media o la mediana, per sintetizzare in breve tempo le stime emesse dai membri.
\end{itemize}

\subsection{Velocita'}

La velocita' istantanea del processo di sviluppo e' espressa come la somma dei punti delle storie eseguite in ogni singola iterazione.
La velocita' approssimata del processo si esprime come la media delle ultime N iterazioni.
