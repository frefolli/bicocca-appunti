\chapter{Introduzione}

\section{Classical AI}

\paragraph{Agentificazione dell'AI}

Un singolo agente che risolve problemi, adottando varie soluzioni e tecniche.
Ma cos'e' un agente?

\paragraph{Secondo Russel e Norvig}, un agente e' qualsiasi cosa che percepisce il suo ambiente tramite dei sensori e agisce su esso tramite degli attuatori.
\paragraph{Secondo Wooldridge-Jennings} un agente e' un sistema computazionale che gode di queste proprieta': autonomia, abilita' sociale, reattivita' e pro-attivita'.
\paragraph{Secondo Jacques Ferber} un Agente e' un’entita' fisica o virtuale:

\begin{itemize}
  \item Che e' capace di agire in un ambiente
  \item Che puo' comunicare direttamente con altri agenti
  \item Che e' guidato da un insieme di tendenze (sotto forma di obiettivi individuali o di a funzione di soddisfazione/sopravvivenza che cerca di ottimizzare)
  \item Che possiede risorse proprie
  \item Che e' in grado di percepire il suo ambiente (ma in misura limitata)
  \item Che ha solo una rappresentazione parziale di questo ambiente (e forse nessuna)
  \item Che possiede competenze e puo' offrire servizi
  \item Il cui comportamento tende a soddisfare i propri obiettivi, tenendo conto delle risorse e delle competenze a sua disposizione e in funzione della sua percezione, della sua rappresentazione e della comunicazioni che riceve (Jacques Ferber)
\end{itemize}

\putimage{images/01-01.png}{Schema}{png:1-1}

\paragraph{Agenti Umani}: occhi, orecchie e altri organi sono sensori; mani, piedi ... etc sono gli attuatori.

\paragraph{Agenti Robotici}: telecamere e scanner infrarossi sono sensori; varie pompe idrauliche o motori possono essere gli attuatori.

\paragraph{La agent function} mappa lo storico delle percezioni alle azioni ($f: P^\star \rightarrow A$).

\paragraph{L'agent program} esegue sull'architettura fisica per produrre le azioni di f.

\paragraph{Agent = Architecture + Program}

\section{Distributed AI}

Passiamo da un singolo agente intelligente che implementa una soluzione ad un sistema di entita' che nell'insieme risolve un problema interagendo con l'ambiente circostante.
Ma cos'e' un sistema?

\paragraph{Un sistema} puo' essre definito come un gruppo di elementi che interagiscono regolarmente o indipendenti che formano un tutt'uno (il sistema capitalista, un insieme di organi, un gruppo di devices ... etc).
Dare una definizione di sistema comunque non e' facile.

Gia' definire cosa e' distribuito non e' poca cosa:

\begin{itemize}
  \item Risoluzione distribuita di problemi
  \item Risoluzione di problemi distribuiti
  \item tecniche distribuite di risoluzione di problemi
\end{itemize}

\paragraph{Maes} definisce gli agenti autonomi come sistemi computazionali che \textit{abitano} un ambiente dinamico e complesso, sentono e agiscono in modo autonomo in questo ambiente per realizzare un set di goal o attivita' per cui sono stati creati.
\paragraph{Hayes-Roth} definiscono gli agenti intelligenti come continuamente performanti tre funzioni: percezione di condizioni dinamiche dell'ambiente, azioni che hanno effetto su queste condizioni e ragionamenti per intepretare le percezioni, risolvere problemi, fare inferenze e determinare le azioni.

