\chapter{Installare Rust}

Il processo di installazione consigliato e' mediante \textbf{rustup}, uno script di installazione che permette di mantenere un ecosistema consistente e scaricare su qualsiasi ambiente una copia di esso in formato minimale o completo.

All'inizio fa un'analisi della macchina, quindi chiede conferma o permette di modificare le impostazioni di installazione. 
Io ho installato rust con il formato completo.

Una volta terminato usciranno due scritte:

\begin{verbatim}
  stable-i686-unknown-linux-gnu installed - rustc 1.67.1 (d5a82bbd2 2023-02-07)
  Rust is installed now. Great!
\end{verbatim}

La prima dipende dalla macchina su chi si e'. Nel mio caso linux 32bit.

Quindi chiede di riavviare la shell per aggiornare la variabile d'ambiente PATH per avere al suo interno anche
\begin{verbatim}
  "\$HOME/.cargo/bin"
\end{verbatim}
ovvero il percorso root dell'installazione rust.

Se si preferisce mantenere la sessione bastera' aggiornare manualmente la PATH con
\begin{lstlisting}
  source "\$HOME/.cargo/env"
\end{lstlisting}
