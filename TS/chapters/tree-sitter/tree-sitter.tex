\chapter{Tree Sitter}

\section{Progetto Cargo}

\paragraph{Creare un progetto Cargo}

\begin{verbatim}
	cargo new <project-name>
\end{verbatim}

\paragraph{Configurare il progetto Cargo}

sotto la sezione \textbf{dependencies} bisogna aggiungere le dipendenze di tree sitter, quindi.

\begin{verbatim}
  [dependencies]
  tree-sitter = "*"
\end{verbatim}

Quindi per ogni linguaggio che si vuole supportare si aggiunge la seguente riga alla sezione sopra citata:

\begin{verbatim}
  tree-sitter-<language-name> = "*"
\end{verbatim}

Nel caso del \textbf{c++}:

\begin{verbatim}
  tree-sitter-cpp = "*"
\end{verbatim}

Se possibile si puo' inserire anche il semver della versione del pacchetto.

\paragraph{Compilazione}

\begin{verbatim}
	cargo build
\end{verbatim}

\paragraph{Esecuzione}

\begin{verbatim}
	cargo run
\end{verbatim}

\paragraph{Codice}

Per usare tree sitter bisogna importare il suo modulo e quindi creare un parser.

\begin{verbatim}
  let mut the_parser = tree_sitter::Parser::new():
\end{verbatim}

Quindi e' necessario dirgli il linguaggio di cui fare il parsing

\begin{verbatim}
  the_parser.set_language(the_language);
\end{verbatim}

\textbf{the_language} e' una variabile di tipo \begin{verbatim}tree_sitter::Language\end{verbatim};
Usando l'appropriato modulo tree_sitter_<language_name> si puo' importare il linguaggio desiderato e quindi assegnarlo al parser.
Nel caso del \textbf{c++}:

\begin{verbatim}
  let the_parser = tree_sitter_cpp::language();
\end{verbatim}

Quindi per fare il parsing e ottenere un albero con il DSL di tree sitter basta usare:

\begin{verbatim}
  let tree = the_parser.parse(the_text);
\end{verbatim}

\textbf{theText} e' una stringa che contiene il codice da analizzare; puo' essere caricata per esempio da un file con:

\begin{verbatim}
  std::fs::read_to_string(file_path);
\end{verbatim}
