\chapter{Problema LIS}

\section{Introduzione}

\paragraph{LIS}

\subparagraph{Introduzione}

Il problema LIS (Longest Increasing Subsequence) consiste nel cercare in tempo ragionevole la sottosequenza piu' lunga crescente all'interno di una stringa o sequenza.

\subparagraph{Esempi}

\begin{itemize}

\item

$X$ = <0, 8, 4, 12, 2, 10, 6, 14, 1, 9, 5, 13, 3, 11, 7, 15> \\
$Z$ = <0, 2, 6, 9, 11, 15> \\

Ma non esiste solo quella soluzione, altri candidati sono

$Z$ = <0, 4, 6, 9, 11, 15> \\
$Z$ = <0, 2, 6, 9, 13, 15> \\
$Z$ = <0, 4, 6, 9, 13, 15> \\

\end{itemize}

\section{Algoritmo TOP-DOWN}

\paragraph{Ragionamento}

Sia $X = <1,..,n>$ la sequenza di input. \\*
Sia $Z = <i,..,k>$ la soluzione $LIS(X)$. \\*
Chiamo $Z_i = LIS(X_i)$. \\

Detta $P : X \rightarrow X$ la funzione che associa ad ogni $x_i$ un $x_j$ \\*
$x_j$ = max-index$(\{\forall x_j \in X_i \mid j < i \land x_j < x_j\})$. \\

Ragiono quindi per prefissi, e imposto la ricorrenza secondo un teorema strutturale. \\

\begin{enumerate}
    \item $x_i \in Z_i \Leftrightarrow Z_i = Z_{P(i)} \cup x_i$
    \item $x_i \notin Z_i \Leftrightarrow Z_i = Z_{i-1}$
\end{enumerate}

Non sapendo se $x_i$ appartiene o meno a $Z_i$, l'algoritmo dovra' testare entrambe le possibilita'.
Sono invece immediati i casi base:

\begin{enumerate}
    \item $X_i = \emptyset \Leftrightarrow Z_i = \emptyset$
    \item $X_i = \{x_i\} \Leftrightarrow Z_i = \{x_i\}$
\end{enumerate}

L'equazione di ricorrenza di conseguenza e':
\[
    Z_i =
    \begin{cases}
        \text{$\emptyset$} & \text{$X_i = \emptyset$} \\
        \text{$\{x_i\}$} & \text{$X_i = \{x_i\}$} \\
        \text{$Z_{P(i)} \cup x_i$} & \text{$x_i \in Z_i$} \\
        \text{$Z_{i-1}$} & \text{$x_i \notin Z_i$}
    \end{cases}
\]

Ridotta a:

\[
    Z_i =
    \begin{cases}
        \text{$\emptyset$} & \text{$X_i = \emptyset$} \\
        \text{$\{x_i\}$} & \text{$X_i = \{x_i\}$} \\
        \text{$Z_{P(i)} \cup x_i$} & \text{$|Z_{P(i)} \cup x_i| \geq |Z_{i-1}|$} \\
        \text{$Z_{i-1}$} & \text{$|Z_{P(i)} \cup x_i|< |Z_{i-1}|$}
    \end{cases}
\]

\paragraph{Dimostrazione}

Per dimostrare il teorema mi avvalgo dei sottoproblemi di LIS. \\*
Assumo quindi che $\forall j \in \{0,..,i-1\} \mid Z_j = LIS(X_j)$. \\*
Inoltre assumo che il problema per $X_i$ abbia almeno una soluzione.

\subparagraph{1}

Sia $x_i \in Z_i$. \\
Supponiamo per assurdo che $Z_i \ne Z_{P(i)} \cup x_i$. \\
Poiche' il problema ammette soluzione, $\exists Z^I \mid Z^I = LIS(X_i)$. \\
Visto che $Z_i \ne Z_{P(i)} \cup x_i$ allora $|Z^I| > |Z^{P(i)} \cup x_i|$.\\
Siccome $x_i \in Z_i$ allora $Z^I = Z^{II} \cup x_i$. \\
Data la definizione di $P(i)$, necessariamente $Z^{II} \subseteq X_{P(i)}$ \\
Inoltre $|Z^{II}| > |Z^{P(i)} \cup x_i| \Rightarrow |Z^{II}| > |Z^{P(i)}|$. \\
Sapendo $(|Z^{II}| > |Z^{P(i)}|) \land (Z^{II} \subseteq X_{P(i)})$ posso affermare che $Z^{II} = LIS(X_{P(i)})$. \\
Il che e' impossibile, perche' so che $Z_{P(i)} = LIS(X_{P(i)})$. \\
Quindi $Z_i = Z_{P(i)} \cup x_i$.

\subparagraph{2}

Sia $x_i \notin Z_i$. \\
Sia per assurdo che $Z_i \ne Z_{i-1}$. \\
Poiche' il problema ammette soluzione, $\exists Z^I \mid Z^I = LIS(X_i)$. \\
Visto che $Z_i \ne Z_{i-1}$ allora $|Z^I| > |Z^{i-1}|$. \\
Ma se $x_i \notin Z_i$ allora $Z^{I} \subseteq X_{i-1}$. \\
Sapendo $(|Z^I| > |Z^{i-1}|) \land (Z^{I} \subseteq X_{i-1})$ posso affermare che $Z^{I} = LIS(X_{i-1})$. \\
Il che e' impossibile, perche' so che $Z_{i-1} = LIS(X_{i-1})$. \\
Quindi $Z_i = Z_{i-1}$.

\newpage

\section{Procedura TOP-DOWN}

\begin{algorithm}
    \begin{algorithmic}
        \Procedure{LIS}{$X_{i}$}
            \If {$i = 0$}
                \State \Return <>
            \ElsIf {$i = 1$}
                \State \Return <$x_i$>
            \Else
                \State $A \gets append(LIS(P(i)), x_i)$
                \State $B \gets LIS(i-1)$
                \If{$|A| \geq |B|$}
                    \State \Return A
                \Else
                    \State \Return B
                \EndIf
            \EndIf
        \EndProcedure
    \end{algorithmic}
\end{algorithm}

\paragraph{Complessita'}

Esplorando ogni volta due rami, nel caso medio il tempo di esecuzione e' $T(n) = O(2^n)$.

\newpage

\section{Procedura BOTTOM-UP}

Mi servo dei vettori $L$ e $Z$, che contengono rispettivamente $|Z_i|$ e $Z_i$.

\begin{algorithm}
    \begin{algorithmic}
        \Procedure{INIZIALIZZA-VETTORI}{$X$}
            \State $L[0] \gets 0$
            \State $L[1] \gets 1$
            \State $Z[0] \gets X_0$
            \State $Z[1] \gets X_1$
        \EndProcedure
    \end{algorithmic}
\end{algorithm}

\begin{algorithm}
    \begin{algorithmic}
        \Procedure{LIS-ITER}{$X$}
            \State INIZIALIZZA-VETTORI($X$)
            \For{$i = 2$ to $n$}
                \If{$L[P(i)] + 1 \geq L[i-1]$}
                    \State $L[i] \gets L[P(i)] + 1$
                    \State $Z[i] \gets append(Z[P(i)], x_i)$
                \Else
                    \State $L[i] \gets L[i-1]$
                    \State $Z[i] \gets Z[i-1]$
                \EndIf
            \EndFor
            \State \Return $L[n]$
        \EndProcedure
    \end{algorithmic}
\end{algorithm}

\paragraph{Complessita'}

La funzione $P(i)$ ha complessita' $T_P(n) = O(n)$, perche' esplora linearmente l'array alla ricerca di un indice compatibile.
Quindi la complessita' totale nel caso medio dell'algoritmo LIS-ITER e' data da $T(n) = O(n * T_P(n)) = O(n^2)$. \\

Per chiarezza ho riscritto $P(i)$ ma si intende che per raggiungere la massima efficienza e' necessario calcolare una sola volta quel valore e poi applicarlo. \\ 

L'array $Z$ e' sostituibile con un vettore di numeri i cui bit indicano la presenza o meno degli elementi di $X$.
Possiamo risparmiare direttamente la memoria occupata da quel vettore predisponendo una preocedura che ricostruisca alla fine i passaggi ottimali di LIS-ITER.