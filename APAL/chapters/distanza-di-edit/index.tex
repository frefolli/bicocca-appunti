\chapter{Applicazione di PD}

\section{Distanza di Edit}

\paragraph{Introduzione}

E' definita come numero minimo di \textbf{cancellazioni}, \textbf{sostituzioni}, \textbf{inserimenti} che trasformano una stringa $X$ in una seconda stringa $Y$. E' un problema simmetrico. $D_{edit}(X, Y) = D_{edit}(Y, X)$.

\subparagraph{Esempio}

$X = <2, 4, 10, 3, 1>$
$Y = <2, 4, 2, 1>$

\begin{itemize}
    \item Cancellazione di 10.
    \item Sostituisco 3 con 2.
\end{itemize}

Quindi $D_{edit}(X, Y) = 2$.

\begin{itemize}
    \item Inserimento di 10.
    \item Sostituisco 2 con 3.
\end{itemize}

Quindi $D_{edit}(Y, X) = 2$.

\begin{itemize}
    \item Cancellazione di 3.
    \item Sostituisco 10 con 2.
\end{itemize}

Possono non esistere soluzioni uniche.

\paragraph{Problemi}

\subparagraph{PR}

Date due sequenze X, Y, trovare la distanza di edit di X in Y.

\subparagraph{P}

Date due sequenze X, Y, trovare il minimo insieme di operazioni di cancellazione, inserimento, sostituzione che trasformano X in Y.

\paragraph{Soluzione}

\subparagraph{Sottoproblema di PR}

Trovare la distanza di edit dei prefissi $X_i$ e $Y_j$.
Il numero di sottoproblemi e' $(m+1) \cdot (n+1)$.
$d_{i,j} = distanza di edit dei prefissi X_i Y_j$.
Soluzione 'e $d_{m, n}$

\subparagraph{Casi base di PR}

\begin{itemize}
    \item $i = 0 \land j = 0 \Rightarrow d_{0,0} = 0$
    \item $i > 0 \land j = 0 \Rightarrow d_{i,0} = i$
    \item $i = 0 \land j > 0 \Rightarrow d_{0,j} = j$
\end{itemize}

\subparagraph{Caso ricorsivo di PR}

Con $i > 0 \land j > 0$:

\[
    \text{$d_{i,j}$} =
    \begin{cases}
      \text{$d_{i-1, j-1}$} & \text{$x_i = y_j$} \\
      \text{$min(d_{i-1, j-1}, d_{i, j-1}, d_{i-1, j}) + 1$} & altrimenti
    \end{cases}
\]

Il caso "altrimenti" si spiega in questo modo:
\begin{itemize}
    \item $Sostituzione(x_i \rightarrow y_j) + d_{i-1, j-1}$
    \item $Inserimento(y_j) + d_{i,j-1}$
    \item $Cancellazione(x_i) + d_{i-1, j}$
\end{itemize}

\subparagraph{Complessita'}

La complessita' dell'algoritmo ricorsivo e' $\O(3^(n+m))$.

\paragraph{Algoritmo Ricorsivo}

\begin{algorithm}
    \begin{algorithmic}
        \Procedure{ED-RIC}{$X, Y$}
            \If{$j = 0$}
                \State \Return i
            \ElsIf{$i = 0$}
                \State \Return j
            \Else
                \State $A \gets ED-RIC(X_{i-1}, Y_{j}) + 1$
                \State $B \gets ED-RIC(X_{i}, Y_{j-1}) + 1$
                \State $C \gets ED-RIC(X_{i-1}, Y_{j-1}) + 1$
                \If{$A \leq B \land A \leq C$}
                    \State \Return $A$
                \ElsIf{$B \leq A \land B \leq C$}
                    \State \Return $B$
                \Else
                    \State \Return $C$
                \EndIf
            \EndIf
        \EndProcedure
    \end{algorithmic}
\end{algorithm}

\paragraph{Algoritmo Iterativo PD}

Preparo una Matrice $B[m][n]$ che contiene la ricostruzione del percorso iterativo.
Ogni cella $B[i][j] = <Direction, Operation>$

\begin{algorithm}
    \begin{algorithmic}
        \Procedure{ED-ITER-RM}{$m,n$}
            \For{$i = 0$ to $m$}
                \State $M[i][0] \gets i$
                \State $B[i][0] \gets <"\uparrow", Null>$
            \EndFor
            \For{$j = 0$ to $n$}
                \State $M[0][j] \gets j$
                \State $B[0][j] \gets <"\leftarrow", Null>$
            \EndFor
        \EndProcedure
    \end{algorithmic}
\end{algorithm}


\begin{algorithm}
    \begin{algorithmic}
        \Procedure{ED-ITER}{$X, Y$}
            \State $ED-ITER-RM(m, n)$
            \For{$i = 1$ to $m$}
                \For{$j = 0$ to $n$}
                    \If{$x_i = y_j$}
                        \State $M[i][j] \gets M[i-1][j-1]$
                        \State $B[i][j] \gets <"", Null>$
                    \ElsIf{$M[i-1][j] \leq M[i][j-1] \land M[i-1][j] \leq M[i-1][j-1]$}
                        \State $M[i][j] \gets M[i-1][j] + 1$
                        \State $B[i][j] \gets <"\uparrow", Delete>$
                    \ElsIf{$M[i][j-1] \leq M[i-1][j] \land M[i][j-1] \leq M[i-1][j-1]$}
                        \State $M[i][j] \gets M[i][j-1] + 1$
                        \State $B[i][j] \gets <"\leftarrow", Insert>$
                    \Else
                        \State $M[i][j] \gets M[i-1][j-1] + 1$
                        \State $B[i][j] \gets <"\nwarrow", Change>$
                    \EndIf
                \EndFor
            \EndFor
            \State \Return $M[m][n]$
        \EndProcedure
    \end{algorithmic}
\end{algorithm}

\paragraph{Complessita' di $ED-ITER(X, Y)$}

L'algoritmo e' formato da due cicli innestati, quindi $T(n) = \Theta(n \cdot m)$
