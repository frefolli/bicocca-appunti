\chapter{Inter Veaving}
Data: 18-10-2022

\section{Inter Veaving}

\paragraph{Introduzione}

\paragraph{Esempio}

X = <$C, I, A, O$>
Y = <$M, A, M, M, A$>
W = <$C, I, M, A, A, M, M, A, O$>

W e' un inter veaving. L'algoritmo deve decidere se W e' un inter veaving.

\paragraph{Ragionamento}

Definisco i prefissi $X_i$, $Y_j$, $W_{i+j}$.
Vero sse $W_{i+j} e' un inter veaving$.

\paragraph{Caso Base}

\[
    \begin{cases}
        \text{$true$} & \text{$i = 0 \land j = 0$}
    \end{cases}
\]

\paragraph{Caso Ricorsivo}

\[
    \begin{cases}
        \text{$IV(X_i, Y_{j - 1}, W_{j-1})$} & \text{$i = 0 \land j > 0 \land y_j = w_j$} \\
        \text{$IV(X_i, Y_{j - 1}, W_{j-1})$} & \text{$j = 0 \land i > 0 \land x_i = w_i$} \\
        \text{$false$} & \text{$i = 0 \land j > 0 \land y_j \ne w_j$} \\
        \text{$false$} & \text{$j = 0 \land i > 0 \land x_i \ne w_i$} \\
        \text{$IV(X_{i-1}, Y_j, W_{i+j-1})\lor IV(X_i, Y_{j-1}, W_{i+j-1})$} & \text{$i,j > 0$} \\
    \end{cases}
\]

\paragraph{Caso Generale}

\[
    \begin{cases}
        \text{$true$} & \text{$i = 0 \land j = 0$} \\
        \text{$IV(X_i, Y_{j - 1}, W_{j-1})$} & \text{$i = 0 \land j > 0 \land y_j = w_j$} \\
        \text{$IV(X_i, Y_{j - 1}, W_{j-1})$} & \text{$j = 0 \land i > 0 \land x_i = w_i$} \\
        \text{$false$} & \text{$i = 0 \land j > 0 \land y_j \ne w_j$} \\
        \text{$false$} & \text{$j = 0 \land i > 0 \land x_i \ne w_i$} \\
        \text{$IV(X_{i-1}, Y_j, W_{i+j-1})\lor IV(X_i, Y_{j-1}, W_{i+j-1})$} & \text{$i,j > 0$} \\
    \end{cases}
\]

\newpage

\section{Procedura Bottom-Up}

\begin{algorithm}
    \begin{algorithmic}
        \Procedure{IV}{$X_i, Y_j, W_j$}
        \EndProcedure
    \end{algorithmic}
\end{algorithm}
