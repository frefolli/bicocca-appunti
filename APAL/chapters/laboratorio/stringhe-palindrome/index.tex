\chapter{Stringhe Palindrome}
Data 25-10-22

\section{Introduzione}

Data una stringa $S$ voglio il numero minimo di caratteri da aggiungere per renderla palindroma.
I caratteri possono essere aggiunti in qualsiasi posizione.

\paragraph{Esempio 1}

$S$ = "" \\
$PAL(S)$ = $0$. \\
caso base

\paragraph{Esempio 2}

$S$ = "ADDA" \\
$PAL(S)$ = $0$. \\
E' gia' palindroma

\paragraph{Esempio 3}

$S$ = "ARDA" \\
$PAL(S)$ = $1$. \\
Non e' palindroma

\section{Ragionamento}

Un approccio banale e' accodare alla stringa di partenza l'inverso di se stessa.
Ma cosi' si aggiungono $n$ caratteri. Io voglio il minimo numero possibile. \\

Definisco una funzione $f : \Sigma^* \rightarrow \mathbb{N}$ \\*
$\forall S \in \Sigma^*, f(S) = $ minimo numero di caratteri da aggiungere per rendere la stringa palindroma. \\

$S = "" \Rightarrow f(S) = 0$ \\
$\forall \alpha \in Sigma \mid S = \alpha \Rightarrow f(S) = 1$ \\

Se $|S| > 1$ allora $S = \alpha \cdot S^{I} \cdot \beta$.
Se $\alpha = \beta$ allora $f(S) = f(S^{I})$.
Se invece $\alpha \ne \beta$,
$f(S) = min(f(S^{I} \cdot \beta), f(\alpha \cdot S^{I})) + 1$. \\

In questo problema quindi non bastera' mantenere un indice per il prefisso, ma trattandosi di sottostringhe avremo bisogno di due indici $i, j$.

\section{Algoritmo}

\paragraph{Caso Base}

\[
    \begin{cases}
        \text{$0$} & \text{$i \geq j$}
    \end{cases}
\]

\paragraph{Caso Ricorsivo}

\[
    \begin{cases}
        \text{$f(S, i+1, j-1)$} & \text{$i < j \land s_i = s_j$} \\
        \text{$f(S, i+1, j) + 1$} & \text{$i < j \land s_i \ne s_j \land f(S, i+1, j) \leq f(S, i, j-1)$} \\
        \text{$f(S, i, j-1) + 1$} & \text{$i < j \land s_i \ne s_j \land f(S, i+1, j) > f(S, i, j-1)$}
    \end{cases}
\]

\paragraph{Caso Generale}

\[
    \begin{cases}
        \text{$0$} & \text{$i \geq j$} \\
        \text{$f(S, i+1, j-1)$} & \text{$i < j \land s_i = s_j$} \\
        \text{$f(S, i+1, j) + 1$} & \text{$i < j \land s_i \ne s_j \land f(S, i+1, j) \leq f(S, i, j-1)$} \\
        \text{$f(S, i, j-1) + 1$} & \text{$i < j \land s_i \ne s_j \land f(S, i+1, j) > f(S, i, j-1)$}
     \end{cases}
\]

\newpage

\section{Procedura}

\begin{algorithm}
    \begin{algorithmic}
        \Procedure{PAL}{$S$}
            \For{$i = n$ to $1$}
                \For{$j = 1$ to $n$}
                    \If{$i >= j$}
                        \State $M[i][j] \gets 0$
                    \ElsIf{$s_i = s_j$}
                        \State $M[i][j] \gets M[i+1][j-1]$
                    \ElsIf{$s_i \ne s_j$}
                        \If{$M[i+1][j] \leq M[i][j-1]$}
                            \State $M[i][j] \gets M[i+1][j] + 1$
                        \Else
                            \State $M[i][j] \gets M[i][j-1] + 1$
                        \EndIf
                    \EndIf
                \EndFor
            \EndFor
            \State \Return $M[|S|][|S|]$
        \EndProcedure
    \end{algorithmic}
\end{algorithm}

\paragraph{Complessita'}

$T(n) = \Theta(n^2)$

\chapter{Esercizio Genoa}

Determinare la lunghezza di una piu' lunga sottosequenza comune di $X$ e $Y$ con al massimo $R$ simboli colorati di rosso. $COL : \Sigma \rightarrow \{ rosso, blu \}$.
Chiamo la sottosequenza in questione $SEQ(X, Y)$.

\section{Ragionamento}

\paragraph{Caso Base}

\[
    \begin{cases}
        \text{$0$} & \text{$i = 0 \lor j = 0$} \\
    \end{cases}
\]

\paragraph{Caso Ricorsivo}
Uso una funzione $ROS : \Sigma^* \cdot \Sigma^* \rightarrow \mathbb{N}$, con \\*
$ROS(X, Y) = |\{ \alpha \mid \forall \alpha \in SEQ(X, Y) \mid COL(\alpha) = rosso\}|$

\[
    \begin{cases}
        \text{$GEN(X_{i-1}, Y_{j-1}) + 1$} & \text{$x_i = y_j \land (ROS(X_{i-1}, Y_{j-1}) < R \lor COL(x_i) \ne rosso)$} \\
        \text{$GEN(X_{i-1}, Y_{j-1})$} & \text{$x_i = y_j \land ROS(X_{i-1}, Y_{j-1}) = R \land COL(x_i) = rosso$} \\
        \text{$GEN(X_{i-1}, Y_j)$} & \text{$x_i \ne y_j \land GEN(X_{i-1}, Y_j) \geq GEN(X_i, Y_{j-1})$} \\
        \text{$GEN(X_i, Y_{j-1})$} & \text{$x_i \ne y_j \land GEN(X_{i-1}, Y_j) < GEN(X_i, Y_{j-1})$} \\
    \end{cases}
\]

\paragraph{Caso Generale}

\[
    \begin{cases}
        \text{$0$} & \text{$i = 0 \lor j = 0$} \\
        \text{$GEN(X_{i-1}, Y_{j-1}) + 1$} & \text{$x_i = y_j \land (ROS(X_{i-1}, Y_{j-1}) < R \lor COL(x_i) \ne rosso)$} \\
        \text{$GEN(X_{i-1}, Y_{j-1})$} & \text{$x_i = y_j \land ROS(X_{i-1}, Y_{j-1}) = R \land COL(x_i) = rosso$} \\
        \text{$GEN(X_{i-1}, Y_j)$} & \text{$x_i \ne y_j \land GEN(X_{i-1}, Y_j) \geq GEN(X_i, Y_{j-1})$} \\
        \text{$GEN(X_i, Y_{j-1})$} & \text{$x_i \ne y_j \land GEN(X_{i-1}, Y_j) < GEN(X_i, Y_{j-1})$} \\
    \end{cases}
\]

\chapter{Esercizio a Casa}


