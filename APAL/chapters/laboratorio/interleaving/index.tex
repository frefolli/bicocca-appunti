\chapter{Interleaving}
Data: 18-10-2022

\section{Interleaving}

\paragraph{Introduzione}

\paragraph{Esempio}

X = <$C, I, A, O$>\\
Y = <$M, A, M, M, A$>\\
W = <$C, I, M, A, A, M, M, A, O$>\\

W e' un interleaving. L'algoritmo deve decidere se W e' un interleaving.
Nel problema si assume che $|W| = i+j$, perche' altrimenti la soluzione nel caso specifico $|W| \ne i+j$ sarebbe triviale.

\paragraph{Ragionamento}

Definisco i prefissi $X_i$, $Y_j$, $W_{i+j}$.
Vero sse $W_{i+j}$ e' un interleaving.

\paragraph{Caso Base}

\[
    \begin{cases}
        \text{$true$} & \text{$i = 0 \land j = 0$} \\
        \text{$false$} & \text{$i = 0 \land j > 0 \land y_j \ne w_{i+j}$} \\
        \text{$false$} & \text{$j = 0 \land i > 0 \land x_i \ne w_{i+j}$} \\
        \text{$false$} & \text{$i, j > 0 \land y_j \ne w_{i+j} \land x_i \ne w_{i+j}$}
    \end{cases}
\]

\paragraph{Caso Ricorsivo}

\[
    \begin{cases}
        \text{$IL(X_{i-1}, Y_j, W_{i+j-1})$} & \text{$i > 0 \land (j = 0 \lor y_j \ne w_{i+j}) \land x_i = w_{i+j}$} \\
        \text{$IL(X_i, Y_{j-1}, W_{i+j-1})$} & \text{$j > 0 \land (i = 0 \lor x_i \ne w_{i+j}) \land y_j = w_{i+j}$} \\
        \text{$IL(X_{i-1}, Y_j, W_{i+j-1})\lor IL(X_i, Y_{j-1}, W_{i+j-1})$} & \text{$i,j > 0 \land x_i = w_{i+j} \land y_j = w_{i+j}$}
    \end{cases}
\]

\paragraph{Caso Generale}

\[
    \begin{cases}
        \text{$true$} & \text{$i = 0 \land j = 0$} \\
        \text{$false$} & \text{$i = 0 \land j > 0 \land y_j \ne w_{i+j}$} \\
        \text{$false$} & \text{$j = 0 \land i > 0 \land x_i \ne w_{i+j}$} \\
        \text{$false$} & \text{$i, j > 0 \land y_j \ne w_{i+j} \land x_i \ne w_{i+j}$} \\
        \text{$IL(X_{i-1}, Y_j, W_{i+j-1})$} & \text{$i > 0 \land (j = 0 \lor y_j \ne w_{i+j}) \land x_i = w_{i+j}$} \\
        \text{$IL(X_i, Y_{j-1}, W_{i+j-1})$} & \text{$j > 0 \land (i = 0 \lor x_i \ne w_{i+j}) \land y_j = w_{i+j}$} \\
        \text{$IL(X_{i-1}, Y_j, W_{i+j-1})\lor IL(X_i, Y_{j-1}, W_{i+j-1})$} & \text{$i,j > 0 \land x_i = w_{i+j} \land y_j = w_{i+j}$}
    \end{cases}
\]

\section{Procedura Bottom-Up}

\begin{algorithm}
    \begin{algorithmic}
        \Procedure{INIZIALIZZA}{$n, m$}
            \State $M[0][0] \gets true$
            \For{$i = 1$ to $n$}
                \State $M[i][0] \gets x_i = w_i$
            \EndFor
            \For{$j = 1$ to $m$}
                \State $M[0][j] \gets y_j = w_j$
            \EndFor
        \EndProcedure
    \end{algorithmic}
\end{algorithm}

\begin{algorithm}
    \begin{algorithmic}
        \Procedure{IL}{$X, Y, W$}
            \State INIZIALIZZA($n, m$)
            \For{$i = 1$ to $n$}
                \For{$j = 1$ to $m$}
                    \If{$x_i = w_{i+j}$}
                        \If{$y_j = w_{i+j}$}
                            \State $M[i][j] \gets M[i-1][j] \lor M[i][j-1]$
                        \Else
                            \State $M[i][j] \gets M[i-1][j]$
                        \EndIf
                    \Else
                        \If{$y_j = w_{i+j}$}
                            \State $M[i][j] \gets M[i][j-1]$
                        \Else
                            \State $M[i][j] \gets false$
                        \EndIf
                    \EndIf
                \EndFor
            \EndFor
        \EndProcedure
    \end{algorithmic}
\end{algorithm}

\paragraph{Complessita'}

La complessita' computazionale e' banalmente $\Theta(n \cdot m)$.