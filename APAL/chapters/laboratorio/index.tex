\part{Laboratorio}
\newcommand{\R}{\mathbb{R}}

\chapter{11-10-2022}

\section{Scheduling di Attivita'}
Un esempio di Programmazione per Intervalli Pesati. \\*
Siano date $n$ attivita' da svolgersi nello stesso spazio fisico.
Determinare un sottoinsieme di attivita' che non si sovrappongono e che sia il massimo possibile.

\begin{center}
    \begin{tabular}{||c c||}
        \hline
        i & v \\
        \hline
        \hline
        1 & 10 \\
        \hline
        2 & 2 \\
        \hline
        3 & 8 \\
        \hline
        4 & 1 \\
        \hline
        5 & 1 \\
        \hline
        6 & 3 \\
        \hline
    \end{tabular}
\end{center}

Ad occhio si ricava $<1, 3, 6>$. \\

L'algoritmo naive e' quello combinatorio, ma e' estremamente inefficiente. Il tempo e' $T(n) = \Omega(2^n)$.

\paragraph{Soluzione PD}

\begin{itemize}
    \item $n \leftrightarrow X = {1,..,n}$
    \item $\forall i \in {1,..,n}$, $s _ {i}$ e' il tempo di inizio dell'attivita' $i$
    \item $\forall i \in {1,..,n}$, $f _ {i}$ e' il tempo di fine dell'attivita' $i$
    \item $\forall i \in {1,..,n}$, $v _ {i}$ e' il valore dell'attivita' i
\end{itemize}

Definisco la funzione: \\*
$COMP : \P({1,..,n}) \rightarrow {true, false}$ \\*
$\forall i,j \in A \mid i \ne j \lor attivitaCompatibili(A_i,A_j) \Rightarrow COMP(A) = true$ \\*

Dette poi $i,j$ due attivita' si dice:
\[
    attivitaCompatibili(i,j) =
    \begin{cases}
        \text{$true$} & \text{$[s_i, f_i) \land [s_j, f_j) = \emptyset$} \\
        \text{$false$} & altrimenti
    \end{cases}
\]

Quindi si definisce: \\*
$V : \mathbb{P}({1,..,n}) \rightarrow \R$
\[
    V(i,j) =
    \begin{cases}
        \text{$\Sigma _ {i \in A} v_i$} & \text{$A \ne \emptyset$} \\
        \text{$0$} & \text{$A = \emptyset$}
    \end{cases}
\]

La Soluzione e' $S \subseteq X \mid COMP(S) = true \land \forall A \subset X V(S) \leq V(A)$

\paragraph{Processo}

Detto $S_n \Leftrightarrow sol(X_n)$, e quindi $S_{n-k} \Leftrightarrow sol(X_{n-k})$.
Nella soluzione di $S_n$ si assume di conoscere:

\begin{itemize}
    \item $\forall k \in {1,..,n} S_{n-k}$
    \item $\forall k \in {1,..,n} sol(X_{n-k})$
\end{itemize}

Dividendo in sottoproblemi e potendo disponere a piacimento di ognuno di questi, riesco facilmente a individuare il caso base.
E' immediato sapere $sol(\emptyset)$ e $sol({1})$, quindi questi possono essere i casi base.
