\chapter{11-10-2022}

\section{Scheduling di Attivita'}
Un esempio di Programmazione per Intervalli Pesati. \\*
Siano date $n$ attivita' da svolgersi nello stesso spazio fisico.
Determinare un sottoinsieme di attivita' che non si sovrappongono e che sia il massimo possibile.

\begin{center}
    \begin{tabular}{||c c c||}
        \hline
        i & p(i) & v \\
        \hline
        \hline
        1 & 0 & 10 \\
        \hline
        2 & 0 & 2 \\
        \hline
        3 & 2 & 8 \\
        \hline
        4 & 2 & 1 \\
        \hline
        5 & 1 & 1 \\
        \hline
        6 & 4 & 3 \\
        \hline
    \end{tabular}
\end{center}

Ad occhio si ricava $<1, 3, 6>$. \\

L'algoritmo naive e' quello combinatorio, ma e' estremamente inefficiente. Il tempo e' $T(n) = \Omega(2^n)$.

\paragraph{Soluzione PD}

\begin{itemize}
    \item $n \leftrightarrow X = {1,..,n}$
    \item $\forall i \in {1,..,n}$, $s _ {i}$ e' il tempo di inizio dell'attivita' $i$
    \item $\forall i \in {1,..,n}$, $f _ {i}$ e' il tempo di fine dell'attivita' $i$
    \item $\forall i \in {1,..,n}$, $v _ {i}$ e' il valore dell'attivita' i
\end{itemize}

Definisco la funzione: \\*
$COMP : \mathbb{P}(\{1,..,n\}) \rightarrow \{true, false\}$ \\*
$\forall i,j \in A \mid i = j \lor attivitaCompatibili(A_i,A_j) \Rightarrow COMP(A) = true$ \\*

Dette poi $i,j$ due attivita' si dice:
\[
    attivitaCompatibili(i,j) =
    \begin{cases}
        \text{$true$} & \text{$[s_i, f_i) \cap [s_j, f_j) = \emptyset$} \\
        \text{$false$} & altrimenti
    \end{cases}
\]

Quindi si definisce: \\*
$V : \mathbb{P}(\{1,..,n\}) \rightarrow \mathbb{R}$
\[
    V(i,j) =
    \begin{cases}
        \text{$\Sigma _ {i \in A} v_i$} & \text{$A \ne \emptyset$} \\
        \text{$0$} & \text{$A = \emptyset$}
    \end{cases}
\]

La Soluzione e' $(S \subseteq X \mid COMP(S) = true) \land (\forall A \subseteq X \mid V(S) \geq V(A))$

\paragraph{Processo}

Detto $S_n \Leftrightarrow sol(X_n)$, e quindi $S_{n-k} \Leftrightarrow sol(X_{n-k})$.
Nella soluzione di $S_n$ si assume di conoscere:

\begin{itemize}
    \item $\forall k \in {1,..,n} \; S_{n-k}$
    \item $\forall k \in {1,..,n} \; sol(X_{n-k})$
\end{itemize}

Detto $OPT(i) = V(S_i)$.
Dividendo in sottoproblemi e potendo disponere a piacimento di ognuno di questi, riesco facilmente a individuare il caso base.
E' immediato sapere $sol(\emptyset)$ e $sol({1})$, quindi questi possono essere i casi base.

\paragraph{Ragionamento}

\subparagraph{Caso Base}

\begin{align}
    \text{$S_0  \Leftrightarrow X_0 = \emptyset \land V(X_0) = 0$} \\
    \text{$S_1  \Leftrightarrow X_1 = {1} \land V(X_1) = 1$}
\end{align}

\subparagraph{Caso Ricorsivo}

Voglio risolvere $S_i$ e $OPT(i)$. \\*
Assumo di avere gia' risolto $\forall j \in X \mid j < i S_j$. \\*

Se sapessi che $i \in S_i$ il sotto problema da considerare sarebbe: \\*
$sol(\{j \mid \forall j \in X_{i-1} \mid attivitaCompatibile(i, j)\})$. \\
Questo si traduce nel risolvere $S_j$, dove $j$ e' il massimo indice di una attivita' compatibile con $i$. \\*
Se sapessi al contrario che $i \in S_i$ allora dovrei risolvere $S_{i-1}$.

Detto $p(i) : X \rightarrow X$, la funzione che associa ad ogni attivita', l'indice dell'attivita' compatibile precedente piu' vicina.
Potendo approssimare, per questioni di prestazione dei confronti, il problema $S_j$ ad al problema leggermente piu' grande $S_{p(i)}$, mi riduco a:

\begin{align}
    \text{$S_{p(i)} \cup {i}$} & \text{$V(p(i)) + v_i \geq V(i-1)$} \\
    \text{$S_{i-1}$} & \text{$V(p(i)) + v_i < V(i-1)$} \\
\end{align}

Che verranno dimostrati successivamente.
Il tutto si traduce in:

\[
    S_i =
    \begin{cases}
        \text{$0$} & \text{$i = 0$} \\
        \text{$1$} & \text{$i = 1$} \\
        \text{$S_{p(i)} \cup {i}$} & \text{$V(p(i)) + v_i \geq V(i-1)$} \\
        \text{$S_{i-1}$} & \text{$V(p(i)) + v_i < V(i-1)$} \\
    \end{cases}
\]

Ovvero, riprendo i casi base esattamente come scritti sopra, e in piu' vincolo la scelta della soluzione (e di fatto del sottoproblema $S_{i-1}$ o $S_{p(i)}$) all'unico criterio che e' in grado di stabilire la buona riuscita dell'ottimizzazione.

\pagebreak

\paragraph{Procedura TOP-DOWN}

\begin{algorithm}
    \begin{algorithmic}
        \Procedure{WIS-OPT}{$i$}
            \If{$i = 0$}
                \State \Return 0
            \ElsIf{i = 1}
                \State \Return 1
            \Else
                \State $Z1 \gets append(WIS-OPT(p(i)), x_i)$
                \State $Z2 \gets WIS-OPT(i - 1)$
                \If {$OPT(Z1) \geq Z2$}
                    \State \Return Z1
                \Else
                    \State \Return Z2
                \EndIf
            \EndIf
       \EndProcedure
    \end{algorithmic}
\end{algorithm}

\pagebreak

\paragraph{Procedura BOTTOM-UP}

\begin{algorithm}
    \begin{algorithmic}
        \Procedure{INIZIALIZZA-VETTORI}{}
            \State $OPT[0] \gets 0$
            \State $OPT[1] \gets 1$
            \State $WIS[0] \gets X_0$
            \State $WIS[1] \gets X_1$
        \EndProcedure
    \end{algorithmic}
\end{algorithm}

\begin{algorithm}
    \begin{algorithmic}
        \Procedure{WIS-OPT-ITER}{$i$}
            \State $INIZIALIZZA-VETTORI()$
            \For{$i = 2$ to $n$}
                \State $Z1 \gets append(WIS[p(i)], x_i)$
                \State $Z2 \gets WIS[i-1]$
                \If{$OPT[p(i)] < OPT[i-1]$}
                    \State $OPT[i] \gets OPT[p(i)] + v_i$
                    \State $WIS[i] \gets Z1$
                \Else
                    \State $OPT[i] \gets OPT[i-1]$
                    \State $WIS[i] \gets Z2$
                \EndIf
            \EndFor
            \State \Return WIS[n]
        \EndProcedure
    \end{algorithmic}
\end{algorithm}

Il vettore $WIS$ e' implementabile ragionevolmente con una matrice di booleani che caratterizzano la presenza di un elemento nell'insieme. Visto che questo richiederebbe una quantita' di spazio non indifferente, una cosa comoda potrebbe essere codificare le righe o le colonne in un numero intero decimale.

\paragraph{Complessita'}

La procedura WIS-OPT-ITER comprende un solo ciclo di $\Theta(n)$ iterazioni.
Il calcolo di $p(i)$ richiede $O(n)$, perche' e' un ciclo inverso semplice che dipende dalla disposizione degli elementi in $X$.
Quindi l'algoritmo e' $T(n) = \O(n^2)$ nel caso medio.

\paragraph{Osservazioni}

E' possibile scrivere una procedura che esplori linearmente l'array $OPT$ per verificare i passi che sono stati effettuati per costruire OPT. Quindi si puo' risparmiare lo spazio occupato da $WIS$.

\paragraph{Dimostrazione}

\subparagraph{}
\textbf{$S_i = S_{p(i)} \cup {i}$} \\
\begin{enumerate}
    \item Assumo che $i \in S_i$, devo mostrare che $S_i = S_{p(i)} \cup {i}$. \\
    \item Si supponga per assurdo che $S_{p(i)} \cup {i} \ne sol(X_i)$. \\
    \item Chiamo $S^I$ la soluzione ${S_i}$. \\
    \item Posso affermare con certezza che $V(S^I) > V(S_{p(i)}) + v_i$. \\
    \item Visto che $i \in S_i$, deduco che allora $S^I = S^{II} \cup {i}$. \\
    \item Ragionevolmente $S^{II}$ contiene attivita' compatibili insieme a $i$. \\
    \item Per la riduzione di qualche paragrafo precedente posso affermare che $S^{II} \subseteq \{1,..,p(i)\}$.
    \item Ma allora $V(S^{II}) > V(S_{p(i)})$. \\
    \item Il che e' impossibile, perche' sappiamo che $S_{p(i)} = sol(X_{p(i)})$. \\
    \item Dovendo essere necessariamente $S_{p(i)} = sol(X_{p(i)})$, ne ricaviamo che $S^{II} = S_{p(i)}$.
    \item Quindi $S_i = S_{p(i)} \cup {i}$.
\end{enumerate}

\subparagraph{}
\textbf{$S_i = S_{i-1}$} \\

\begin{enumerate}
    \item Assumo che $i \notin S_i$, devo mostrare che $S_i = S_{i-1}$. \\
    \item Supponiamo per assurdo che $S_i \ne S_{i-1}$ \\
    \item Allora necessariamente $\exists S^I \ne S_{i-1} \mid V(S^I) > V(S_{i-1})$. \\
    \item Visto che $S^I = S_i$ allora $i \notin S^I$, dato che $i \notin S_i$. \\
    \item Ma quindi allora $S^I \subseteq {1,..,i-1} \mid V(S^I) > V(S_{i-1})$.
    \item Il che e' impossibile, perche' sappiamo che $S_{i-1} = sol(X_{i-1})$. \\
    \item Dovendo essere necessariamente $S_{i-1} = sol(X_{i-1})$, ne ricaviamo che $S^{I} = S_{i-1}$. \\
    \item Quindi $S_i = S_{i-1}$.
\end{enumerate}

\paragraph{Consegna Esercizio}
Da risolvere per 18-10-2022.

L'istanza e' simile ma con le case al posto delle attivita'.
Ci sono n case adiacenti in linea retta.
Ad ogni casa e' associato un valore $d$, la donazione che l'abitante e' disposto a fare.
Trovare un sottoinsieme di abitanti massimo tale che le case non siano adiacenti.

\[
    S_i =
    \begin{cases}
        \text{$0$} & \text{$i = 0$} \\
        \text{$1$} & \text{$i = 1$} \\
        \text{$S_{p(i)} \cup {i}$} & \text{$V(p(i)) + v_i \geq V(i-1)$} \\
        \text{$S_{i-1}$} & \text{$V(p(i)) + v_i < V(i-1)$} \\
    \end{cases}
\]

In questo caso $p(i)$ scorre l'array di case fino a individuare la prima non adiacente.
Questa operazione e' logicamente a tempo $T(n) = \Theta(1)$, perche' si parla di spostarsi a sinistra di due case.

\pagebreak

\paragraph{Procedura Esercizio}

\begin{algorithm}
    \begin{algorithmic}
        \Procedure{INIZIALIZZA-VETTORI}{}
            \State $OPT[0] \gets 0$
            \State $OPT[1] \gets 1$
            \State $WIS[0] \gets X_0$
            \State $WIS[1] \gets X_1$
        \EndProcedure
    \end{algorithmic}
\end{algorithm}

\begin{algorithm}
    \begin{algorithmic}
        \Procedure{ESERCIZIO}{$i$}
            \State $INIZIALIZZA-VETTORI()$
            \For{$i = 2$ to $n$}
                \State $Z1 \gets append(WIS[p(i)], x_i)$
                \State $Z2 \gets WIS[i-1]$
                \If{$OPT[p(i)] < OPT[i-1]$}
                    \State $OPT[i] \gets OPT[p(i)] + v_i$
                    \State $WIS[i] \gets Z1$
                \Else
                    \State $OPT[i] \gets OPT[i-1]$
                    \State $WIS[i] \gets Z2$
                \EndIf
            \EndFor
            \State \Return WIS[n]
        \EndProcedure
    \end{algorithmic}
\end{algorithm}

\pagebreak

\paragraph{Esercizio Considerazioni}

Valgono le considerazioni di WIS-OPT-ITER, ma questa volta $T(n) = \Theta(n)$, perche' la complessita' di $p(i)$ e' mutata.
