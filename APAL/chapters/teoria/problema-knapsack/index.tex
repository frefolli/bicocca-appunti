\chapter{Problema Knapsack}
\section{Introduzione}

Il problema dello zaino o Knapsack, e' un problema di ottimizzazione combinatoria che prevede di trovare un'opportuna composizione di multiinsieme con peso inferiore o uguale alla capacita' fissata affinche' il valore (o premio) totale sia ottimale (il massimo possibile).
Si hanno: \\

\begin{itemize}
    \item $N$ oggetti con un peso $w_{i}$ e un valore $c_{i}$.
    \item il peso massimo $W$ sopportabile dallo zaino.
    \item la quantita' di oggetto i-esimo $x_i$ inserito nello zaino.
\end{itemize}

Di questo problema esistono varianti che fissano ulteriori vincoli sul multiinsieme.

\paragraph{Knapsack Senza Limiti}

In questa versione lo zaino puo' contenere un oggetto in una quantita' $x_i \geq 0$ a piacere e illimitata superiormente.

\paragraph{Knapsack Con Limiti}

Ad ogni elemento i-esimo e' associata una disponibilita' massima $d_i \geq 0$ tale che $x_i \leq d_i$.

\paragraph{Knapsack 0-1}

I limiti di disponibilita' degli oggetti sono binari, ovvero $d_i = 1$.

\chapter{Knapsack Senza Limiti}

\section{Ragionamento}

\paragraph{Intuizione}

Si riduce il problema al cercare il massimo valore della soluzione. \\
Quindi ragiono in termini di capacita' $k \leq W$ dello zaino con $A(k)$ la soluzione del problema ridotto con massimo peso $k$. \\
Si puo' quindi ragionevolmente pensare che se k = 0 allora A(k) = 0. \\
E' naturale pensare che $\forall j \leq k \mid A(k) \geq A(j)$. \\
D'altra parte se $A(k) > 0$ allora $\exists j \in [1,N]$ tale che $w_j \leq k \land A(k) = A(k-w_j) + c_j$. \\
Questo implica ovviamente anche che se $A(k) = A(k-w_j) + c_j$ allora $\forall i \in [1,N] \mid w_i \leq k \Rightarrow A(k-w_j) + c_j \geq A(k-w_i) + c_i$. \\
Di conseguenza posso impostare la soluzione $A(k)$ ricorsivamente rispetto a tutti i pesi $w_j \leq k$ e adoperare la sottosoluzione migliore.

\paragraph{Caso Base}

\[
    \begin{cases}
        \text{$0$} & \text{$k = 0$}
    \end{cases}
\]

\paragraph{Caso Ricorsivo}

\[
    \begin{cases}
        \text{$max(\{\forall j \in [1,N] \land w_j \leq k \mid A(k - w_j) + c_j\})$} & \text{$k > 0$}
    \end{cases}
\]

\paragraph{Caso Generale}

\[
    \begin{cases}
        \text{$0$} & \text{$k = 0$} \\
        \text{$max(\{\forall j \in [1,N] \land w_j \leq k \mid A(k - w_j) + c_j\})$} & \text{$k > 0$}
    \end{cases}
\]

\newpage

\section{Procedura PD}

Detti $X$ i pesi, $Y$ i valori, $W$ la capacita' massima.

\begin{algorithm}
    \begin{algorithmic}
        \Procedure{KS}{$X, Y, W$}
            \State $A[0] \gets 0$
            \For{$k = 1$ to $W$}
                \State $max \gets 0$
                \For{$j = 0$ to $N$}
                    \If{$x_j \leq k \land A[k - x_j] + y_j > max$}
                        \State $max \gets A[k - x_j] + y_j$
                    \EndIf
                \EndFor
                \State $A[k] \gets max$
            \EndFor
        \EndProcedure
    \end{algorithmic}
\end{algorithm}

\paragraph{Ricostruzione}

Per ricostruire il Knapsack e' sufficiente mantenere uno storico dei passi e seguire il percorso a partire da $A[W]$.
In alternativa si puo' simulare la ricorsione top-down con $A[i]$ al posto della chiamata ricorsiva.

\paragraph{Complessita'}

La complessita' temporale e' $O(N \cdot W)$.
Questa complessita' temporale e' pseudo-polinomiale perche' $W$ e' proporzionale alla sua lunghezza in bit.

\chapter{Knapsack 0-1}

\section{Ragionamento}

Oltre a iterare sul peso a disposizione e' utile iterare sulla varieta' degli oggetti a disposizione. \\

Detti quindi $i \in [0, N]$, $j \in [0, k]$, $A(i, j)$ e' la soluzione con $i$ oggetti e capacita' massima $j$. \\
Ovviamente $j = 0 \lor i = 0 \Rightarrow A(i, j) = 0$. \\
Naturalmente se $w_i > j$ allora $A(i, j) = A(i - 1, j)$. \\
Se invece $w_i \leq j$ le uniche opzioni possibili sono di prendere o non prendere l'oggetto. \\
Se $i \in Knapsack$ allora $A(i, j) = A(i - 1, j - w_i) + c_i$, altrimenti $A(i, j) = A(i - 1, j)$. \\
Per stabilirlo e' sufficiente prelevare la sottosoluzione migliore, ovvero il valore totale piu' grande.

\paragraph{Caso Base}

\[
    \begin{cases}
        \text{$0$} & \text{$i = 0 \lor j = 0$}
    \end{cases}
\]

\paragraph{Caso Ricorsivo}

\[
    \begin{cases}
        \text{$A(i - 1, j)$} & \text{$w_i > j$} \\
        \text{$A(i - 1, j - w_i) + c_i$} & \text{$w_i \leq j \land A(i - 1, j - w_i) + c_i \geq A(i - 1, j)$} \\
        \text{$A(i - 1, j)$} & \text{$w_i \leq j \land A(i - 1, j - w_i) + c_i < A(i - 1, j)$}
    \end{cases}
\]

\paragraph{Caso Generale}

\[
    \begin{cases}
        \text{$0$} & \text{$i = 0 \lor j = 0$} \\
        \text{$A(i - 1, j)$} & \text{$w_i > j$} \\
        \text{$A(i - 1, j - w_i) + c_i$} & \text{$w_i \leq j \land A(i - 1, j - w_i) + c_i \geq A(i - 1, j)$} \\
        \text{$A(i - 1, j)$} & \text{$w_i \leq j \land A(i - 1, j - w_i) + c_i < A(i - 1, j)$}
    \end{cases}
\]

\newpage

\section{Procedura PD}

Detti $X$ i pesi, $Y$ i valori, $W$ la capacita' massima.

\begin{algorithm}
    \begin{algorithmic}
        \Procedure{INIZIALIZZA}{$N, W$}
            \For{$i = 0$ to $N$}
                \State $M[i][0] \gets 0$
            \EndFor
            \For{$j = 0$ to $W$}
                \State $M[0][j] \gets 0$
            \EndFor
        \EndProcedure
    \end{algorithmic}
\end{algorithm}

\begin{algorithm}
    \begin{algorithmic}
        \Procedure{KS}{$X, Y, W$}
            \State INIZIALIZZA($N, W$)
            \For{$i = 0$ to $N$}
                \For{$j = 0$ to $W$}
                    \If{$x_i \leq j$}
                        \If{$M[i - 1][j - x_i] + y_i \geq M[i - 1][j]$}
                            \State $M[i][j] \gets M[i - 1][j - x_i] + y_i$
                        \Else
                            \State $M[i][j] \gets M[i - 1][j]$
                        \EndIf
                    \Else
                        \State $M[i][j] \gets M[i - 1][j]$
                    \EndIf
                \EndFor
            \EndFor
        \EndProcedure
    \end{algorithmic}
\end{algorithm}

\paragraph{Ricostruzione}

Per ricostruire il Knapsack e' sufficiente mantenere uno storico degli svincoli e seguire il percorso a partire da $M[N][W]$.
In alternativa si puo' simulare la ricorsione top-down con $M[i][j]$ al posto della chiamata ricorsiva.

\paragraph{Complessita'}

La complessita' temporale e' $O(N \cdot W)$.
Questa complessita' temporale e' pseudo-polinomiale perche' $W$ e' proporzionale alla sua lunghezza in bit.

\chapter{Knapsack Con Limiti}

\section{Ragionamento}

TODO

\section{Procedura PD}

TODO