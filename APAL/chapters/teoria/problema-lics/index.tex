\chapter{Problema LICS}

\section{Introduzione}

\paragraph{LICS}

\subparagraph{Introduzione}

Il problema LICS (Longest Increasing Common Subsequence) consiste nel cercare in tempo ragionevole la sottosequenza comune piu' lunga crescente all'interno di due stringhe o sequenze.

\subparagraph{Esempi}

\begin{itemize}

\item

$X$ = <3,4,9,1> \\
$Y$ = <5, 3, 8, 9, 10, 2, 1> \\

La soluzione sarebbe $Z$ = <3,9>

\end{itemize}

\section{Algoritmo}

\paragraph{Ragionamento}

\subparagraph{Notazione}

\begin{align}
    \text{$X_i = <x_1, x_2, x_3, x_4, x_5, x_6, .., x_i>$} \\
    \text{$Y_i = <y_1, y_2, y_3, y_4, y_5, y_6, .., y_j>$} \\
    \text{$Z_k = <z_1, z_2, z_3, .., z_k>$} \\
    \text{$Z = LIS(X, Y)$}
\end{align}

Definisco una funzione $V : R \times R \rightarrow R$ tale che: $V(i, j) = |LICS(X_i, Y_j)|$ \\

E una funzione analoga alla $P(i)$ di $LIS$: \\

Definisco una funzione $P : R \times R \rightarrow R \times R$ tale che: \\*
$P(i, j) = max_{V(p, q)}(\{ \forall p,q \mid p \in [1, i) \land q \in [1, j) \mid z_{k} < x_i \land z_{k} < y_i\})$. \\

\subparagraph{Caso 1}

$x_i = y_i \land x_i = z_k$ \\

$Z_k = Z_{k-1} \cup \{x_i\}$ \\*
$Z_{k-1} = LICS(X_p, Y_q)$ con $p \in [1, i) \land q \in [1, j)$ \\

Data la definizione di LICS: \\*
$\forall w \in [1, k) \;\; z_{w} < x_i$. \\*
$\forall w \in [1, k) \;\; z_{w} < y_j$. \\

In particolare: \\*
$z_{k-1} < x_i$ \\*
$z_{k-1} < y_j$ \\

Ora, visto che $x_i = y_i$, la $P(i, j)$ si semplifica: \\*
$P(i, j) = max_{V(p, q)}(\{ \forall p,q \mid p \in [1, i) \land q \in [1, j) \mid z_{k} < x_i\})$. \\

Quindi scrivo: \\*
$\exists p,q \mid P(i, j) = (p, q) \Rightarrow Z_{k-1} = LICS(X_p, Y_q)$ \\

\subparagraph{Caso 2}

$x_i = y_i \land x_i \ne z_k$ \\

Logicamente, devo escludere questi valori: \\*
$Z_k = LICS(X_{i-1}, Y_{j-1})$ \\

\subparagraph{Caso 3}

$x_i \ne y_i \land x_i = z_k \Rightarrow y_j \ne z_k$ \\
$Z_k = LICS(X_{i}, Y_{j-1})$ \\

\subparagraph{Caso 4}

$x_i \ne y_i \land y_j = z_k \Rightarrow x_i \ne z_k$ \\
$Z_k = LICS(X_{i-1}, Y_{j})$ \\

\subparagraph{Caso 5}

$x_i \ne y_i \land x_i \ne z_k \land y_j \ne z_k$ \\
$Z_k = LICS(X_{i-1}, Y_{j-1})$ \\

\paragraph{Ricorrenza}

Non conoscendo a priori il valore di Z, l'esplorazione binaria fara' si che si esplori il Caso 5 come sottocaso dei casi 3 e 4.
Da qui, $Z_{i,j} = LIS(X_i, Y_j)$

\subparagraph{Caso Base}

\begin{align}
    \text{$i = 0 \Rightarrow Z_{i,j} = <>$} \\
    \text{$j = 0 \Rightarrow Z_{i,j} = <>$}
\end{align}

\subparagraph{Caso Ricorsivo}

\[
    Z_i =
    \begin{cases}
        \text{$Z_{P(i,j)}$} & \text{$(x_i = y_j) \land (|Z_{P(i,j)}| + 1 \geq |Z_{i-1, j-1}|)$} \\
        \text{$Z_{i-1, j-1}$} & \text{$(x_i = y_j) \land (|Z_{P(i,j)}| + 1 < |Z_{i-1, j-1}|)$} \\
        \text{$Z_{i, j-1}$} & \text{$(x_i \ne y_j) \land (|Z_{i, j-1}| \geq |Z_{i-1, j}|)$} \\
        \text{$Z_{i-1, j}$} & \text{$(x_i \ne y_j) \land (|Z_{i, j-1}| < |Z_{i-1, j}|)$} \\
    \end{cases}
\]

\subparagraph{Caso Generale}

\[
    Z_i =
    \begin{cases}
        \text{$<>$} & \text{$i = 0 \lor j = 0$} \\
        \text{$Z_{P(i,j)}$} & \text{$(x_i = y_j) \land (|Z_{P(i,j)}| + 1 \geq |Z_{i-1, j-1}|)$} \\
        \text{$Z_{i-1, j-1}$} & \text{$(x_i = y_j) \land (|Z_{P(i,j)}| + 1 < |Z_{i-1, j-1}|)$} \\
        \text{$Z_{i, j-1}$} & \text{$(x_i \ne y_j) \land (|Z_{i, j-1}| \geq |Z_{i-1, j}|)$} \\
        \text{$Z_{i-1, j}$} & \text{$(x_i \ne y_j) \land (|Z_{i, j-1}| < |Z_{i-1, j}|)$} \\
    \end{cases}
\]

\newpage

\section{Procedura BOTTOM-UP}

La matrice $C$: $C[i][j] = |LICS(X_i, Y_j)|$. \\*
La matrice $L$: $L[i][j] = $LAST-ELEMENT($LICS(X_i, Y_j)$). \\*
La matrice $H$: $H[i][j] = P(i,j)$. \\

\begin{algorithm}
    \begin{algorithmic}
        \Procedure{LICS}{$X, Y$}
            \State $INIZIALIZZA(|X|, |Y|)$
            \For{$i = 1$ to $n$}
                \For{$j = 1$ to $m$}
                    \If{$x_i = y_i$}
                        \State $(p, q) \gets P(i, j)$
                        \State $H[i][j] \gets (p, q)$
                        \If{$C[p][q] + 1 \geq C[i-1][j-1]$}
                            \State $C[i][j] \gets C[p][q] + 1$
                            \State $L[i][j] \gets (i,j)$
                        \Else
                            \State $C[i][j] \gets C[i-1][j-1]$
                            \State $L[i][j] \gets L[i-1][j-1]$
                        \EndIf
                    \Else
                        \If{$C[i][j-1] \geq C[i-1][j]$}
                            \State $C[i][j] \gets C[i][j-1]$
                            \State $L[i][j] \gets L[i][j-1]$
                        \Else
                            \State $C[i][j] \gets C[i-1][j]$
                            \State $L[i][j] \gets L[i-1][j]$
                        \EndIf
                    \EndIf
                \EndFor
            \EndFor
        \EndProcedure
    \end{algorithmic}
\end{algorithm}

\begin{algorithm}
    \begin{algorithmic}
        \Procedure{INIZIALIZZA}{$n, m$}
            \For{$i = 0$ to $n$}
                \State $C[i][0] \gets 0$
                \State $H[i][0] \gets (0,0)$
            \EndFor
            \For{$j = 0$ to $m$}
                \State $C[0][j] \gets 0$
                \State $H[0][j] \gets (0,0)$
            \EndFor
        \EndProcedure
    \end{algorithmic}
\end{algorithm}

\begin{algorithm}
    \begin{algorithmic}
        \Procedure{P}{$i, j$}
            \State $p \gets 0$
            \State $q \gets 0$
            \For{$w = 1$ to $i-1$}
                \For{$z = 1$ to $j-1$}
                    \If{$L[w][z][0] = 0 \land C[w][z] = 0$}
                        \State $p \gets w$
                        \State $q \gets z$
                    \ElsIf{$x_{L[w][z][0]} < x_i \land C[w][z] > C[p][q]$}
                        \State $p \gets w$
                        \State $q \gets z$
                    \EndIf
                \EndFor
            \EndFor
            \State \Return $(p,q)$
        \EndProcedure
    \end{algorithmic}
\end{algorithm}

\begin{algorithm}
    \begin{algorithmic}
        \Procedure{LEGGI-LICS}{$X, n, m$}
            \State $i \gets n$
            \State $j \gets m$
            \State $Z \gets <>$
            \While{$(i,j) \ne (0,0)$}
                \If{$(i,j) = L[i][j]$}
                    \State $Z \gets prepend(Z, x_i)$
                    \State $(i,j) \gets H[i][j]$
                \Else
                    \State $(i,j) \gets L[i][j]$
                \EndIf
            \EndWhile
            \State \Return Z
        \EndProcedure
    \end{algorithmic}
\end{algorithm}

\paragraph{Complessita'}
$T(n) = \Theta(n \cdot m) * O(n \cdot m) + O(n)$ \\*
$T(n) = O(n^2 \cdot m^2) + O(n) = O(n^2 \cdot m^2)$

\newpage

\section{Variante Rizzi}

$LICS_V$ e' un sottoproblema di LICS, ovvero la LICS che termina con l'elemento i-esimo di $X_i$. \\*
La matrice $C$: $C[i][j] = |LICS_V(X_i, Y_j)|$. \\*
La matrice $H$: $H[i][j] = P(i, j)$. \\

\begin{algorithm}
    \begin{algorithmic}
        \Procedure{LICS}{$X, Y$}
            \State $Z \gets 0$
            \For{$i = 1$ to $m$}
                \For{$j = 1$ to $n$}
                    \If{$x_i \ne y_j$}
                        \State $C[i][j] \gets 0$
                    \Else
                        \State $max \gets 0$
                        \State $H[i][j] \gets (0,0)$
                        \For{$h = 1$ to $i-1$}
                            \For{$k = 1$ to $j-1$}
                                \If{$C[h][k] > max \land x_h < x_i$}
                                    \State $max \gets C[h][k]$
                                    \State $H[i][j] \gets (h,k)$
                                \EndIf
                            \EndFor
                        \EndFor
                        \State $C[i][j] \gets 1 + max$
                        \If{$C[i][j] > Z$}
                            \State $Z \gets C[i][j]$
                        \EndIf
                    \EndIf
                \EndFor
            \EndFor
            \State \Return Z
        \EndProcedure
    \end{algorithmic}
\end{algorithm}

\begin{algorithm}
    \begin{algorithmic}
        \Procedure{BUILD-LICS}{X}
            \If{$H[i][j] \ne (0,0)$}
                \State BUILD-LICS($H[i][j]$)
            \EndIf
            \State $print(x_i)$
        \EndProcedure
    \end{algorithmic}
\end{algorithm}

\paragraph{Complessita'}
$T(n) = \Theta(n \cdot m) * O(n \cdot m) + O(n)$ \\*
$T(n) = O(n^2 \cdot m^2) + O(n) = O(n^2 \cdot m^2)$
