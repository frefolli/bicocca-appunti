\chapter{Kruskal}

\section{l'Algoritmo}

L'algoritmo di Kruskal e' una rielaborazione dell'algoritmo per le Componenti Connesse di un grafo. Sia un grafo G = (E, V). \\
Creo una foresta di $|V|$ alberi, ognuno contiene solo un vertice del grafo, tramite la struttura dati degli Insiemi Disgiunti. \\
L'Albero di Copertura Minimo F e' rappresentato tramite una lista di archi. \\
Si itera sugli archi ordinati in senso crescente rispetto al peso. Ad ogni iterazione si cerca di collegare due alberi disgiunti della foresta.
Visto che si visitano per primi gli archi piu' leggeri, e' triviale che l'Albero di Copertura cosi' generato sara' minimo.

\newpage

\section{Procedura K}

\begin{algorithm}
    \begin{algorithmic}
        \Procedure{K}{$V,E$}
            \State $F \gets \emptyset$
            \For{$v \in V$}
                \State $makeset(v)$
            \EndFor
            \For{$(v_i, v_j) \in SORT_{weight}(E)$}
                \If{$findset(v_i) \ne findset(v_j)$}
                    \State $union(v_i, v_j)$
                    \State $F \gets F \cup \{(v_i, v_j)\} \cup \{(v_j, v_i)\}$
                \EndIf
            \EndFor
        \EndProcedure
    \end{algorithmic}
\end{algorithm}