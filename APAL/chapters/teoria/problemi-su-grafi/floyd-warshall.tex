\chapter{Floyd-Warshall APSP}

\section{l'Algoritmo}

Questo algoritmo risolve il All Pairs Shortest Path.
Un cammino da $v_{i}$ a $v_{j}$ e' una sequenza finita di vertici $s \subseteq V$, con $s_1 = v_{i}$ e $s_{|s|} = v_{j}$, tali che $\forall i \in [1, |s|) \mid \exists e \in E \;\; con \;\; e = (s_{i}, s_{i+1})$. Il peso del cammino e' la somma dei pesi degli archi che lo compongono. \\

Questo algoritmo prende in input i pesi sottoforma di Matrice di Adiancenza, le cui celle:
\begin{align*}
    i = j & \Rightarrow w_{i,j} = 0 \\
    i \ne j \land (i,j) \in E & \Rightarrow w_{i,j} = p_{i,j} \\
    i \ne j \land (i,j) \notin E & \Rightarrow w_{i,j} = \infty
\end{align*}

L'algoritmo restituisce due Matrici di Adiacenza tali che:
\begin{align*}
    i = j & \Rightarrow d_{i,j} = 0 \\
    i \ne j \land \exists cammino(i, j) & \Rightarrow d_{i,j} = \text{peso del cammino} \\
    i \ne j \land \nexists cammino(i, j) & \Rightarrow d_{i,j} = \infty
\end{align*}

\begin{align*}
    i = j & \Rightarrow \pi_{i,j} = NIL \\
    i \ne j \land \exists cammino(i, j) & \Rightarrow \pi_{i,j} = u \land (u,j) \in E \\
    i \ne j \land \nexists cammino(i, j) & \Rightarrow \pi_{i,j} = NIL
\end{align*}

\section{Riempimento di $d_{i,j,k}$}

Ora si ipotizzi di disporre per il cammino minimo intermedio a $(v_i,v_j)$ di tutti e soli i primi $k$ vertici.
La matrice $d_{i,j,k}$ quindi contiene il peso del cammino minimo che dispone dei primi $k$ vertici.

\paragraph{Caso Base}

\[
    d_{i,j,k} = 
    \begin{cases}
        0 & k = 0 \land i = j\\
        w_{i,j} & k = 0 \land (i,j) \in E \\
        \infty & k = 0 \land (i,j) \notin E
    \end{cases}
\]

Questo corrisponde a dire che $d_{i,j,0} = w_{i,j}$.
Ad ogni modo la soluzione sara' contenuta in $d_{i,j,n}$.

\paragraph{Caso Ricorsivo}

Ora si presentano due scenari.

\begin{align*}
    k \in cammino & \Rightarrow d_{i,j,k} = d_{i,k,k-1} + d_{k,j,k-1} \\
    k \notin cammino & \Rightarrow d_{i,j,k} = d_{i,j,k-1}
\end{align*}

Visto che si tratta del cammino minimo, allora $k \in cammino$ sse $d_{i,k,k-1} + d_{k,j,k-1} < d_{i,j,k-1}$.
Quindi:

\[
    d_{i,j,k} = 
    \begin{cases}
        d_{i,k,k-1} + d_{k,j,k-1} & d_{i,k,k-1} + d_{k,j,k-1} \leq d_{i,j,k-1} \\
        d_{i,j,k-1} & d_{i,k,k-1} + d_{k,j,k-1} > d_{i,j,k-1} 
    \end{cases}
\]

\section{Riempimento di $\pi_{i,j,k}$}

\paragraph{Caso Base}

\[
    \pi_{i,j,k} = 
    \begin{cases}
        NIL & k = 0 \land i = j\\
        i & k = 0 \land (i,j) \in E \\
        NIL & k = 0 \land (i,j) \notin E
    \end{cases}
\]

\paragraph{Caso Ricorsivo}

Usiamo ora il caso ricorsivo di $d_{i,j,k}$:

\[
    \pi_{i,j,k} = 
    \begin{cases}
        \pi_{i,j,k-1} & d_{i,k,k-1} + d_{k,j,k-1} < d_{i,j,k-1} \\
        \pi_{k,j,k-1} & d_{i,k,k-1} + d_{k,j,k-1} \geq d_{i,j,k-1} 
    \end{cases}
\]

Come prima la soluzione $\pi_{i,j} = \pi_{i,j,n}$.

\section{Procedura FW}

\begin{algorithm}
    \begin{algorithmic}
        \Procedure{FLOYD-WARSHALL}{$V, E, W$}
        \State $D^0 \gets W$
        \For{$i = 1$ to n}
            \For{$j = 1$ to n}
                \If{$i \ne j \land W[i,j] \ne \infty$}
                    \State $\Pi^0[i,j] \gets i$
                \Else
                    \State $\Pi^0[i,j] \gets NIL$
                \EndIf
            \EndFor
        \EndFor
        \For{$k = 1$ to n}
            \For{$i = 1$ to n}
                \For{$j = 1$ to n}
                    \State $a \gets D^{k-1}[i,k] + D^{k-1}[k,j]$
                    \State $b \gets D^{k-1}[i,j]$
                    \If{$a < b$}
                        \State $D^k[i,j] \gets a$
                        \State $\Pi^k[i,j] \gets \Pi^{k-1}[k,j]$
                    \Else
                        \State $D^k[i,j] \gets b$
                        \State $\Pi^k[i,j] \gets \Pi^{k-1}[i,j]$
                    \EndIf
                \EndFor
            \EndFor
        \EndFor
        \EndProcedure
    \end{algorithmic}
\end{algorithm}

\begin{algorithm}
    \begin{algorithmic}
        \Procedure{STAMPA-CAMMINO}{$\Pi, i, j$}
            \If{$\Pi[i,j] \ne NIL$}
                \State STAMPA-CAMMINO($\Pi, \Pi[i,j], j$)
                \State PRINT(j)
            \EndIf
        \EndProcedure
    \end{algorithmic}
\end{algorithm}

\paragraph{Tempi di calcolo}

Il riempimento delle matrici $D, K$ e' $T(n) = \Theta(n^3)$.


