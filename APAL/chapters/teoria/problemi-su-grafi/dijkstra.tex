\chapter{Dijkstra}

Questo algoritmo trova tutti i percorsi minimi tra un nodo $v_x \in V$ e tutti gli altri nodi $v_y \in V \setminus x$.

Si puo' modificare questo algoritmo per cercare un certo percorso tra due nodi definiti in partenza $v_x \in V$ e $v_y \in V \setminus x$, semplicemente fermandosi quando il percorso minimo viene trovato dall'algoritmo.

\section{Algoritmo}

Si marcano tutti i nodi come non-visitati. Quindi si impostano le distanze minime a $\infty$, tranne $d(x, x) = 0$.

Quindi si preleva ad ogni turno il nodo non-visitato $c$ con distanza minima piu' bassa e si notano le lunghezze totali dei percorsi verso i nodi $v$ adiacenti a $c$, ($x, v$), vincolato a passare per $c$. Se questo percorso e' migliore di quello fin'ora trovato per ($x, v$), si aggiorna questa informazione e si segna il passaggio per $c$.

Quando tutti gli archi verso nodi adiacenti sono stati processati, si segna come visitato il nodo corrente $c$ e si preleva il prossimo.

Quando $y$ e' stato visitato o non rimangono nodi da visitare, l'algoritmo termina.

\newpage

\section{Procedura}

Si mantiene un array D delle minime distanze da $x$ al generico nodo $v \in V \setminus x$.

Si mantiene un array H dove ogni cella rappresenta il predecessore del generico nodo $v \in V \setminus x$ nel percorso minimo tra $x$ e $v$.

Si mantiene una priority queue dei nodi non visitati ordinati in base alla minima distanza in D.

Utilizzo le Liste di Adiacenza $L_n$ del generico nodo $n \in V$.

\begin{algorithm}
    \begin{algorithmic}
        \Procedure{DIJKSTRA}{$V, L, x, y$}
            \For{$i = 1$ to $n$}
                \State $D[v_i] \gets \infty$
                \State $H[v_i] \gets NIL$
            \EndFor
            \State $D[v_x] \gets 0$
            \State $Q \gets PriorityQueue(V)$
            \State $c \gets x$
            \While{$v_y \in Q$}
                \For{$(v, d) \in L_{c}$}
                    \State $length \gets D[v_c] + d$
                    \If{$length < D[v]$}
                        \State $D[v] \gets length$
                        \State $H[v] \gets c$
                    \EndIf
                \EndFor
                \State $c \gets dequeue(Q)$
            \EndWhile
            \State \Return $D[v_y]$
        \EndProcedure
    \end{algorithmic}
\end{algorithm}

\begin{algorithm}
    \begin{algorithmic}
        \Procedure{STAMPA-RICORSIVA}{$H, c$}
            \If{$H[v_c] \ne NIL$}
                STAMPA-RICORSIVA($H, H[v_c]$)
            \EndIf
            print($v_c$)
        \EndProcedure
    \end{algorithmic}
\end{algorithm}

\paragraph{Complessita'}

Definisco $h$ come la media del numero di nodi adiacenti per un generico nodo $v$.
Definisco $n = |V|$.

La complessita' di DIJKSTRA e' $O(n \times h)$, la complessita' di STAMPA-RICORSIVA e' O(n).
