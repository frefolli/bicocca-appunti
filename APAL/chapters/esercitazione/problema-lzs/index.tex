\chapter{Problema LZS}

\section{Introduzione}

Il problema Longest Zig-Zag Subsequence consiste nel trovare una sottosequenza piu' lunga tale che

\[
    \forall i \in [0, |LZS(X)|) \mid P :
    \begin{cases}
        \text{$z_i < z_{i+1}$} & \text{$dispari(i)$} \\
        \text{$z_i > z_{i+1}$} & \text{$pari(i)$}
    \end{cases}
\]

\section{Ragionamento}

Ragiono per prefissi.
Sia $X_{i}$ il prefisso i-esimo della sequenza di input.
Sia $Z_k = LZS(X)$ e quindi $S_i = LZS(X_{i})$. \\

Se $ 0\leq i < 2$ allora $Z_k = X_{i}$ e $S_i = X_i = Z_k$. \\
Questo perche' banalmente LZS(<>) = <> e LZS(<$\alpha$>) = <$\alpha$>. \\

Mi rifaccio ai due sotto problemi $LZS_{p:i}$, $LZS_{d:i}$ che risolvono rispettivamente
la LZS pari vincolata a terminare con $x_i$ e
la LZS dispari vincolata a terminare con $x_i$. \\

Assegno $S_{i} = max \{ LZS_{p:i}, LZS_{d:i} \}$.
Trovare $LZS_{p:i}$ e $LZS_{d:i}$ a questo punto coincide con il cercare il $S_{i}$ compatibile piu' lungo. \\
Quindi, $Z_k = max_{length} \{ S_{i} \mid \forall i \in [1,n] \}$.

\paragraph{Caso Base}

\[
    \begin{cases}
        \text{$<>$} & \text{$i = 0$} \\
        \text{$<x_1>$} & \text{$i = 1$} \\
    \end{cases}
\]

\paragraph{Caso Ricorsivo}

\[
    \begin{cases}
        \text{$LZS_{p:i}$} & \text{$i > 1 \land z_{k} < x_i \land |LZS_{p:i}| > |LZS_{d:i}|$} \\
        \text{$LZS_{d:i}$} & \text{$i > 1 \land z_{k} > x_i \land |LZS_{p:i}| \leq |LZS_{d:i}|$} \\
    \end{cases}
\]

\newpage

\section{Procedura}

\begin{algorithm}
    \begin{algorithmic}
        \Procedure{LZS}{$X$}
            \State $C[1] \gets 1$
            \State $H[1] \gets 0$
            \State $lenLZS \gets 1$
            \For{$i = 2$ to $n$}
                \State $max_p, max_d, h_p, h_d \gets 0$
                \For{$h = 1$ to $i-1$}
                    \If{$pari(C[h]) \land C[h] > max_p \land x_h > x_i$}
                        \State $max_p \gets C[h]$
                        \State $h_p \gets h$
                    \ElsIf{$dispari(C[h]) \land C[h] > max_d \land x_h < x_i$}
                        \State $max_d \gets C[h]$
                        \State $h_d \gets h$
                    \EndIf
                \EndFor
                \If{$max_p > max_d$}
                    \State $C[i] \gets max_p + 1$
                    \State $H[i] \gets h_p$
                \Else
                    \State $C[i] \gets max_d + 1$
                    \State $H[i] \gets h_d$
                \EndIf
                \If{$C[i] > lenLZS$}
                    \State $lenLZS \gets C[i]$
                \EndIf
            \EndFor
            \State \Return $lenLZS$
        \EndProcedure
    \end{algorithmic}
\end{algorithm}

\begin{algorithm}
    \begin{algorithmic}
        \Procedure{RICOSTRUISCI}{$X$, $C$, $H$}
            \State $i \gets $trova-indice$(C, lenLZS)$
            \While{$i > 0$}
                \State $print(x_{i})$
                \State $i \gets H[i]$
            \EndWhile
        \EndProcedure
    \end{algorithmic}
\end{algorithm}

\paragraph{Complessita'}
La ricerca del massimo sottoproblema richiede $T_s(n) = \O(n)$.
Quindi l'algoritmo ha complessita' totale $T(n) = \O(n^2)$.