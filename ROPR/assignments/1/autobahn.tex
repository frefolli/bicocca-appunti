
\begin{center}
	\begin{tabular}{|c|c|c|c|c|c|c|}
		\hline
		base & Z & $x_1$ & $x_2$ & $x_3$ & $x_4$ & b\\
		\hline
		Z & 1 & -1 & -1 & 0 & 0 & 0\\
		3 & 0 & 1 & 1 & 1 & 0 & 2\\
		4 & 0 & 2 & -1 & 0 & 1 & 0\\
		\hline
	\end{tabular}
\end{center}

\paragraph{iterazione 1}

Scelgo la colonna 1 perche' non esiste un coefficiente in prima riga negativo piu' basso. \\

Scelgo la riga 2 che ha il rapporto minimo. \\

Ricalcolo la tabella.

Questo ha l'effetto di scambiare $x_4$ della base con $x_1$. \

\begin{center}
	\begin{tabular}{|c|c|c|c|c|c|c|}
		\hline
		base & Z & $x_1$ & $x_2$ & $x_3$ & $x_4$ & b\\
		\hline
		Z & 1.0 & 0.0 & -$\frac 3 2$ & 0.0 & $\frac 1 2$ & 0.0\\
		3 & 0.0 & 0.0 & $\frac 3 2$ & 1.0 & -$\frac 1 2$ & 2.0\\
		1 & 0.0 & 1.0 & -$\frac 1 2$ & 0.0 & $\frac 1 2$ & 0.0\\
		\hline
	\end{tabular}
\end{center}

\paragraph{iterazione 2}

Scelgo la colonna 2 perche' non esiste un coefficiente in prima riga negativo piu' basso. \\

Scelgo la riga 1 che ha il rapporto minimo. \\

Ricalcolo la tabella.

Questo ha l'effetto di scambiare $x_3$ della base con $x_2$. \

\begin{center}
	\begin{tabular}{|c|c|c|c|c|c|c|}
		\hline
		base & Z & $x_1$ & $x_2$ & $x_3$ & $x_4$ & b\\
		\hline
		Z & 1.0 & 0.0 & 0.0 & 1.0 & 0.0 & 2.0\\
		2 & 0.0 & 0.0 & 1.0 & $\frac 2 3$ & -$\frac 1 3$ & $\frac 4 3$\\
		1 & 0.0 & 1.0 & 0.0 & $\frac 1 3$ & $\frac 1 3$ & $\frac 2 3$\\
		\hline
	\end{tabular}
\end{center}

\paragraph{iterazione 3}

La prima riga non contiene piu' valori negativi, l'algoritmo del simplesso si arresta. \\

La soluzione di base corrente e' <$x_1$,$x_2$,$x_3$,$x_4$> = <$\frac 2 3$,$\frac 4 3$,0,0>\*

Quindi una soluzione al problema PL e' <$x_1$,$x_2$> = <$\frac 2 3$,$\frac 4 3$>\*
