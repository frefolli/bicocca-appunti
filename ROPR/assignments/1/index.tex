\section{Esercizio 1}

\begin{align}
    \text{$max \; x_1 + x_2$} \\
    \text{$x_1 + x_2 \leq 2$} \\
    \text{$2 x_1 - x_2 \leq 0$} \\
    \text{$x_1 \geq 0$} \\
    \text{$x_2 \geq 0$}
\end{align}

\section{Esercizio 2}

\begin{align}
    \text{$max \; x_1 + x_2$} \\
    \text{$x_1 + x_2 - x_3 = 2$} \\
    \text{$2 x_1 - x_2 \leq 0$} \\
    \text{$x_1, x_2 \geq 0$} \\
    \text{$x_3 \leq 0$}
\end{align}

\paragraph{Conversione in forma standard}

\subparagraph{1}

La forma standard non prevede vincoli di non positivita', quindi inverto il segno di $x_3$ in tutti i vincoli:

\begin{align}
    \text{$max \; x_1 + x_2$} \\
    \text{$x_1 + x_2 + x_3 = 2$} \\
    \text{$2 x_1 - x_2 \leq 0$} \\
    \text{$x_1, x_2, x_3 \geq 0$}
\end{align}

\subparagraph{2}

I vincoli devono essere esclusivamente in forma $\leq$.
Quindi sostituisco il vincolo $x_1 + x_2 + x_3 = 2$ con l'equivalente in termini di disuguaglianze e inverto il segno di quella con $\geq$.

\begin{align}
    \text{$max \; x_1 + x_2$} \\
    \text{$x_1 + x_2 + x_3 \leq 2$} \\
    \text{$- x_1 - x_2 - x_3 \leq - 2$} \\
    \text{$2 x_1 - x_2 \leq 0$} \\
    \text{$x_1, x_2, x_3 \geq 0$}
\end{align}

\paragraph{Conversione in forma aumentata}

\subparagraph{1}

Aggiungo tre variabili di slack per portare i tre vincoli $\leq$ in vincoli $=$.

\begin{align}
    \text{$max \; x_1 + x_2$} \\
    \text{$x_1 + x_2 + x_3 + x_4 = 2$} \\
    \text{$- x_1 - x_2 - x_3 + x_5 = - 2$} \\
    \text{$2 x_1 - x_2 + x_6 = 0$} \\
    \text{$x_1, x_2, x_3 \geq 0$} \\
    \text{$x_4, x_5, x_6 \geq 0$}
\end{align}

\subparagraph{2}

Quindi esporto la funzione obiettivo $f(x)$ in un vincolo $Z - f(x) = 0$.

\begin{align}
    \text{$max \; Z$} \\
    \text{$Z - x_1 - x_2 = 0$} \\
    \text{$x_1 + x_2 + x_3 + x_4 = 2$} \\
    \text{$- x_1 - x_2 - x_3 + x_5 = - 2$} \\
    \text{$2 x_1 - x_2 + x_6 = 0$} \\
    \text{$x_1, x_2, x_3 \geq 0$} \\
    \text{$x_4, x_5, x_6 \geq 0$}
\end{align}

\paragraph{Risoluzione con tableau}

\subparagraph{Iterazione 0}

\begin{center}
    \begin{tabular}{|c|c|c|c|c|c|c|c|c|c|}
        base & riga & Z & $x_1$ & $x_2$ & $x_3$ & $x_4$ & $x_5$ & $x_6$ & termine noto \\
        Z & 0 & 1 & -1 & -1 &  0 &  0 &  0 &  0 &  0 \\
        $x_4$ & 1 & 0 &  1 &  1 &  1 &  1 &  0 &  0 &  2 \\
        $x_5$ & 2 & 0 & -1 & -1 & -1 &  0 &  1 &  0 & -2 \\
        $x_6$ & 3 & 0 &  2 & -1 &  0 &  0 &  0 &  1 &  0
    \end{tabular}
\end{center}

\subparagraph{Iterazione 1}

Sono presenti nella prima riga coefficienti negativi. Seleziono arbitrariamente $x_1$, perche' tutti i coefficienti negativi hanno pari valore. \\

Nella prima colonna considero i coefficienti delle righe 1, 3. \\*
Seleziono il minimo rapporto, ovvero 0 della riga 3. \\

Questo ha l'effetto di togliere dalla base $x_1$ e inserire $x_6$.

\begin{center}
    \begin{tabular}{|c|c|c|c|c|c|c|c|c|c|}
        base & riga & Z & $x_1$ & $x_2$ & $x_3$ & $x_4$ & $x_5$ & $x_6$ & termine noto \\
        Z & 0 & 1 &  0 & -$\frac 3 2$ &  0 &  0 &  0 &  $\frac 1 2$ &  0 \\
        $x_4$ & 1 & 0 &  0 &  $\frac 3 2$ &  1 &  1 &  0 & -$\frac 1 2$ &  2 \\
        $x_5$ & 2 & 0 &  0 & -$\frac 3 2$ & -1 &  0 &  1 &  $\frac 1 2$ & -2 \\
        $x_1$ & 3 & 0 &  1 & -$\frac 1 2$ &  0 &  0 &  0 &  $\frac 1 2$ &  0
    \end{tabular}
\end{center}

\paragraph{Iterazione 2}

A questo punto seleziono come variabile entrante $x_2$, l'ultima variabile non di base con coefficiente negativo nella prima riga. \\

Seleziono l'unica riga con coefficiente strettamente positivo, ovvero la riga 1. Quindi $x_4$ esce dalla base. \\

\begin{center}
    \begin{tabular}{|c|c|c|c|c|c|c|c|c|c|}
        base  & riga & Z & $x_1$ & $x_2$        & $x_3$        & $x_4$        & $x_5$ & $x_6$        & termine noto \\
        Z     & 0    & 1 &  0    & 0            &  1           &  1           &  0    & 0            & 2 \\
        $x_2$ & 1    & 0 &  0    & 1            &  $\frac 2 3$ &  $\frac 2 3$ &  0    & -$\frac 1 3$ & $\frac 4 3$ \\
        $x_5$ & 2    & 0 &  0    & 0            &  0           &  1           &  1    & 0            &  0 \\
        $x_1$ & 3    & 0 &  1    & 0            &  $\frac 1 3$ &  $\frac 1 3$ &  0    &  $\frac 1 3$ & $\frac 2 3$
    \end{tabular}
\end{center}

\paragraph{Iterazione 3}

Non sono piu' presenti coefficienti negativi nella riga 0, quindi l'algoritmo si arresta. \\

La soluzione ottimale e': \\*
<$x_1, x_2, x_3, x_4, x_5, x_6$> = <$\frac 2 3, \frac 4 3, 0, 0, 0, 0$> \\

Visto che le variabili decisionali sono $x_1, x_2$, la soluzione al problema PL e: \\*
<$x_1, x_2$> = <$\frac 2 3, \frac 4 3$> \\
