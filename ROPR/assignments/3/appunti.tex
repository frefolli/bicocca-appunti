\documentclass[a4paper,12pt,oneside]{article}
\usepackage{graphicx}
\usepackage{hyperref}
\usepackage[T1]{fontenc}
\usepackage[utf8]{inputenc}
\usepackage{setspace}
\usepackage{amsmath}
\usepackage{amssymb}

\begin{document}

    \thispagestyle{plain}
    \begin{center}
        \normalsize
        \textbf{Assignment 2}
            
        \vspace{0.2cm}
        \normalsize
        12/11/2022
            
        \vspace{0.2cm}
        \textbf{Francesco Refolli 865955}
    \end{center}

    \section{Esercizio 1}

    \paragraph{Consegna}

    Una catena di supermercati deve decidere quali tra 5 dei suoi magazzini aprire per rifornire 10 nuovi punti vendita. Il costo per l’apertura e mantenimento del magazzino i e' di $\theta_i$ euro. In ciascun magazzino i possono essere stoccati $\alpha_i$ Kg di merce. Ciascun supermercato j si aspetta giornalmente di ricevere almeno $\beta_j$ Kg di merce. Il costo di trasporto unitario di trasporto della merce dal magazzino i al punto vendita j è stimato a $y_{ij}$ euro/Kg. Elaborare un modello di Programmazione Lineare Intera che aiuti la catena di supermercati a decidere quali magazzini aprire e a soddisfare le domande dei punti vendita, minimizzando i costi totali.
    Si tenga inoltre in considerazione che la catena:
    \begin{enumerate}
        \item vuole aprire al massimo 4 magazzini
        \item per ragioni logistiche, vuole aprire il magazzino 5 solo se il magazzino 4 e il magazzino 3 non vengono aperti
        \item impone che il magazzino 1 consegni la merce al supermercato 5 (il più distante) solo in pallet da $k$ kg (si supponga che $a_i$ sia multiplo di k)
        \item vuole che almeno uno dei seguenti scenari sia verificato: scenario a) siano aperti almeno 3 magazzini scenario b) sia aperto almeno 1 tra i magazzini 1, 2 e 5.
    \end{enumerate}

    \paragraph{Modello - funzione obiettivo}
    $d_{ij}$ e' la quantita' di prodotto intera mandata dal magazzino $i$ al supermercato $j$.
    \begin{align*}
        min & \Sigma ^ {5} _ {i=1} \Sigma ^ {10} _ {j=1} d_{ij} \cdot y_{ij} & funzione \;\; obiettivo \\
        d_{ij} &\geq 0 \; e \;\; intera & \forall i,j \in [1,5] x [1,10]
    \end{align*}

    \paragraph{Modello - vincoli di coerenza del problema di trasporto}
    \begin{align*}
        \Sigma ^ {10} _ {j} d_{ij} &= a_{i} & \forall i \in [1,5] \\
        \Sigma ^ {5} _ {i} d_{ij} &= b_{j} & \forall j \in [1,10]
    \end{align*}
 
    \paragraph{Modello - condizione 1}
    
    Si vuole aprire al massimo 4 magazzini. Sfrutto la ricetta \textbf{Vincoli di tipo Either-Or}.
    $M_i$ variabili con valore enorme di supporto per le variabili binarie $v_i$ create e necessarie alla condizione 1.
    
    \begin{align*}
        \Sigma ^ {10} _ {j=1} d_{ij} &\leq 0 + M \cdot v_i & \forall i [1,5] \\
        \Sigma ^ {10} _ {j=1} d_{ij} &\geq 1 - M \cdot (1 - v_i) & \forall i [1,5] \\
        \Sigma ^ {5} _ {i=1} v_i &\geq 0 \; e \;\; intera \\
        v_i &\geq 0 \; e \;\; intera & \forall i \in [1,5] \\
        v_i &\leq 1 \; e \;\; intera & \forall i \in [1,5]
    \end{align*}

    \paragraph{Modello - condizione 2}
    Si vuole aprire il magazzino 5 solo se il magazzino 4 e il magazzino 3 non vengono aperti.

    \paragraph{Modello - condizione 3}
    Si impone che il magazzino 1 consegni la merce al supermercato 5 (il più distante) solo in pallet da $k$ kg.
    x e' la quantita' di pallet che il magazzino 1 spedisce al supermercato 5.
    \begin{align*}
        d_{ij} - k \cdot x &= 0 \\
        x &\geq 0 \; e \;\; intera
    \end{align*}

    \paragraph{Modello - condizione 4}

\end{document}
