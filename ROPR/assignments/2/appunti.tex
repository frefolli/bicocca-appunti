\documentclass[a4paper,12pt,oneside]{article}
\usepackage{graphicx}
\usepackage{hyperref}
\usepackage[T1]{fontenc}
\usepackage[utf8]{inputenc}
\usepackage{setspace}
\usepackage{amsmath}
\usepackage{amssymb}

\begin{document}

    \thispagestyle{plain}
    \begin{center}
        \normalsize
        \textbf{Assignment 2}
            
        \vspace{0.2cm}
        \normalsize
        12/11/2022
            
        \vspace{0.2cm}
        \textbf{Francesco Refolli 865955}
    \end{center}

    \paragraph{Il Problema}
    
    \begin{align*}
        Max \; Z = x_1 &- 2 x_2 \\
        x_1 - x_2 & \geq 1 \\
        x_1 - 2 x_2 & \leq 6 \\
        2 x_1 - x_2 & \leq 6 \\
        x_1 & \geq 0 \\
        x_2 & \leq 0
    \end{align*}

    \section{Risolvere il problema primale con l’algoritmo del simplesso}

    \paragraph{Conversione del problema in forma aumentata}
    Per poter risolvere il problema primale usando l'algoritmo del simplesso devo convertire prima il problema in forma aumentata. \\
    Voglio tutti i vincoli di segno al $\geq$, quindi creo la variabile $y = - x_2$, pongo il vincolo $y \geq 0$ e sostituisco $y = -x_2$ in tutti i vincoli e nella funzione obiettivo.

    \begin{align*}
        Max \; Z = x_1 &+ 2 y \\
        x_1 + y & \geq 1 \\
        x_1 + 2 y & \leq 6 \\
        2 x_1 + y & \leq 6 \\
        x_1, y & \geq 0
    \end{align*}

    Quindi aggiungo una variabile surplus al vincolo 1: $x_3$.

    \begin{align*}
        Max \; Z = x_1 &+ 2 y \\
        x_1 + y - x_3 & = 1 \\
        x_1 + 2 y & \leq 6 \\
        2 x_1 + y & \leq 6 \\
        x_1, y, x_3 & \geq 0
    \end{align*}

    Quindi aggiungo due variabile slack ai vincolo 2 e 3: $x_4, x_5$.

    \begin{align*}
        Max \; Z = x_1 &+ 2 y \\
        x_1 + y - x_3 & = 1 \\
        x_1 + 2 y + x_4 & = 6 \\
        2 x_1 + y + x_5 & = 6 \\
        x_1, y, x_3, x_4, x_5 & \geq 0
    \end{align*}

    \paragraph{Forma tabellare}

    \begin{center}
        \begin{tabular}{||c c c c c c c c||}
            \hline
            base & Z & $x_1$ & $y$ & $x_3$ & $x_4$ & $x_5$ & b \\
            \hline
            \hline
            & 1 & -1 & -2 &  0 & 0 & 0 & 0 \\
            \hline
            & 0 &  1 &  1 & -1 & 0 & 0 & 1 \\
            \hline
            & 0 &  1 &  2 &  0 & 1 & 0 & 6 \\
            \hline
            & 0 &  2 &  1 &  0 & 0 & 1 & 6 \\
            \hline
        \end{tabular}
    \end{center}

    La soluzione di partenza (0, 0, -1, 6, 6) non e' ammissibile quindi introduco una variabile artificiale $A \geq 0$.

    \begin{center}
        \begin{tabular}{||c c c c c c c c c||}
            \hline
            base & Z & $x_1$ & $y$ & $x_3$ & $x_4$ & $x_5$ & A & b \\
            \hline
            \hline
            Z     & 1 & -1 & -1 &  1 & 0 & 0 & 0 & -1 \\
            \hline
            A     & 0 &  1 &  1 & -1 & 0 & 0 & 1 &  1 \\
            \hline
            $x_4$ & 0 &  1 &  2 &  0 & 1 & 0 & 0 &  6 \\
            \hline
            $x_5$ & 0 &  2 &  1 &  0 & 0 & 1 & 0 &  6 \\
            \hline
        \end{tabular}
    \end{center}

    In questo modo ottengo la soluzione di base ammissibile (0,0,0,1,6,6).

    \paragraph{Iterazione 0}

    $x_1$ e $y$ hanno entrambe coefficiente -1 in prima riga, quindi scelgo arbitrariamente $x_1$.
    Calcolando i rapporti dei coefficienti in colonna strettamente positivi, la riga 1 con rapporto $\frac 1 1 = 1$, ha rapporto minimo.
    Quindi la variabile di base uscente e' $A$, la variabile non di base entrante e' $x_1$.
    Ricalcolo le righe di conseguenza. Il risultato dell'iterazione e':

    \begin{center}
        \begin{tabular}{||c c c c c c c c c||}
            \hline
            base & Z & $x_1$ & $y$ & $x_3$ & $x_4$ & $x_5$ & A & b \\
            \hline
            \hline
            Z     & 1 &  0 &  0 &  0 & 0 & 0 &  1 &  0 \\
            \hline
            $x_1$ & 0 &  1 &  1 & -1 & 0 & 0 &  1 &  1 \\
            \hline
            $x_4$ & 0 &  0 &  1 &  1 & 1 & 0 & -1 &  5 \\
            \hline
            $x_5$ & 0 &  0 & -1 &  2 & 0 & 1 & -2 &  4 \\
            \hline
        \end{tabular}
    \end{center}

    La variabile A e' uscita dalla base, non ci sono piu' variabili artificiali nella base, quindi trasformo il tableau per risolvere il problema originale.
    Sostituisco la riga 0 con la funzione obiettivo originale. Quindi annullo i coefficienti in riga 0 corrispondenti alle variabili in base sottraendo una combinazione lineare delle righe 1,2,3.
    Ottengo:

    \begin{center}
        \begin{tabular}{||c c c c c c c c||}
            \hline
            base & Z & $x_1$ & $y$ & $x_3$ & $x_4$ & $x_5$ & b \\
            \hline
            \hline
            Z     & 1 &  0 & -1 & -1 & 0 & 0 &  1 \\
            \hline
            $x_1$ & 0 &  1 &  1 & -1 & 0 & 0 &  1 \\
            \hline
            $x_4$ & 0 &  0 &  1 &  1 & 1 & 0 &  5 \\
            \hline
            $x_5$ & 0 &  0 & -1 &  2 & 0 & 1 &  4 \\
            \hline
        \end{tabular}
    \end{center}

    \paragraph{Iterazione 1}

    $x_3$ e $y$ hanno entrambe coefficiente -1 in prima riga, quindi scelgo arbitrariamente $y$.
    Calcolando i rapporti dei coefficienti in colonna strettamente positivi, la riga 1 con rapporto $\frac 1 1 = 1$, ha rapporto minimo.
    Quindi la variabile di base uscente e' $x_1$, la variabile non di base entrante e' $y$.
    Ricalcolo le righe di conseguenza. Il risultato dell'iterazione e':

    \begin{center}
        \begin{tabular}{||c c c c c c c c||}
            \hline
            base & Z & $x_1$ & $y$ & $x_3$ & $x_4$ & $x_5$ & b \\
            \hline
            \hline
            Z     & 1 &  1 &  0 & -2 & 0 & 0 &  2 \\
            \hline
            $y$   & 0 &  1 &  1 & -1 & 0 & 0 &  1 \\
            \hline
            $x_4$ & 0 & -1 &  0 &  2 & 1 & 0 &  4 \\
            \hline
            $x_5$ & 0 &  1 &  0 &  1 & 0 & 1 &  5 \\
            \hline
        \end{tabular}
    \end{center}

    \paragraph{Iterazione 2}
    
    $x_3$ ha coefficiente -2 in prima riga, quindi lo scelgo come variabile entrante.
    Calcolando i rapporti dei coefficienti in colonna strettamente positivi, la riga 2 con rapporto $\frac 4 2 = 2$, ha rapporto minimo.
    Quindi la variabile di base uscente e' $x_4$, la variabile non di base entrante e' $x_3$.
    Ricalcolo le righe di conseguenza. Il risultato dell'iterazione e':

    \begin{center}
        \begin{tabular}{||c c c c c c c c||}
            \hline
            base & Z & $x_1$ & $y$ & $x_3$ & $x_4$ & $x_5$ & b \\
            \hline
            \hline
            Z     & 1 &            0 &  0 &  0 &            1 & 0 &  6 \\
            \hline
            $y$   & 0 &  $\frac 1 2$ &  1 &  0 &  $\frac 1 2$ & 0 &  3 \\
            \hline
            $x_3$ & 0 & -$\frac 1 2$ &  0 &  1 &  $\frac 1 2$ & 0 &  2 \\
            \hline
            $x_5$ & 0 &  $\frac 3 2$ &  0 &  0 & -$\frac 1 2$ & 1 &  3 \\
            \hline
        \end{tabular}
    \end{center}

    \paragraph{iterazione 3}
    
    La prima riga non contiene piu' valori negativi, l'algoritmo del simplesso si arresta. \\
    La soluzione di base corrente e' ($x_1$,$y$,$x_3$,$x_4$,$x_5$) = (0,3,2,0,3) \\
    Quindi una soluzione al problema PL e' ($x_1$,$y$) = (0,3). \\
    Tuttavia ricordando $y = -x_2$, e' piu' significativo dire ($x_1$, $x_2$) = (0, -3).
    
    Ad ogni modo, siccome una delle variabili non di base ha coefficiente 0 in prima posizione e' possibile continuare con le iterazioni per individuare una ulteriore soluzione ottimale al problema PL. \\

    Seleziono quindi $x_1$ che ha coefficiente 0.
    Calcolando i rapporti dei coefficienti in colonna strettamente positivi, la riga 3 con rapporto $\frac 3 {\frac 3 2} = 2$, ha rapporto minimo.
    Quindi la variabile di base uscente e' $x_5$, la variabile non di base entrante e' $x_1$.
    Ricalcolo le righe di conseguenza. Il risultato dell'iterazione e':

    \begin{center}
        \begin{tabular}{||c c c c c c c c||}
            \hline
            base & Z & $x_1$ & $y$ & $x_3$ & $x_4$ & $x_5$ & b \\
            \hline
            \hline
            Z     & 1 & 0 & 0 & 0 &            1 &            0 &  6 \\
            \hline
            $y$   & 0 & 0 & 1 & 0 &  $\frac 2 3$ & -$\frac 1 3$ &  2 \\
            \hline
            $x_3$ & 0 & 0 & 0 & 1 &  $\frac 1 3$ &  $\frac 1 3$ &  3 \\
            \hline
            $x_1$ & 0 & 1 & 0 & 0 & -$\frac 1 3$ &  $\frac 2 3$ &  2 \\
            \hline
        \end{tabular}
    \end{center}

    \paragraph{iterazione 4}

    La prima riga non contiene valori negativi, l'algoritmo del simplesso si arresta. \\
    La soluzione di base corrente e' ($x_1$,$y$,$x_3$,$x_4$,$x_5$) = (2,2,3,0,0) \\
    Quindi un'altra soluzione al problema PL e' ($x_1$,$y$) = (2,2)\\
    Tuttavia ricordando $y = -x_2$, e' piu' significativo dire ($x_1$, $x_2$) = (2, -2). \\

    Essendoci due soluzioni ottimali di base, tutte le soluzioni ottimali sono una combinazione convessa delle due precedentemente ottenute: \\
    ($x_1, x_2, x_3, x_4, x_5$) = $w_1 \cdot $ (0,-3,2,0,3) $+$ $w_2 \cdot $ (2,-2,3,0,0)

    \newpage

    \section{Risolvere il problema primale graficamente}

    Costruisco il grafico con le equazioni dei vincoli lungo l'asse $x_1 \times x_2$.

    \begin{center}
        \includegraphics[width=12cm]{prima-fase.png}
    \end{center}

    Quindi disegno il gradiente della funzione obiettivo $g$ e la funzione obiettivo.

    \begin{center}
        \includegraphics[width=12cm]{seconda-fase.png}
    \end{center}

    Il vettore del gradiente della funzione obiettivo $g$ = <$1, -2$> , composto dalle derivate parziali delle componenti della funzione obiettivo, e' perpedincolare al vincolo $x_1 - 2x_2 \leq 6$.
    Il problema ha \textbf{Infinite Soluzioni Ottime}.
    Le soluzioni sono tutte le coppie <$x_1, x_2$> che risiedono nello spigolo della regione obiettivo su cui si poggia il vincolo $x_1 - 2x_2 \leq 6$.
    Per calcolare il segmento e' sufficiente calcolare l'intersezione del vincolo $x_1 - 2x_2 \leq 6$ con i vincoli $2x_1 - x_2 \leq 6$ e $x_1 \geq 0$.
    Il risultato sono i punti: ($2, -2$),  ($0, -3$). Le soluzioni sono tutti i punti compresi nel segmento delimitato da essi.

    \section{Costriuire il problema duale}

    Parto dal problema originale:
    
    \begin{align*}
        Max \; Z = x_1 &- 2 x_2 \\
        x_1 - x_2 & \geq 1 \\
        x_1 - 2 x_2 & \leq 6 \\
        2 x_1 - x_2 & \leq 6 \\
        x_1 & \geq 0 \\
        x_2 & \leq 0
    \end{align*}

    Lo porto in forma standard invertendo il primo vincolo e creando la variabile $y = -x_2$
    
    \begin{align*}
        Max \; Z = x_1 &+ 2 y \\
        -x_1 - y & \leq -1 \\
        x_1 + 2 y & \leq 6 \\
        2 x_1 + y & \leq 6 \\
        x_1, y & \geq 0
    \end{align*}

    Si costruisce il problema duale.
    I coefficienti della funzione obiettivo del primale diventano i termini noti dei vincoli del duale.
    I termini noti dei vincoli del primale diventano i coefficienti della funzione obiettivo del duale.
    L'operazione di ottimizzazione diventa $Min$.
    
    \begin{align*}
        Min \; W = - a + 6 b &+ 6 c \\
        -a + b + 2 c \geq 1 \\
        -a + 2 b + c \geq 2 \\
        a, b, c & \geq 0
    \end{align*}

    Quello qua sopra e' il duale del problema primale.
    
    \newpage
    \section{Calcolare la soluzione ottima del duale utilizzando la teoria della dualita'.}
    
    Si costruiscono le condizioni di complementarieta' dei problemi Primale e Duale.
    Considero il problema in forma standard usato per la costruzione del duale, ovvero:
    
    \begin{align*}
        Max \; Z = x_1 &+ 2 y \\
        -x_1 - y & \leq -1 \\
        x_1 + 2 y & \leq 6 \\
        2 x_1 + y & \leq 6 \\
        x_1, y & \geq 0
    \end{align*}

    Per il Primale sono le seguenti:

    \begin{align*}
        a (1 - x_1 - y) & = 0 \\
        b (6 - x_1 - 2 y) & = 0 \\
        c (6 - 2 x_1 - y) & = 0
    \end{align*}

    Per il Duale:

    \begin{align*}
        x_1 (1 + a - b - 2 c) &= 0 \\
        y (2 + a - 2 b - c) &= 0
    \end{align*}

    Quindi abbiamo il seguente sistema di equazioni:

    \begin{align*}
        a (1 - x_1 - y) & = 0 \\
        b (6 - x_1 - 2 y) & = 0 \\
        c (6 - 2 x_1 - y) & = 0 \\
        x_1 (1 + a - b - 2 c) &= 0 \\
        y (2 + a - 2 b - c) &= 0
    \end{align*}

    Voglio ottenere le soluzioni del problema Duale conoscendo una soluzione ottimale del Primale, per esempio $(x_1, y) = (0, 3)$, quindi sostituisco le variabili del Primale con le relative soluzioni:

    \begin{align*}
        a (1 - 0 - 3) & = 0 \\
        b (6 - 0 - 2*3) & = 0 \\
        c (6 - 2*0 - 3) & = 0 \\
        0 (1 + a - b - 2 c) &= 0 \\
        3 (2 + a - 2 b - c) &= 0
    \end{align*}

    Quindi:

    \begin{align*}
        a * -2 & = 0 \\
        b *  0 & = 0 \\
        c *  3 & = 0 \\
        0 (1 + a - b - 2 c) &= 0 \\
        3 (1 + a - b - 2 c) &= 0
    \end{align*}

    Prelevo le equazioni che non siano nella forma $0 * f = 0$:

    \begin{align*}
        a * -2 & = 0 \\
        c *  3 & = 0 \\
        6 + 3a - 6 b - 3 c &= 0
    \end{align*}

    Da cui ricavo:

    \begin{align*}
        a & = 0 \\
        c & = 0 \\
        6 + 3a - 6 b - 3 c &= 0
    \end{align*}

    E quindi sostituendo le prime due nella terza:

    \begin{align*}
        a & = 0 \\
        c & = 0 \\
        6 b &= 6  \\
    \end{align*}

    Quindi una soluzione del duale e' $(0,1,0)$. \\ 
    Visto che il problema primale ha soluzione ed e' limitato, secondo il Teorema di Dualita' sia la proprieta' debole della dualita' che la proprietà forte della dualita' sono applicabili.
    Pertanto la soluzione del problema duale e' ricavabile dai coefficienti delle variabili slack in riga 0 del tableau risolto per il problema primale.
    In questo caso l'algoritmo del simplesso nell'Esercizio 1 aveva trovato due righe 0 ottimali:

    \begin{center}
        \begin{tabular}{||c c c c c c c c||}
            \hline
            base & Z & $x_1$ & $y$ & $x_3$ & $x_4$ & $x_5$ & b \\
            \hline
            \hline
            Z     & 1 & 0 & 0 & 0 &            1 &            0 &  6 \\
            \hline
        \end{tabular}
    \end{center}
    e
    \begin{center}
        \begin{tabular}{||c c c c c c c c||}
            \hline
            base & Z & $x_1$ & $y$ & $x_3$ & $x_4$ & $x_5$ & b \\
            \hline
            \hline
            Z     & 1 & 0 & 0 & 0 &            1 &            0 &  6 \\
            \hline
        \end{tabular}
    \end{center}

    In entrambe queste righe le variabili slack hanno coefficienti (0, 1, 0), che corrispondono con la soluzione trovata con le condizioni di complementarieta'.

\end{document}
