\documentclass[a4paper,12pt,oneside]{book}
\usepackage{graphicx}
\usepackage{hyperref}
\usepackage[T1]{fontenc}
\usepackage[utf8]{inputenc}
\usepackage{setspace}
\usepackage[paper=a4paper,margin=1in]{geometry}
\usepackage{imakeidx}
\usepackage{amsmath}
\usepackage{amssymb}
\usepackage{algorithm}
\usepackage{algpseudocode}
\newcommand{\N}{\mathbb{N}}
\newcommand{\Z}{\mathbb{Z}}
\newcommand{\R}{\mathbb{R}}

\usepackage{mathtools}
\DeclarePairedDelimiter\ceil{\lceil}{\rceil}
\DeclarePairedDelimiter\floor{\lfloor}{\rfloor}

\makeindex

\begin{document}
    
    \begin{titlepage}
        
	\vspace{40mm}
        
	\begin{center}
            {\LARGE{
                    \setstretch{1.2}
                    \textbf{Appunti di \\ Ricerca Operativa e Pianificazione delle Risorse}
                    \par
            }}
        \end{center}
        
        \vspace{50mm}

        \begin{flushright}
            {\large \textbf{A cura di:}} \\
            \large{Francesco Refolli} \\
            \large{Matricola 865955} 
        \end{flushright}
        
        \vspace{40mm}
        \begin{center}
            {\large{\bf Anno Accademico 2022-2023}}
        \end{center}

        \restoregeometry
        
    \end{titlepage}
    
    \printindex
    \chapter{Introduzione alla Programmazione Lineare}

\paragraph{Esempio}

La Wyndor Glass CO produce vetri di elevata qualita', incluso finestre e porte.

\begin{itemize}
    \item Impianto 1: produce cornici in alluminio e le altre componenti
    \item Impiano 2: produce cornici di legno
    \item Impianto 3: produce i vetri e assembla i prodotti
\end{itemize}

Si vuole produrre:
\begin{itemize}
    \item Prodotto 1 - una porta di vetro (impianti 1, 3)
    \item Prodotto 2 - finestra con doppia apertura (impianti 2, 3)
\end{itemize}

La domanda e' virtualmente infinita.
I prodotti sono raggruppati in lotti da 20 unita'.
I tassi sono $Lotti / Settimana$.
Determinare i tassi di produzione per massimizzare la produzione e il profitto finale.

\subparagraph{Dati}

\begin{center}
    \begin{tabular}{||c c c c||}
        \hline
        Impianto & \text{Prodotto 1} & \text{Prodotto 2} & \text{Tempo Produttivo} \\
        \hline
        Impianto 1 & 1 & 0 & 4 \\
        \hline
        Impianto 2 & 0 & 2 & 12 \\
        \hline
        Impianto 3 & 3 & 2 & 18 \\
        \hline
        Profitto & 3000 & 5000 & \\
        \hline
    \end{tabular}
\end{center}

\begin{align}
    \text{$max Z = 3 \cdot x_1 + 5 \cdot x_2$}
    \text{$x_1 \leq 4$}
    \text{$2 \cdot x_2 \leq 12$}
    \text{$3 \cdot x_1 + 2 \cdot x_2 \leq 18$}
    \text{$x_1 \geq 0$}
    \text{$x_2 \geq 0$}
\end{align}

\paragraph{Considerazione}

TODO: Piano di Isolivello!
TODO: Curve di Isolivello!

$Z = valore di misura di prestazione$
$x_j = livello di attivita' j$
$c_j = incremento del valore della misura di prestazione corrispondente all'incremento di attivita' x_j$
$b_i = $
$a_{ij} = $

TODO: completa

In Programmazione Lineare la regione ammissibile e' un Poliedro Convesso in $R^n$
Se la regione e' limitata si dice Politopo.

TODO: Algoritmo del Simplesso

$opt Z = \Sigma _ {j=1} ^ {n} c_j \cdot x_j$
$vincoli \Rightarrow \Sigma _ {j=1} ^ {n} a_{ij} \cdot x_j \leq b_i$

$c_j \rightarrow coefficiente di costo$
$a_{ij} \rightarrow termine noto sinistro$
$b_{i} \rightarrow termine noto destro$

\paragraph{Assunzioni}
Le assunzioni di un problema PL.

\subparagraph{Proporzionalita'}
Il contributo di ogni attivita' al valore della funzione obiettivo e del vincolo e' proporzionale al livello di attivita'.

\subparagraph{Additivita'}
Il valore della funzione obiettivo e dei vincoli e' dato dalla somma dei contributi individuali delle rispettive attivita'.

\subparagraph{Divisibilita'}
Le variabili decisionali possono assumere qualsiasi valore all'interno della Regione di Ammissibilita'

\subparagraph{Certezza'}
I parametri di un problema di PL devono essere noti con certezza.

\subparagraph{Considerazioni Pratiche}

Alcuni vincoli possono includerne altri, in quel caso e' inutile conservarli entrambi, si puo' lasciare quello piu' "forte".
In programmazione lineare le soluzioni possono non essere intere.
TODO: Copia Slide


    \part{Teoria}
    \chapter{Introduzione alla Programmazione Lineare}

\paragraph{Esempio}

La Wyndor Glass CO produce vetri di elevata qualita', incluso finestre e porte.

\begin{itemize}
    \item Impianto 1: produce cornici in alluminio e le altre componenti
    \item Impiano 2: produce cornici di legno
    \item Impianto 3: produce i vetri e assembla i prodotti
\end{itemize}

Si vuole produrre:
\begin{itemize}
    \item Prodotto 1 - una porta di vetro (impianti 1, 3)
    \item Prodotto 2 - finestra con doppia apertura (impianti 2, 3)
\end{itemize}

La domanda e' virtualmente infinita.
I prodotti sono raggruppati in lotti da 20 unita'.
I tassi sono $Lotti / Settimana$.
Determinare i tassi di produzione per massimizzare la produzione e il profitto finale.

\subparagraph{Dati}

\begin{center}
    \begin{tabular}{||c c c c||}
        \hline
        Impianto & \text{Prodotto 1} & \text{Prodotto 2} & \text{Tempo Produttivo} \\
        \hline
        Impianto 1 & 1 & 0 & 4 \\
        \hline
        Impianto 2 & 0 & 2 & 12 \\
        \hline
        Impianto 3 & 3 & 2 & 18 \\
        \hline
        Profitto & 3000 & 5000 & \\
        \hline
    \end{tabular}
\end{center}

\begin{align}
    \text{$max Z = 3 \cdot x_1 + 5 \cdot x_2$}
    \text{$x_1 \leq 4$}
    \text{$2 \cdot x_2 \leq 12$}
    \text{$3 \cdot x_1 + 2 \cdot x_2 \leq 18$}
    \text{$x_1 \geq 0$}
    \text{$x_2 \geq 0$}
\end{align}

\paragraph{Considerazione}

TODO: Piano di Isolivello!
TODO: Curve di Isolivello!

$Z = valore di misura di prestazione$
$x_j = livello di attivita' j$
$c_j = incremento del valore della misura di prestazione corrispondente all'incremento di attivita' x_j$
$b_i = $
$a_{ij} = $

TODO: completa

In Programmazione Lineare la regione ammissibile e' un Poliedro Convesso in $R^n$
Se la regione e' limitata si dice Politopo.

TODO: Algoritmo del Simplesso

$opt Z = \Sigma _ {j=1} ^ {n} c_j \cdot x_j$
$vincoli \Rightarrow \Sigma _ {j=1} ^ {n} a_{ij} \cdot x_j \leq b_i$

$c_j \rightarrow coefficiente di costo$
$a_{ij} \rightarrow termine noto sinistro$
$b_{i} \rightarrow termine noto destro$

\paragraph{Assunzioni}
Le assunzioni di un problema PL.

\subparagraph{Proporzionalita'}
Il contributo di ogni attivita' al valore della funzione obiettivo e del vincolo e' proporzionale al livello di attivita'.

\subparagraph{Additivita'}
Il valore della funzione obiettivo e dei vincoli e' dato dalla somma dei contributi individuali delle rispettive attivita'.

\subparagraph{Divisibilita'}
Le variabili decisionali possono assumere qualsiasi valore all'interno della Regione di Ammissibilita'

\subparagraph{Certezza'}
I parametri di un problema di PL devono essere noti con certezza.

\subparagraph{Considerazioni Pratiche}

Alcuni vincoli possono includerne altri, in quel caso e' inutile conservarli entrambi, si puo' lasciare quello piu' "forte".
In programmazione lineare le soluzioni possono non essere intere.
TODO: Copia Slide

    \chapter{Introduzione alla Programmazione Lineare}

\paragraph{Esempio}

La Wyndor Glass CO produce vetri di elevata qualita', incluso finestre e porte.

\begin{itemize}
    \item Impianto 1: produce cornici in alluminio e le altre componenti
    \item Impiano 2: produce cornici di legno
    \item Impianto 3: produce i vetri e assembla i prodotti
\end{itemize}

Si vuole produrre:
\begin{itemize}
    \item Prodotto 1 - una porta di vetro (impianti 1, 3)
    \item Prodotto 2 - finestra con doppia apertura (impianti 2, 3)
\end{itemize}

La domanda e' virtualmente infinita.
I prodotti sono raggruppati in lotti da 20 unita'.
I tassi sono $Lotti / Settimana$.
Determinare i tassi di produzione per massimizzare la produzione e il profitto finale.

\subparagraph{Dati}

\begin{center}
    \begin{tabular}{||c c c c||}
        \hline
        Impianto & \text{Prodotto 1} & \text{Prodotto 2} & \text{Tempo Produttivo} \\
        \hline
        Impianto 1 & 1 & 0 & 4 \\
        \hline
        Impianto 2 & 0 & 2 & 12 \\
        \hline
        Impianto 3 & 3 & 2 & 18 \\
        \hline
        Profitto & 3000 & 5000 & \\
        \hline
    \end{tabular}
\end{center}

\begin{align}
    \text{$max Z = 3 \cdot x_1 + 5 \cdot x_2$}
    \text{$x_1 \leq 4$}
    \text{$2 \cdot x_2 \leq 12$}
    \text{$3 \cdot x_1 + 2 \cdot x_2 \leq 18$}
    \text{$x_1 \geq 0$}
    \text{$x_2 \geq 0$}
\end{align}

\paragraph{Considerazione}

TODO: Piano di Isolivello!
TODO: Curve di Isolivello!

$Z = valore di misura di prestazione$
$x_j = livello di attivita' j$
$c_j = incremento del valore della misura di prestazione corrispondente all'incremento di attivita' x_j$
$b_i = $
$a_{ij} = $

TODO: completa

In Programmazione Lineare la regione ammissibile e' un Poliedro Convesso in $R^n$
Se la regione e' limitata si dice Politopo.

TODO: Algoritmo del Simplesso

$opt Z = \Sigma _ {j=1} ^ {n} c_j \cdot x_j$
$vincoli \Rightarrow \Sigma _ {j=1} ^ {n} a_{ij} \cdot x_j \leq b_i$

$c_j \rightarrow coefficiente di costo$
$a_{ij} \rightarrow termine noto sinistro$
$b_{i} \rightarrow termine noto destro$

\paragraph{Assunzioni}
Le assunzioni di un problema PL.

\subparagraph{Proporzionalita'}
Il contributo di ogni attivita' al valore della funzione obiettivo e del vincolo e' proporzionale al livello di attivita'.

\subparagraph{Additivita'}
Il valore della funzione obiettivo e dei vincoli e' dato dalla somma dei contributi individuali delle rispettive attivita'.

\subparagraph{Divisibilita'}
Le variabili decisionali possono assumere qualsiasi valore all'interno della Regione di Ammissibilita'

\subparagraph{Certezza'}
I parametri di un problema di PL devono essere noti con certezza.

\subparagraph{Considerazioni Pratiche}

Alcuni vincoli possono includerne altri, in quel caso e' inutile conservarli entrambi, si puo' lasciare quello piu' "forte".
In programmazione lineare le soluzioni possono non essere intere.
TODO: Copia Slide

    \chapter{Introduzione alla Programmazione Lineare}

\paragraph{Esempio}

La Wyndor Glass CO produce vetri di elevata qualita', incluso finestre e porte.

\begin{itemize}
    \item Impianto 1: produce cornici in alluminio e le altre componenti
    \item Impiano 2: produce cornici di legno
    \item Impianto 3: produce i vetri e assembla i prodotti
\end{itemize}

Si vuole produrre:
\begin{itemize}
    \item Prodotto 1 - una porta di vetro (impianti 1, 3)
    \item Prodotto 2 - finestra con doppia apertura (impianti 2, 3)
\end{itemize}

La domanda e' virtualmente infinita.
I prodotti sono raggruppati in lotti da 20 unita'.
I tassi sono $Lotti / Settimana$.
Determinare i tassi di produzione per massimizzare la produzione e il profitto finale.

\subparagraph{Dati}

\begin{center}
    \begin{tabular}{||c c c c||}
        \hline
        Impianto & \text{Prodotto 1} & \text{Prodotto 2} & \text{Tempo Produttivo} \\
        \hline
        Impianto 1 & 1 & 0 & 4 \\
        \hline
        Impianto 2 & 0 & 2 & 12 \\
        \hline
        Impianto 3 & 3 & 2 & 18 \\
        \hline
        Profitto & 3000 & 5000 & \\
        \hline
    \end{tabular}
\end{center}

\begin{align}
    \text{$max Z = 3 \cdot x_1 + 5 \cdot x_2$}
    \text{$x_1 \leq 4$}
    \text{$2 \cdot x_2 \leq 12$}
    \text{$3 \cdot x_1 + 2 \cdot x_2 \leq 18$}
    \text{$x_1 \geq 0$}
    \text{$x_2 \geq 0$}
\end{align}

\paragraph{Considerazione}

TODO: Piano di Isolivello!
TODO: Curve di Isolivello!

$Z = valore di misura di prestazione$
$x_j = livello di attivita' j$
$c_j = incremento del valore della misura di prestazione corrispondente all'incremento di attivita' x_j$
$b_i = $
$a_{ij} = $

TODO: completa

In Programmazione Lineare la regione ammissibile e' un Poliedro Convesso in $R^n$
Se la regione e' limitata si dice Politopo.

TODO: Algoritmo del Simplesso

$opt Z = \Sigma _ {j=1} ^ {n} c_j \cdot x_j$
$vincoli \Rightarrow \Sigma _ {j=1} ^ {n} a_{ij} \cdot x_j \leq b_i$

$c_j \rightarrow coefficiente di costo$
$a_{ij} \rightarrow termine noto sinistro$
$b_{i} \rightarrow termine noto destro$

\paragraph{Assunzioni}
Le assunzioni di un problema PL.

\subparagraph{Proporzionalita'}
Il contributo di ogni attivita' al valore della funzione obiettivo e del vincolo e' proporzionale al livello di attivita'.

\subparagraph{Additivita'}
Il valore della funzione obiettivo e dei vincoli e' dato dalla somma dei contributi individuali delle rispettive attivita'.

\subparagraph{Divisibilita'}
Le variabili decisionali possono assumere qualsiasi valore all'interno della Regione di Ammissibilita'

\subparagraph{Certezza'}
I parametri di un problema di PL devono essere noti con certezza.

\subparagraph{Considerazioni Pratiche}

Alcuni vincoli possono includerne altri, in quel caso e' inutile conservarli entrambi, si puo' lasciare quello piu' "forte".
In programmazione lineare le soluzioni possono non essere intere.
TODO: Copia Slide

    \chapter{Introduzione alla Programmazione Lineare}

\paragraph{Esempio}

La Wyndor Glass CO produce vetri di elevata qualita', incluso finestre e porte.

\begin{itemize}
    \item Impianto 1: produce cornici in alluminio e le altre componenti
    \item Impiano 2: produce cornici di legno
    \item Impianto 3: produce i vetri e assembla i prodotti
\end{itemize}

Si vuole produrre:
\begin{itemize}
    \item Prodotto 1 - una porta di vetro (impianti 1, 3)
    \item Prodotto 2 - finestra con doppia apertura (impianti 2, 3)
\end{itemize}

La domanda e' virtualmente infinita.
I prodotti sono raggruppati in lotti da 20 unita'.
I tassi sono $Lotti / Settimana$.
Determinare i tassi di produzione per massimizzare la produzione e il profitto finale.

\subparagraph{Dati}

\begin{center}
    \begin{tabular}{||c c c c||}
        \hline
        Impianto & \text{Prodotto 1} & \text{Prodotto 2} & \text{Tempo Produttivo} \\
        \hline
        Impianto 1 & 1 & 0 & 4 \\
        \hline
        Impianto 2 & 0 & 2 & 12 \\
        \hline
        Impianto 3 & 3 & 2 & 18 \\
        \hline
        Profitto & 3000 & 5000 & \\
        \hline
    \end{tabular}
\end{center}

\begin{align}
    \text{$max Z = 3 \cdot x_1 + 5 \cdot x_2$}
    \text{$x_1 \leq 4$}
    \text{$2 \cdot x_2 \leq 12$}
    \text{$3 \cdot x_1 + 2 \cdot x_2 \leq 18$}
    \text{$x_1 \geq 0$}
    \text{$x_2 \geq 0$}
\end{align}

\paragraph{Considerazione}

TODO: Piano di Isolivello!
TODO: Curve di Isolivello!

$Z = valore di misura di prestazione$
$x_j = livello di attivita' j$
$c_j = incremento del valore della misura di prestazione corrispondente all'incremento di attivita' x_j$
$b_i = $
$a_{ij} = $

TODO: completa

In Programmazione Lineare la regione ammissibile e' un Poliedro Convesso in $R^n$
Se la regione e' limitata si dice Politopo.

TODO: Algoritmo del Simplesso

$opt Z = \Sigma _ {j=1} ^ {n} c_j \cdot x_j$
$vincoli \Rightarrow \Sigma _ {j=1} ^ {n} a_{ij} \cdot x_j \leq b_i$

$c_j \rightarrow coefficiente di costo$
$a_{ij} \rightarrow termine noto sinistro$
$b_{i} \rightarrow termine noto destro$

\paragraph{Assunzioni}
Le assunzioni di un problema PL.

\subparagraph{Proporzionalita'}
Il contributo di ogni attivita' al valore della funzione obiettivo e del vincolo e' proporzionale al livello di attivita'.

\subparagraph{Additivita'}
Il valore della funzione obiettivo e dei vincoli e' dato dalla somma dei contributi individuali delle rispettive attivita'.

\subparagraph{Divisibilita'}
Le variabili decisionali possono assumere qualsiasi valore all'interno della Regione di Ammissibilita'

\subparagraph{Certezza'}
I parametri di un problema di PL devono essere noti con certezza.

\subparagraph{Considerazioni Pratiche}

Alcuni vincoli possono includerne altri, in quel caso e' inutile conservarli entrambi, si puo' lasciare quello piu' "forte".
In programmazione lineare le soluzioni possono non essere intere.
TODO: Copia Slide

    \chapter{Introduzione alla Programmazione Lineare}

\paragraph{Esempio}

La Wyndor Glass CO produce vetri di elevata qualita', incluso finestre e porte.

\begin{itemize}
    \item Impianto 1: produce cornici in alluminio e le altre componenti
    \item Impiano 2: produce cornici di legno
    \item Impianto 3: produce i vetri e assembla i prodotti
\end{itemize}

Si vuole produrre:
\begin{itemize}
    \item Prodotto 1 - una porta di vetro (impianti 1, 3)
    \item Prodotto 2 - finestra con doppia apertura (impianti 2, 3)
\end{itemize}

La domanda e' virtualmente infinita.
I prodotti sono raggruppati in lotti da 20 unita'.
I tassi sono $Lotti / Settimana$.
Determinare i tassi di produzione per massimizzare la produzione e il profitto finale.

\subparagraph{Dati}

\begin{center}
    \begin{tabular}{||c c c c||}
        \hline
        Impianto & \text{Prodotto 1} & \text{Prodotto 2} & \text{Tempo Produttivo} \\
        \hline
        Impianto 1 & 1 & 0 & 4 \\
        \hline
        Impianto 2 & 0 & 2 & 12 \\
        \hline
        Impianto 3 & 3 & 2 & 18 \\
        \hline
        Profitto & 3000 & 5000 & \\
        \hline
    \end{tabular}
\end{center}

\begin{align}
    \text{$max Z = 3 \cdot x_1 + 5 \cdot x_2$}
    \text{$x_1 \leq 4$}
    \text{$2 \cdot x_2 \leq 12$}
    \text{$3 \cdot x_1 + 2 \cdot x_2 \leq 18$}
    \text{$x_1 \geq 0$}
    \text{$x_2 \geq 0$}
\end{align}

\paragraph{Considerazione}

TODO: Piano di Isolivello!
TODO: Curve di Isolivello!

$Z = valore di misura di prestazione$
$x_j = livello di attivita' j$
$c_j = incremento del valore della misura di prestazione corrispondente all'incremento di attivita' x_j$
$b_i = $
$a_{ij} = $

TODO: completa

In Programmazione Lineare la regione ammissibile e' un Poliedro Convesso in $R^n$
Se la regione e' limitata si dice Politopo.

TODO: Algoritmo del Simplesso

$opt Z = \Sigma _ {j=1} ^ {n} c_j \cdot x_j$
$vincoli \Rightarrow \Sigma _ {j=1} ^ {n} a_{ij} \cdot x_j \leq b_i$

$c_j \rightarrow coefficiente di costo$
$a_{ij} \rightarrow termine noto sinistro$
$b_{i} \rightarrow termine noto destro$

\paragraph{Assunzioni}
Le assunzioni di un problema PL.

\subparagraph{Proporzionalita'}
Il contributo di ogni attivita' al valore della funzione obiettivo e del vincolo e' proporzionale al livello di attivita'.

\subparagraph{Additivita'}
Il valore della funzione obiettivo e dei vincoli e' dato dalla somma dei contributi individuali delle rispettive attivita'.

\subparagraph{Divisibilita'}
Le variabili decisionali possono assumere qualsiasi valore all'interno della Regione di Ammissibilita'

\subparagraph{Certezza'}
I parametri di un problema di PL devono essere noti con certezza.

\subparagraph{Considerazioni Pratiche}

Alcuni vincoli possono includerne altri, in quel caso e' inutile conservarli entrambi, si puo' lasciare quello piu' "forte".
In programmazione lineare le soluzioni possono non essere intere.
TODO: Copia Slide

    \chapter{Introduzione alla Programmazione Lineare}

\paragraph{Esempio}

La Wyndor Glass CO produce vetri di elevata qualita', incluso finestre e porte.

\begin{itemize}
    \item Impianto 1: produce cornici in alluminio e le altre componenti
    \item Impiano 2: produce cornici di legno
    \item Impianto 3: produce i vetri e assembla i prodotti
\end{itemize}

Si vuole produrre:
\begin{itemize}
    \item Prodotto 1 - una porta di vetro (impianti 1, 3)
    \item Prodotto 2 - finestra con doppia apertura (impianti 2, 3)
\end{itemize}

La domanda e' virtualmente infinita.
I prodotti sono raggruppati in lotti da 20 unita'.
I tassi sono $Lotti / Settimana$.
Determinare i tassi di produzione per massimizzare la produzione e il profitto finale.

\subparagraph{Dati}

\begin{center}
    \begin{tabular}{||c c c c||}
        \hline
        Impianto & \text{Prodotto 1} & \text{Prodotto 2} & \text{Tempo Produttivo} \\
        \hline
        Impianto 1 & 1 & 0 & 4 \\
        \hline
        Impianto 2 & 0 & 2 & 12 \\
        \hline
        Impianto 3 & 3 & 2 & 18 \\
        \hline
        Profitto & 3000 & 5000 & \\
        \hline
    \end{tabular}
\end{center}

\begin{align}
    \text{$max Z = 3 \cdot x_1 + 5 \cdot x_2$}
    \text{$x_1 \leq 4$}
    \text{$2 \cdot x_2 \leq 12$}
    \text{$3 \cdot x_1 + 2 \cdot x_2 \leq 18$}
    \text{$x_1 \geq 0$}
    \text{$x_2 \geq 0$}
\end{align}

\paragraph{Considerazione}

TODO: Piano di Isolivello!
TODO: Curve di Isolivello!

$Z = valore di misura di prestazione$
$x_j = livello di attivita' j$
$c_j = incremento del valore della misura di prestazione corrispondente all'incremento di attivita' x_j$
$b_i = $
$a_{ij} = $

TODO: completa

In Programmazione Lineare la regione ammissibile e' un Poliedro Convesso in $R^n$
Se la regione e' limitata si dice Politopo.

TODO: Algoritmo del Simplesso

$opt Z = \Sigma _ {j=1} ^ {n} c_j \cdot x_j$
$vincoli \Rightarrow \Sigma _ {j=1} ^ {n} a_{ij} \cdot x_j \leq b_i$

$c_j \rightarrow coefficiente di costo$
$a_{ij} \rightarrow termine noto sinistro$
$b_{i} \rightarrow termine noto destro$

\paragraph{Assunzioni}
Le assunzioni di un problema PL.

\subparagraph{Proporzionalita'}
Il contributo di ogni attivita' al valore della funzione obiettivo e del vincolo e' proporzionale al livello di attivita'.

\subparagraph{Additivita'}
Il valore della funzione obiettivo e dei vincoli e' dato dalla somma dei contributi individuali delle rispettive attivita'.

\subparagraph{Divisibilita'}
Le variabili decisionali possono assumere qualsiasi valore all'interno della Regione di Ammissibilita'

\subparagraph{Certezza'}
I parametri di un problema di PL devono essere noti con certezza.

\subparagraph{Considerazioni Pratiche}

Alcuni vincoli possono includerne altri, in quel caso e' inutile conservarli entrambi, si puo' lasciare quello piu' "forte".
In programmazione lineare le soluzioni possono non essere intere.
TODO: Copia Slide

    \chapter{Introduzione alla Programmazione Lineare}

\paragraph{Esempio}

La Wyndor Glass CO produce vetri di elevata qualita', incluso finestre e porte.

\begin{itemize}
    \item Impianto 1: produce cornici in alluminio e le altre componenti
    \item Impiano 2: produce cornici di legno
    \item Impianto 3: produce i vetri e assembla i prodotti
\end{itemize}

Si vuole produrre:
\begin{itemize}
    \item Prodotto 1 - una porta di vetro (impianti 1, 3)
    \item Prodotto 2 - finestra con doppia apertura (impianti 2, 3)
\end{itemize}

La domanda e' virtualmente infinita.
I prodotti sono raggruppati in lotti da 20 unita'.
I tassi sono $Lotti / Settimana$.
Determinare i tassi di produzione per massimizzare la produzione e il profitto finale.

\subparagraph{Dati}

\begin{center}
    \begin{tabular}{||c c c c||}
        \hline
        Impianto & \text{Prodotto 1} & \text{Prodotto 2} & \text{Tempo Produttivo} \\
        \hline
        Impianto 1 & 1 & 0 & 4 \\
        \hline
        Impianto 2 & 0 & 2 & 12 \\
        \hline
        Impianto 3 & 3 & 2 & 18 \\
        \hline
        Profitto & 3000 & 5000 & \\
        \hline
    \end{tabular}
\end{center}

\begin{align}
    \text{$max Z = 3 \cdot x_1 + 5 \cdot x_2$}
    \text{$x_1 \leq 4$}
    \text{$2 \cdot x_2 \leq 12$}
    \text{$3 \cdot x_1 + 2 \cdot x_2 \leq 18$}
    \text{$x_1 \geq 0$}
    \text{$x_2 \geq 0$}
\end{align}

\paragraph{Considerazione}

TODO: Piano di Isolivello!
TODO: Curve di Isolivello!

$Z = valore di misura di prestazione$
$x_j = livello di attivita' j$
$c_j = incremento del valore della misura di prestazione corrispondente all'incremento di attivita' x_j$
$b_i = $
$a_{ij} = $

TODO: completa

In Programmazione Lineare la regione ammissibile e' un Poliedro Convesso in $R^n$
Se la regione e' limitata si dice Politopo.

TODO: Algoritmo del Simplesso

$opt Z = \Sigma _ {j=1} ^ {n} c_j \cdot x_j$
$vincoli \Rightarrow \Sigma _ {j=1} ^ {n} a_{ij} \cdot x_j \leq b_i$

$c_j \rightarrow coefficiente di costo$
$a_{ij} \rightarrow termine noto sinistro$
$b_{i} \rightarrow termine noto destro$

\paragraph{Assunzioni}
Le assunzioni di un problema PL.

\subparagraph{Proporzionalita'}
Il contributo di ogni attivita' al valore della funzione obiettivo e del vincolo e' proporzionale al livello di attivita'.

\subparagraph{Additivita'}
Il valore della funzione obiettivo e dei vincoli e' dato dalla somma dei contributi individuali delle rispettive attivita'.

\subparagraph{Divisibilita'}
Le variabili decisionali possono assumere qualsiasi valore all'interno della Regione di Ammissibilita'

\subparagraph{Certezza'}
I parametri di un problema di PL devono essere noti con certezza.

\subparagraph{Considerazioni Pratiche}

Alcuni vincoli possono includerne altri, in quel caso e' inutile conservarli entrambi, si puo' lasciare quello piu' "forte".
In programmazione lineare le soluzioni possono non essere intere.
TODO: Copia Slide

    \chapter{Introduzione alla Programmazione Lineare}

\paragraph{Esempio}

La Wyndor Glass CO produce vetri di elevata qualita', incluso finestre e porte.

\begin{itemize}
    \item Impianto 1: produce cornici in alluminio e le altre componenti
    \item Impiano 2: produce cornici di legno
    \item Impianto 3: produce i vetri e assembla i prodotti
\end{itemize}

Si vuole produrre:
\begin{itemize}
    \item Prodotto 1 - una porta di vetro (impianti 1, 3)
    \item Prodotto 2 - finestra con doppia apertura (impianti 2, 3)
\end{itemize}

La domanda e' virtualmente infinita.
I prodotti sono raggruppati in lotti da 20 unita'.
I tassi sono $Lotti / Settimana$.
Determinare i tassi di produzione per massimizzare la produzione e il profitto finale.

\subparagraph{Dati}

\begin{center}
    \begin{tabular}{||c c c c||}
        \hline
        Impianto & \text{Prodotto 1} & \text{Prodotto 2} & \text{Tempo Produttivo} \\
        \hline
        Impianto 1 & 1 & 0 & 4 \\
        \hline
        Impianto 2 & 0 & 2 & 12 \\
        \hline
        Impianto 3 & 3 & 2 & 18 \\
        \hline
        Profitto & 3000 & 5000 & \\
        \hline
    \end{tabular}
\end{center}

\begin{align}
    \text{$max Z = 3 \cdot x_1 + 5 \cdot x_2$}
    \text{$x_1 \leq 4$}
    \text{$2 \cdot x_2 \leq 12$}
    \text{$3 \cdot x_1 + 2 \cdot x_2 \leq 18$}
    \text{$x_1 \geq 0$}
    \text{$x_2 \geq 0$}
\end{align}

\paragraph{Considerazione}

TODO: Piano di Isolivello!
TODO: Curve di Isolivello!

$Z = valore di misura di prestazione$
$x_j = livello di attivita' j$
$c_j = incremento del valore della misura di prestazione corrispondente all'incremento di attivita' x_j$
$b_i = $
$a_{ij} = $

TODO: completa

In Programmazione Lineare la regione ammissibile e' un Poliedro Convesso in $R^n$
Se la regione e' limitata si dice Politopo.

TODO: Algoritmo del Simplesso

$opt Z = \Sigma _ {j=1} ^ {n} c_j \cdot x_j$
$vincoli \Rightarrow \Sigma _ {j=1} ^ {n} a_{ij} \cdot x_j \leq b_i$

$c_j \rightarrow coefficiente di costo$
$a_{ij} \rightarrow termine noto sinistro$
$b_{i} \rightarrow termine noto destro$

\paragraph{Assunzioni}
Le assunzioni di un problema PL.

\subparagraph{Proporzionalita'}
Il contributo di ogni attivita' al valore della funzione obiettivo e del vincolo e' proporzionale al livello di attivita'.

\subparagraph{Additivita'}
Il valore della funzione obiettivo e dei vincoli e' dato dalla somma dei contributi individuali delle rispettive attivita'.

\subparagraph{Divisibilita'}
Le variabili decisionali possono assumere qualsiasi valore all'interno della Regione di Ammissibilita'

\subparagraph{Certezza'}
I parametri di un problema di PL devono essere noti con certezza.

\subparagraph{Considerazioni Pratiche}

Alcuni vincoli possono includerne altri, in quel caso e' inutile conservarli entrambi, si puo' lasciare quello piu' "forte".
In programmazione lineare le soluzioni possono non essere intere.
TODO: Copia Slide

    
    \part{Esercitazione}
    \chapter{Introduzione alla Programmazione Lineare}

\paragraph{Esempio}

La Wyndor Glass CO produce vetri di elevata qualita', incluso finestre e porte.

\begin{itemize}
    \item Impianto 1: produce cornici in alluminio e le altre componenti
    \item Impiano 2: produce cornici di legno
    \item Impianto 3: produce i vetri e assembla i prodotti
\end{itemize}

Si vuole produrre:
\begin{itemize}
    \item Prodotto 1 - una porta di vetro (impianti 1, 3)
    \item Prodotto 2 - finestra con doppia apertura (impianti 2, 3)
\end{itemize}

La domanda e' virtualmente infinita.
I prodotti sono raggruppati in lotti da 20 unita'.
I tassi sono $Lotti / Settimana$.
Determinare i tassi di produzione per massimizzare la produzione e il profitto finale.

\subparagraph{Dati}

\begin{center}
    \begin{tabular}{||c c c c||}
        \hline
        Impianto & \text{Prodotto 1} & \text{Prodotto 2} & \text{Tempo Produttivo} \\
        \hline
        Impianto 1 & 1 & 0 & 4 \\
        \hline
        Impianto 2 & 0 & 2 & 12 \\
        \hline
        Impianto 3 & 3 & 2 & 18 \\
        \hline
        Profitto & 3000 & 5000 & \\
        \hline
    \end{tabular}
\end{center}

\begin{align}
    \text{$max Z = 3 \cdot x_1 + 5 \cdot x_2$}
    \text{$x_1 \leq 4$}
    \text{$2 \cdot x_2 \leq 12$}
    \text{$3 \cdot x_1 + 2 \cdot x_2 \leq 18$}
    \text{$x_1 \geq 0$}
    \text{$x_2 \geq 0$}
\end{align}

\paragraph{Considerazione}

TODO: Piano di Isolivello!
TODO: Curve di Isolivello!

$Z = valore di misura di prestazione$
$x_j = livello di attivita' j$
$c_j = incremento del valore della misura di prestazione corrispondente all'incremento di attivita' x_j$
$b_i = $
$a_{ij} = $

TODO: completa

In Programmazione Lineare la regione ammissibile e' un Poliedro Convesso in $R^n$
Se la regione e' limitata si dice Politopo.

TODO: Algoritmo del Simplesso

$opt Z = \Sigma _ {j=1} ^ {n} c_j \cdot x_j$
$vincoli \Rightarrow \Sigma _ {j=1} ^ {n} a_{ij} \cdot x_j \leq b_i$

$c_j \rightarrow coefficiente di costo$
$a_{ij} \rightarrow termine noto sinistro$
$b_{i} \rightarrow termine noto destro$

\paragraph{Assunzioni}
Le assunzioni di un problema PL.

\subparagraph{Proporzionalita'}
Il contributo di ogni attivita' al valore della funzione obiettivo e del vincolo e' proporzionale al livello di attivita'.

\subparagraph{Additivita'}
Il valore della funzione obiettivo e dei vincoli e' dato dalla somma dei contributi individuali delle rispettive attivita'.

\subparagraph{Divisibilita'}
Le variabili decisionali possono assumere qualsiasi valore all'interno della Regione di Ammissibilita'

\subparagraph{Certezza'}
I parametri di un problema di PL devono essere noti con certezza.

\subparagraph{Considerazioni Pratiche}

Alcuni vincoli possono includerne altri, in quel caso e' inutile conservarli entrambi, si puo' lasciare quello piu' "forte".
In programmazione lineare le soluzioni possono non essere intere.
TODO: Copia Slide

    \chapter{Introduzione alla Programmazione Lineare}

\paragraph{Esempio}

La Wyndor Glass CO produce vetri di elevata qualita', incluso finestre e porte.

\begin{itemize}
    \item Impianto 1: produce cornici in alluminio e le altre componenti
    \item Impiano 2: produce cornici di legno
    \item Impianto 3: produce i vetri e assembla i prodotti
\end{itemize}

Si vuole produrre:
\begin{itemize}
    \item Prodotto 1 - una porta di vetro (impianti 1, 3)
    \item Prodotto 2 - finestra con doppia apertura (impianti 2, 3)
\end{itemize}

La domanda e' virtualmente infinita.
I prodotti sono raggruppati in lotti da 20 unita'.
I tassi sono $Lotti / Settimana$.
Determinare i tassi di produzione per massimizzare la produzione e il profitto finale.

\subparagraph{Dati}

\begin{center}
    \begin{tabular}{||c c c c||}
        \hline
        Impianto & \text{Prodotto 1} & \text{Prodotto 2} & \text{Tempo Produttivo} \\
        \hline
        Impianto 1 & 1 & 0 & 4 \\
        \hline
        Impianto 2 & 0 & 2 & 12 \\
        \hline
        Impianto 3 & 3 & 2 & 18 \\
        \hline
        Profitto & 3000 & 5000 & \\
        \hline
    \end{tabular}
\end{center}

\begin{align}
    \text{$max Z = 3 \cdot x_1 + 5 \cdot x_2$}
    \text{$x_1 \leq 4$}
    \text{$2 \cdot x_2 \leq 12$}
    \text{$3 \cdot x_1 + 2 \cdot x_2 \leq 18$}
    \text{$x_1 \geq 0$}
    \text{$x_2 \geq 0$}
\end{align}

\paragraph{Considerazione}

TODO: Piano di Isolivello!
TODO: Curve di Isolivello!

$Z = valore di misura di prestazione$
$x_j = livello di attivita' j$
$c_j = incremento del valore della misura di prestazione corrispondente all'incremento di attivita' x_j$
$b_i = $
$a_{ij} = $

TODO: completa

In Programmazione Lineare la regione ammissibile e' un Poliedro Convesso in $R^n$
Se la regione e' limitata si dice Politopo.

TODO: Algoritmo del Simplesso

$opt Z = \Sigma _ {j=1} ^ {n} c_j \cdot x_j$
$vincoli \Rightarrow \Sigma _ {j=1} ^ {n} a_{ij} \cdot x_j \leq b_i$

$c_j \rightarrow coefficiente di costo$
$a_{ij} \rightarrow termine noto sinistro$
$b_{i} \rightarrow termine noto destro$

\paragraph{Assunzioni}
Le assunzioni di un problema PL.

\subparagraph{Proporzionalita'}
Il contributo di ogni attivita' al valore della funzione obiettivo e del vincolo e' proporzionale al livello di attivita'.

\subparagraph{Additivita'}
Il valore della funzione obiettivo e dei vincoli e' dato dalla somma dei contributi individuali delle rispettive attivita'.

\subparagraph{Divisibilita'}
Le variabili decisionali possono assumere qualsiasi valore all'interno della Regione di Ammissibilita'

\subparagraph{Certezza'}
I parametri di un problema di PL devono essere noti con certezza.

\subparagraph{Considerazioni Pratiche}

Alcuni vincoli possono includerne altri, in quel caso e' inutile conservarli entrambi, si puo' lasciare quello piu' "forte".
In programmazione lineare le soluzioni possono non essere intere.
TODO: Copia Slide

    \chapter{PNL - Bisezione e Newton}

\section{1}

Si determini nell'intervallo $[-3, 0]$ il punto massimo della funzione $f(x) = x^3 - 3x + 5$.

\paragraph{Analiticamente}

La derivata $g = \frac {df(x)} {dx} = 3x^2 - 3$. Le soluzioni di $g$ sono $(x = \pm 1)$.

La derivata $h = \frac {dg(x)} {dx} = 6x$. Applico le soluzioni di $g$ in h.

$x = 1 \notin [-3, 0]$. $h(-1) = -6$, $x=-1$ e' il punto di massimo.

\paragraph{Con bisezione e massimo 4 iterazioni}

La derivata $g = \frac {df(x)} {dx} = 3x^2 - 3$.

\begin{enumerate}
    \item $x_k = - \frac 3 2$. $g(x_k) = \frac {15} 4$. L'ottimo e' a destra di $x_k$. Il nuovo range e' $[- \frac 3 2, 0]$.
    \item $x_k = - \frac 3 4$. $g(x_k) = -\frac {21} {16}$. L'ottimo e' a sinistra di $x_k$. Il nuovo range e' $[-\frac 3 2, -\frac 3 4]$.
    \item $x_k = - \frac {9} 8$. $g(x_k) = \frac {51} {64}$. L'ottimo e' a destra di $x_k$. Il nuovo range e' $[-\frac {9} {8}, -\frac {3} {4}]$.
    \item $x_k = - \frac {15} {16}$. $g(x_k) = -\frac {93} {256}$. L'ottimo e' a sinistra di $x_k$. Il nuovo range e' $[-\frac {9} {8}, -\frac {15} {16}]$.
\end{enumerate}

Dopo 4 iterazioni $x_k = -\frac {33} {32}$.

\section{2}

Si determini il punto minimo con il metodo della bisezione della funzione $f(x) = x^4 - 14x^3 + 60x^2 - 70x$, con range $[0,1]$ e precisione $\epsilon = 0.1$.

La derivata $g = \frac {df(x)} {dx} = 4x^3 - 42x^2 + 120x - 70$.

\begin{enumerate}
    \item $x_k = \frac 1 2$. $g(x_k) = -20$. L'ottimo e a di $x_k$. Il nuovo range e' $[1, \frac 1 2]$.
    \item $x_k = \frac 1 {4}$. $g(x_k) = -\frac {681} {16}$. L'ottimo e a di $x_k$. Il nuovo range e' $[0, \frac {1} {4}]$.
    \item $x_k = \frac 1 {8}$. $g(x_k) = -\frac {7123} {128}$. L'ottimo e a di $x_k$. Il nuovo range e' $[0, \frac {1} {8}]$.
    \item $x_k = \frac 1 {16}$. $g(x_k) = -\frac {64167} {1024}$. L'ottimo e a di $x_k$. Il nuovo range e' $[0, \frac {1} {16}]$.
    \item $x_k = \frac 1 {32}$. $g(x_k) = -\frac {-543055} {8192}$. L'ottimo e a di $x_k$. Il nuovo range e' $[0, \frac {1} {32}]$.
\end{enumerate}

\section{3}

Si determini il punto di massimo della funzione $f(x) = -5x^2 + 5x + 4$.

La derivata $g = \frac {df(x)} {dx} = -10x + 5$.
La derivata $h = \frac {dg(x)} {dx} = -10$.
Il rapporto $r = \frac g h = \frac {10x - 5} {10} = x - \frac 1 2$

\paragraph{Newton da $x_0 = 0$}

\begin{itemize}
    \item $x_k = 0$. $r(x_k) = - \frac 1 2$. Diventa $x_k = \frac 1 2$.
    \item $x_k = \frac 1 2$. $r(x_k) = 0$. Resta $x_k = \frac 1 2$.
\end{itemize}

Mi fermo con $x_k = \frac 1 2$.

\paragraph{Newton da $x_0 = 6$}

\begin{itemize}
    \item $x_k = 6$. $r(x_k) = \frac {11} 2$. Diventa $x_k = \frac 1 2$.
    \item $x_k = \frac 1 2$. $r(x_k) = 0$. Resta $x_k = \frac 1 2$.
\end{itemize}

Mi fermo con $x_k = \frac 1 2$.

\section{4}

Si determini il punto di massimo della funzione $f(x) = -e^{-x^2}$.

La derivata $g = \frac {df(x)} {dx} = 2e^{-x^2}(x)$.
La derivata $h = \frac {dg(x)} {dx} = 2e^{-x^2}(1 - 2x^2)$.
Il rapporto $r = \frac g h = \frac {2e^{-x^2}(x)} {2e^{-x^2}(1 - 2x^2)} = \frac {x} {1 - 2x^2}$

\paragraph{Newton con $x_0 = \frac 1 4$}

\begin{itemize}
    \item $x_k = \frac 1 4$. $r(x_k) = \frac 2 7$. Diventa $x_k = -\frac 1 {28}$.
    \item $x_k = -\frac 1 {28}$. $r(x_k) = \frac {14} {391}$. Diventa $x_k = -1/28 - 14/391 = -\frac {783} {10948}$.
\end{itemize}

Mi fermo con $x_k = -\frac {783} {10948}$.

\paragraph{Newton con $x_0 = 1$}

\begin{itemize}
    \item $x_k = 1$. $r(x_k) = -1$. Diventa $x_k = 2$.
    \item $x_k = 2$. $r(x_k) = -\frac 2 7$. Diventa $x_k = \frac {16} {7}$.
\end{itemize}

Mi fermo con $x_k = \frac {16} {7}$.



    \include{chapters/esercitazione/programmazione-non-lineare/gradiente}

    \part{Laboratorio}
    \chapter{11-10-2022}

\section{}

\begin{center}
    \begin{tabular}{||c c c c c||}
        \hline
        Outlet & Boys & Women & Men & Cost \\
        \hline
        \hline
        TV & 5 & 1 & 3 & 600 \\
        \hline
        Mag & 2 & 6 & 3 & 500 \\
        \hline
        Target & 24 & 18 & 24 & \\
        \hline
    \end{tabular}
\end{center}

\begin{align}
    \text{$5x + 2y \geq 24$} \\
    \text{$x + 6y \geq 18$} \\
    \text{$3x + 3y \geq 24$} \\
    \text{$C(x,y) = 600x + 500y$}
\end{align}

\section{}

\begin{center}
    \begin{tabular}{||c c c c||}
        \hline
        Gasoline & Vapor & Octane & Price \\
        \hline
        \hline
        Regular & $\leq 7$ & $\geq 80$ & 9.80 \\
        \hline
        Premium & $\leq 6$ & $\geq 100$ & 12 \\
        \hline
    \end{tabular}
\end{center}

\begin{center}
    \begin{tabular}{||c c c c||}
        \hline
        Stock & Vapor & Octane & Barrels \\
        \hline
        \hline
        A & 8 & 83 & 2700 \\
        \hline
        B & 20 & 109 & 1350 \\
        \hline
        C & 4 & 74 & 4100 \\
        \hline
    \end{tabular}
\end{center}

\begin{align}
    \text{$\frac {A_i * V_{A} + B_i * V_{B} + C_i * V_{C}} {A_i+B_i+C_i} \leq V_i$} \\
    \text{$\frac {A_i * O_{A} + B_i * O_{B} + C_i * O_{C}} {A_i+B_i+C_i} \geq O_i$} \\
    \text{$\Sigma _ {i=0} ^ n A_i \leq Q_A$} \\
    \text{$\Sigma _ {i=0} ^ n B_i \leq Q_B$} \\
    \text{$\Sigma _ {i=0} ^ n C_i \leq Q_C$} \\
    \text{$C_{scarto} = P_{scarto} * (Q_A - \Sigma _ {i=0} ^ n A_i + Q_B - \Sigma _ {i=0} ^ n B_i + Q_C - \Sigma _ {i=0} ^ n C_i)$} \\
    \text{$C_{i} = P_i * (A_i + B_i + C_i)$} \\
    \text{$max \; C = C_{scarto} + \Sigma _ {i = 0} ^ n C_i$}
\end{align}

\section{}

Zone

\begin{center}
    \begin{tabular}{||c c c c c c||}
        \hline
        1 & 2 & 3 & 4 & 5 & 6 \\
        \hline
        12 & 11 & 10 & 9 & 8 & 7 \\
        \hline
        13 & 14 & 15 & 16 & 17 & 18 \\
        \hline
        24 & 23 & 22 & 21 & 20 & 19 \\
        \hline
        25 & 26 & 27 & 28 & 29 & 30 \\
        \hline
        36 & 35 & 34 & 33 & 32 & 31 \\
        \hline
    \end{tabular}
\end{center}



Ogni ripetitore copre anche le zone adiacenti.


    \chapter{18-10-2022}

\begin{align}
    \text{$Z - 40x - 50y = 0$} \\
    \text{$x + 2y + s1 = 40$} \\
    \text{$4x + 3y + s2 = 120$}
\end{align}

\begin{center}
    \begin{tabular}{|c|c|c|c|c|}
        -40 & -50 & 0 & 0 & 0 \\
        1 & 2 & 1 & 0 & 40 \\
        4 & 3 & 0 & 1 & 120 \\
    \end{tabular}
\end{center}

\paragraph{1}

\begin{center}
    \begin{tabular}{|c|c|c|c|c|}
        -40 & -50 & 0 & 0 & 0 \\
        $\frac 1 2$ & 1 & $\frac 1 2$ & 0 & 20 \\
        4 & 3 & 0 & 1 & 120 \\
    \end{tabular}
\end{center}

\begin{center}
    \begin{tabular}{|c|c|c|c|c|}
        -15 & 0 & 50 & 0 & 1000 \\
        $\frac 1 2$ & 1 & $\frac 1 2$ & 0 & 20 \\
        $\frac 5 2$ & 0 & $\frac 3 2$ & 1 & 60 \\
    \end{tabular}
\end{center}

\paragraph{2}

\begin{center}
    \begin{tabular}{|c|c|c|c|c|}
        -15 & 0 & 50 & 0 & 1000 \\
        $\frac 1 2$ & 1 & $\frac 1 2$ & 0 & 20 \\
        1 & 0 & $\frac 3 5$ & $\frac 2 5$ & 24 \\
    \end{tabular}
\end{center}

\begin{center}
    \begin{tabular}{|c|c|c|c|c|}
        0 & 0 & 59 & 6 & 1360 \\
        0 & 1 & $\frac 1 5$ & $\frac 1 5$ & 32 \\
        1 & 0 & $\frac 3 5$ & $\frac 2 5$ & 24 \\
    \end{tabular}
\end{center}

\section{Esercizio 2}

max 3x + 5y
x - y <= 1
2x - y <= 4
-2x + y <= 1

    \chapter{25-10-2022}

\begin{align}
    \text{$max 3x + 5y$} \\
    \text{$x - y \leq 1$} \\
    \text{$2x - y \geq 4$} \\
    \text{$-2x + y = 1$} \\
    \text{$x,y \geq 0$}
\end{align}

\begin{align}
\text{$min a + 4b + 3c$} \\
\text{$a - 2b - 2c \geq 3$} \\
\text{$-a - b + c \geq 5$} \\
\text{$a \geq 0$} \\
\text{$b \leq 0$} 
\end{align}

\paragraph{2}

\begin{align}
    \text{$Min : x_1 + 2 x_2 - 9 x_3 + 5 x_4 + 6 x_5$} \\
    \text{$x_1 - 2 x_2 + 3 x_3 - x_4 + 2 x_5 = 30$} \\
    \text{$x_1 + 3 x_2 + 5 x_3 + 2 x_4 - x_5 <= 10$} \\
    \text{$x_1, x_2,x_4,x_5 \geq 0, x_3 \leq 0$}
\end{align}

\begin{align}
    \text{$Min : x_1 + 2 x_2 + 9 x_3 + 5 x_4 + 6 x_5$} \\
    \text{$x_1 - 2 x_2 - 3 x_3 - x_4 + 2 x_5 = 30$} \\
    \text{$x_1 + 3 x_2 - 5 x_3 + 2 x_4 - x_5 <= 10$} \\
    \text{$x_1, x_2,x_3,x_4,x_5 \geq 0$}
\end{align}

\begin{align}
    \text{$Max : 30 y_1 + 10 y_2$} \\
    \text{$   y_1 +   y_2 \geq 1$} \\
    \text{$-2 y_1 + 3 y_2 \geq 2$} \\
    \text{$-3 y_1 - 5 y_2 \geq 9$} \\
    \text{$- y_1 + 2 y_2 \geq 5$} \\
    \text{$ 2 y_1 - y_2 \geq 6$} \\
    \text{$y_2 \geq 0$} 
\end{align}


\end{document}
