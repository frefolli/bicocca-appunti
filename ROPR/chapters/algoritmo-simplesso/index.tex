\chapter{Algoritmo del Simplesso}

\paragraph{Introduzione}

Questo algoritmo di tipo greedy permette di risolvere problemi di Programmazione Lineare.
Il tempo nel caso medio e' $\Theta(n)$.
Il tempo nel caso peggiore e' $O(e^n)$.

\paragraph{Definizioni}

\subparagraph{Frontiera}
La \textbf{Frontiera del Vincolo} e' un vincolo della regione ammissibile che abbia $\leq oppure =$.

\subparagraph{Vertice}
Un \textbf{Vertice} e' l'intersezione di due \textbf{Frontiere di Vincolo}.
I \textbf{Vertici Ammissibili} sono quelli che stanno nella \textbf{Regione Ammissibile}.
Due \textbf{Vertici} si dicono adiacenti se condividono $n-1$ \textbf{Frontiere di Vincolo}.

\subparagraph{Spigolo}
Uno \textbf{Spigolo} e' il segmento che collega due \textbf{Vertici} adiacenti.
Gli \textbf{Spigoli Ammissibili} sono quelli che stanno nella \textbf{Regione Ammissibile}.

\subparagraph{Test di Ottimalita'}
Si consideri ogni problema PL tale da ammettere soluzioni ottimali, se una soluzione vertice non ammette soluzioni certice a lei adiacenti con valore della funzione obiettivo Z migliore, allora la soluzione in questione e' ottimale.

\paragraph{Algoritmo}

\subparagraph{Inizializzazione}
Scegliere il vertice $(0,0)$ come soluzione iniziale, (vantaggiosa senza complicazione) se questa fa parte della Regione Ammissibile.

\subparagraph{Passo}
Si valuta lo spostamento nei vertici ammissibili adiacenti.
Con il test di ottimalita' si valuta se ci si puo' fermare.
Ci si sposta nel vertice che garantisce il valore della funzione obiettivo migliore.

\paragraph{Concetti Chiave}

\subparagraph{1}
Il metodo del simplesso non esplora, ma ispeziona solo i vertici ammissibili adiacenti.
Per ogni problema PL trovare una soluzione richiede di trovare il vertice ammissibile migliore.
Si richiede che la Regione Ammissibile sia limitata.
Il numero di vertici sale esponenzialmente.

\subparagraph{2}
E' un algoritmo iterativo con due passi, Inizializzazione e Test di Ottimalita'.

\subparagraph{3}
Quando possibile l'inizializzazione a $(0,0)$ e' preferibile e "ottimale".
Si possono applicare algoritmi apposititi per garantire l'ammissibilita' della soluzione iniziale.

\subparagraph{4}
E' piu' vantaggioso ascoltare gli adiacenti che tentare di verificare vertici piu' lontani perche' in minore quantita'.

\subparagraph{5}
A partire dal vertice corrente compara i risultati ma non calcola ogni volta i valori della funzione, ma i tassi di miglioramento della funzione obiettivo.

\subparagraph{6}
E' assicurato che si ottiene ad ogni passo una soluzione migliore perche' si cerca il migliore tasso di crescita.
