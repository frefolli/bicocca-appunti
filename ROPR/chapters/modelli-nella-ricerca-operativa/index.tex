\chapter{Modelli nella Ricerca Operativa}

\section{Introduzione}

\paragraph{Problemi di Ottimizzazione}

Dato un problema di ottiizzazione:

\begin{align}
$opt f(x)$
$s.a. x in X$
\end{align}

si danno le seguenti definizioni:

\subparagraph{$f(x)$}

La \textbf{funzione obiettivo} e' la funzione della quale cerchiamo un argmento $x in X$ ottimale.
L'operatore di ottimizzazione $opt$ trova argomenti ottimi o ottimali.
Non potendo sempre disporre di ottimi assoluti, si indagano gli ottimali, i quali, come dice il nome, possono essere multipli.

\subparagraph{$opt$}

Questo operatore e' un placeholder che sta per:
\[
    $opt$ =
    \begin{cases}
        \text {$max$} \\
        \text {$min$} \\
    \end{cases}
\]

In particolare si nota che $max f(x) = - min (- f(x))$. \\*
E' possibile ridursi a entrambe le situazioni per sfruttare i vantaggi di eventuali semplificazioni o riduzioni del problema iniziale!

\subparagraph{$X$}

La \textbf{regione d'ammissibilita'} e' un sottoinsieme $X subset R^n$ che delimita' le possibili soluzioni al problema di ottimizzazione.
Le soluzioni che non soddisfano questo requisito sono dette \textbf{soluzioni inammissibili}.

\subparagraph{$x$}

La generica soluzione $x in X$. In caso di funzione multivariabile, questa assume la forma di una n-upla, come si nota dal dominio della \textbf{funzione obiettivo}.

\paragraph{Ottimizzazione Vincolata o non Vincolata}

\subparagraph{Vincoli}

Se l'ottimizzazione cerca soluzioni in una regione di ammissibilita' $X$ che sia un sottoinsieme proprio di $R^n$, ovvero nel caso in cui non coincida, si parla di \textbf{Ottimizzazione Vincolata}.
Nel caso in cui $X = R^n$ questa e' definita \textbf{Ottimizzazione non Vincolata}.

\subparagraph{Esempi}
Alcuni esempi di ottimizzazione vincolata sono:

\begin{itemize}
Se $X = Z^n$ si definisce \textbf{Ottimizzazione Intera}.
Se $X = Z^n$ si definisce \textbf{Ottimizzazine Binaria}.
Se le variabili appartengono a spazi differenti (es: $N x Z x R$), si parla di {Ottimizzazione Mista}.
\end{itemize}

\subparagraph(Programmazione Matematica}

Se la regione di ammissibilita' $X$ e' espressa tramite vincoli aritmetici (equazioni e disequazioni), si tratta di \textbf{Programmazione Matematica}. \\

Ogni vincolo e' definito come segue:

\[
    v(x) =
    \begin{cases}
        \text{$g(x) \geq 0$} \\
        \text{$g(x) = 0$} \\
        \text{$g(x) \leq 0$} \\
    \end{cases}
\]

Di consequenza la regione $X$ e' esprimibile con:

$X$ = \{ 
    x \mid 
    $X \in R^n$ & $\and$ \{ 
        \begin{cases} 
            $\geq$ \\ $=$ \\ $\leq$
        \end{cases}
    \} & $0$ & \text{for $i$ in $[1,m]$}
\}

\subparagraph{Esiti di Problemi}

\begin{itemize}
    Si dice \textbf{Problema Inammissibile} se $X = empty$.
    Si dice \textbf{Problema Illimitato} se non esise un ottimale, in particolare:
    \begin{itemize}
        Se $opt = min \and for each c \in R, exists x in \in X so that f(x) \leq c$, e' illimitato inferiormente
        Se $opt = max \and for each c \in R, exists x in \in X so that f(x) \geq c$, e' illimitato superiormente
    \end{itemize}
    Si dice \textbf{Problema con Soluzioni Ottime Multiple} (o Infinite) se tutte le soluzioni ottimali hanno lo stesso grado di ottimizzazione, ovvero se non esiste una soluzione migliore delle altre.
    Si dice \textbf{Problema con Soluzione Ottima Unica} nel caso semplice in cui esiste una e una sola soluzione ottimale (= ottimo).
\end{itemize}

\subparagraph{Localizzazione di Ottimi}

Un ottimo locale $y \in X$ si dice globale se:
\[
    \begin{cases}
        $f(y) \geq f(x)$ && \text{$opt = max$} \\
        $f(y) \leq f(x)$ && \text{$opt = min$}
    \end{cases}
\]

E' importante notare che un problema di ottimizzazione puo' avere sia piu' di un ottimo locale, che piu' di un ottimo globale.
Un ottimo globale e' anche locale.
Si mette in evidenza che i vincoli possono assumere caratteristiche non lineari se scomponibili in fattori lineari.

\subparagraph{Linearita'}
Si ricorda che una funzione e' lineare se, per esempio e' nella forma $\Sigma a_i x_i + b$.

\subparagraph{Programmazione Lineare}
I vincoli sono espressi tramite equazioni e disequzioni lineari, la funzione obiettivo e' lineare.

\subparagraph{Programmazione Lineare Intera}
E' un problema di programmazione lineare con regione d'ammissione ristretta a $X = Z^n$.

\subparagraph{Programmazione non Lineare}
I vincoli o la funzione obiettivo hanno caratteristiche non lineari.
