\chapter{Introduzione alla Programmazione Lineare}

\paragraph{Esempio}

La Wyndor Glass CO produce vetri di elevata qualita', incluso finestre e porte.

\begin{itemize}
    \item Impianto 1: produce cornici in alluminio e le altre componenti
    \item Impiano 2: produce cornici di legno
    \item Impianto 3: produce i vetri e assembla i prodotti
\end{itemize}

Si vuole produrre:
\begin{itemize}
    \item Prodotto 1 - una porta di vetro (impianti 1, 3)
    \item Prodotto 2 - finestra con doppia apertura (impianti 2, 3)
\end{itemize}

La domanda e' virtualmente infinita.
I prodotti sono raggruppati in lotti da 20 unita'.
I tassi sono $Lotti / Settimana$.
Determinare i tassi di produzione per massimizzare la produzione e il profitto finale.

\subparagraph{Dati}

\begin{center}
    \begin{tabular}{||c c c c||}
        \hline
        Impianto & \text{Prodotto 1} & \text{Prodotto 2} & \text{Tempo Produttivo} \\
        \hline
        Impianto 1 & 1 & 0 & 4 \\
        \hline
        Impianto 2 & 0 & 2 & 12 \\
        \hline
        Impianto 3 & 3 & 2 & 18 \\
        \hline
        Profitto & 3000 & 5000 & \\
        \hline
    \end{tabular}
\end{center}

\begin{align}
    \text{$max Z = 3 \cdot x_1 + 5 \cdot x_2$}
    \text{$x_1 \leq 4$}
    \text{$2 \cdot x_2 \leq 12$}
    \text{$3 \cdot x_1 + 2 \cdot x_2 \leq 18$}
    \text{$x_1 \geq 0$}
    \text{$x_2 \geq 0$}
\end{align}

\paragraph{Considerazione}

TODO: Piano di Isolivello!
TODO: Curve di Isolivello!

$Z = valore di misura di prestazione$
$x_j = livello di attivita' j$
$c_j = incremento del valore della misura di prestazione corrispondente all'incremento di attivita' x_j$
$b_i = $
$a_{ij} = $

TODO: completa

In Programmazione Lineare la regione ammissibile e' un Poliedro Convesso in $R^n$
Se la regione e' limitata si dice Politopo.

TODO: Algoritmo del Simplesso

$opt Z = \Sigma _ {j=1} ^ {n} c_j \cdot x_j$
$vincoli \Rightarrow \Sigma _ {j=1} ^ {n} a_{ij} \cdot x_j \leq b_i$

$c_j \rightarrow coefficiente di costo$
$a_{ij} \rightarrow termine noto sinistro$
$b_{i} \rightarrow termine noto destro$

\paragraph{Assunzioni}
Le assunzioni di un problema PL.

\subparagraph{Proporzionalita'}
Il contributo di ogni attivita' al valore della funzione obiettivo e del vincolo e' proporzionale al livello di attivita'.

\subparagraph{Additivita'}
Il valore della funzione obiettivo e dei vincoli e' dato dalla somma dei contributi individuali delle rispettive attivita'.

\subparagraph{Divisibilita'}
Le variabili decisionali possono assumere qualsiasi valore all'interno della Regione di Ammissibilita'

\subparagraph{Certezza'}
I parametri di un problema di PL devono essere noti con certezza.

\subparagraph{Considerazioni Pratiche}

Alcuni vincoli possono includerne altri, in quel caso e' inutile conservarli entrambi, si puo' lasciare quello piu' "forte".
In programmazione lineare le soluzioni possono non essere intere.
TODO: Copia Slide
