\chapter{Introduzione alla Programmazione Lineare}

\paragraph{Esempio}

La Wyndor Glass CO produce vetri di elevata qualita', incluso finestre e porte.

\begin{itemize}
    \item Impianto 1: produce cornici in alluminio e le altre componenti
    \item Impiano 2: produce cornici di legno
    \item Impianto 3: produce i vetri e assembla i prodotti
\end{itemize}

Si vuole produrre:
\begin{itemize}
    \item Prodotto 1 - una porta di vetro (impianti 1, 3)
    \item Prodotto 2 - finestra con doppia apertura (impianti 2, 3)
\end{itemize}

La domanda e' virtualmente infinita.
I prodotti sono raggruppati in lotti da 20 unita'.
I tassi sono $Lotti / Settimana$.
Determinare i tassi di produzione per massimizzare la produzione e il profitto finale.

\subparagraph{Dati}

\begin{center}
    \begin{tabular}{||c c c c||}
        \hline
        Impianto & \text{Prodotto 1} & \text{Prodotto 2} & \text{Tempo Produttivo} \\
        \hline
        Impianto 1 & 1 & 0 & 4 \\
        \hline
        Impianto 2 & 0 & 2 & 12 \\
        \hline
        Impianto 3 & 3 & 2 & 18 \\
        \hline
        Profitto & 3000 & 5000 & \\
        \hline
    \end{tabular}
\end{center}

\begin{align}
    \text{$max Z = 3 \cdot x_1 + 5 \cdot x_2$} \\
    \text{$x_1 \leq 4$} \\
    \text{$2 \cdot x_2 \leq 12$} \\
    \text{$3 \cdot x_1 + 2 \cdot x_2 \leq 18$} \\
    \text{$x_1 \geq 0$} \\
    \text{$x_2 \geq 0$}
\end{align}

\paragraph{Considerazioni}

\begin{itemize}
    \item $Z$ = valore di misura di prestazione.
    \item $x_j$ = livello di attivita' $j$.
    \item $c_j$ = incremento del valore della misura di prestazione $Z$ corrispondente all'incremento di un'unita' del valore dell'attivita' $x_j$.
    \item $b_i$ = quantita' di risorsa $i$ allocabile alle attivita' $x_j$.
    \item $a_{ij}$ = quantita' di risorsa $i$ consumata da ogni attivita' $x_j$.
\end{itemize}

In Programmazione Lineare la regione ammissibile e' un Poliedro Convesso in $R^n$
Se la regione e' limitata si dice Politopo.

\begin{equation}
    opt Z = \Sigma _ {j=1} ^ {n} c_j \cdot x_j
\end{equation}

\begin{equation}
    vincoli \Rightarrow \Sigma _ {j=1} ^ {n} a_{ij} \cdot x_j \leq b_i
\end{equation}

\begin{equation}
    c_{j} \rightarrow coefficiente \; di \; costo
\end{equation}

\begin{equation}
    a_{ij} \rightarrow termine \; noto \; sinistro
\end{equation}

\begin{equation}
    b_{i} \rightarrow termine \; noto \; destro
\end{equation}

\paragraph{Assunzioni}
Un problema PL si poggia su quattro assunzioni implicite.

\subparagraph{Proporzionalita'}
Il contributo di ogni attivita' al valore della funzione obiettivo e del vincolo e' proporzionale al livello di attivita'.

\subparagraph{Additivita'}
Il valore della funzione obiettivo e dei vincoli e' dato dalla somma dei contributi individuali delle rispettive attivita'.

\subparagraph{Continuita'}
Qualunque valore delle variabili decisionali in $R^n$ e' accettabile.

\subparagraph{Certezza'}
I parametri di un problema di PL devono essere noti con certezza.

\subparagraph{Divisibilita'}
Questa assunzione non e' sempre garantita, varia in base al problema.
Le variabili decisionali possono assumere qualsiasi valore all'interno della Regione di Ammissibilita', inclusi i valori non interi che soddisfino i vincoli. Quindi le variabili decisionali sono variabili continue. \\*
In certi problemi puo' essere necessario avere soluzioni intere perche' la logica non ci consente di spezzare unita' intrinsecamente e logicamente atomice: non si possono assumere 3.5 dipendenti!

\paragraph{Considerazioni Pratiche}

\subparagraph{Vincoli Prolissi}
Alcuni vincoli possono includerne altri, in quel caso e' inutile conservarli entrambi, si puo' lasciare quello piu' "forte".

\subparagraph{Soluzioni Intere}
La soluzione del problema di PL non garantisce l’assunzione di divisibilità.
In certi casi sono opportune o necessarie per senso logico soluzioni intere, ma non sempre le soluzione di un problema PL le possono garantire. \\

Se la soluzione ottimale del problema PL e' intera allora e' anche ottimale per il problema considerato. Altrimenti si presentano due strade: \\

Aggiungere vincoli che garantiscano che la variaibili di decisione siano interi, riducendo il problema PL in PLI (Programmazione Lineare Intera) \\*

Arrotondare la soluzione (per eccesso o difetto, in modo opportuno), ma questo non garantisce l'ottimalita' della soluzione intera cosi' ottenuta.
