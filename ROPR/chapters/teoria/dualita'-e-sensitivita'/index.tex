\chapter{Teoria della Dualita' e Analisi di Sensitivita'}

\paragraph{Instroduzione}

Ogni problema di programmazione lineare ha associato un altro problema di programmazione lineare chiamato \textbf{duale}.
E' utile in programmazione lineare studiare il rapporto tra il problema duale e il problema originale (chiamato \textbf{primale}).
I \textbf{prezzi ombra} sono forniti dalla soluzione ottimale del problema duale.
Ma esistono ulteriori applicazioni della teoria della dualita'.

\paragraph{Analisi di Sensitivita'}

L'\textbf{analisi di sensitivita'} e' lo studio dell'effetto dei parametri del problema sulle soluzioni del problema stesso.
Infatti molti dei parametri usati in un modello PL sono costruiti con stime di condizioni future o probabili.
Dalla teoria del simplesso uno dei prerequisiti e' la conoscenza a priori con ragionevole certezza dei parametri del problema.
Questo studio aiuta a verificare e affinare il modello.
Inoltre alcuni parametri, come le risorse rese disponibili, sono scelte simil-arbitrarie che vengono gestite dal compartimento manageriale, quindi sono anch'esse perfettibili sotto ogni punto di vista.

\paragraph{Problema Duale}

Prima di tutto puo' essere utile un confronto visivo tra il \textbf{primale} e il relativo \textbf{duale}:

\begin{align*}
    Primale &  &  & Duale \\
    max \; Z &= \Sigma ^ n _ {j=1} c_j \cdot x_j &
    min \; Z &= \Sigma ^ m _ {i=1} b_i \cdot y_i \\
    \Sigma ^ n _ {j=1} a_{ij} \cdot x_{j} \leq b_{i} \; \; & \forall i \in [1,m] &
    \Sigma ^ m _ {i=1} a_{ij} \cdot y_{i} \leq c_{j} \; \; & \forall j \in [1,n] \\
    x_j \geq 0 \; \; & \forall j \in [1,n] &
    y_i \geq 0 \; \; & \forall i \in [1,m] \\
\end{align*}

Come si puo' notare, se il problema primale e' di massimo, il duale e' di minimo e vice versa.
I coefficienti del primale sono i termini noti del duale e i termini noti del primale sono i coefficienti del duale.
I coefficienti di ogni variabile nei vincoli del primale corrisponde a un coefficiente del vincolo del duale.
Da notare l'uso di simboli uguali per indicare lo spostamento di variabili. \\

Da un punto di vista puramente matematico, la matrice del duale e' composta in modo tale da fare supporre che se si sia calcolata la trasposta della matrice del problema primale, infatti:

\begin{center}
    \begin{tabular}{||c c c c c c c||}
        \hline
        & $x_1$ & $x_2$ & ... & $x_n$ & & \\
        \hline
        \hline
        $y_{1}$ & $a_{11}$ & $a_{12}$ & ... & $a_{1n}$ & $\leq$ & $b_{1}$ \\
        \hline
        $y_{2}$ & $a_{21}$ & $a_{22}$ & ... & $a_{2n}$ & $\leq$ & $b_{2}$ \\
        \hline
        ... & ... & ... & ... & ... & ... & ... \\
        \hline
        $y_{m}$ & $a_{m1}$ & $a_{m2}$ & ... & $a_{mn}$ & $\leq$ & $b_{m}$ \\
        \hline
        & VI & VI & ... & VI & & \\
        \hline
        & $c_{1}$ & $c_{2}$ & ... & $c_{n}$ & & \\
        \hline
    \end{tabular}
\end{center}

Nella risoluzione con forma tabellare i valori delle variabili slack in riga 0 nel problema primale equivale alla soluzione del problema duale e vice versa.

\paragraph{Proprieta' di Dualita' Debole}
fonte: Teoria della Dualita' - slide prof Stella \\

se x è una soluzione ammissibile per il problema primale, e y è una soluzione ammissibile per il corrispondente problema duale, allora vale la seguente diseguaglianza:
\[
    cx \leq yb
\]

\paragraph{Proprieta' di Dualita' Forte}

se $x^*$ è una soluzione ottimale per il problema primale, e $y^*$ è una soluzione ottimale per il corrispondente problema duale, allora vale la seguente eguaglianza:
\[
    cx^* = y^*b
\]

\paragraph{Considerazioni}

Queste due proprietà, se considerate insieme, implicano che la diseguaglianza vale per soluzioni ammissibili se una o entrambe non sono ottimali per i corrispondenti problemi, l’uguaglianza vale solo se entrambe le soluzioni sono ottimali.
\[
    cx < yb
\]

Ad ogni iterazione, il metodo del simplesso trova una specifica coppia di soluzioni per i due problemi, dove la soluzione del primale è ammissibile mentre quella del duale non è ammissibile, fatta eccezione per l’ultima iterazione, quella che vede l’arresto del metodo del simplesso.

\newpage

\paragraph{Prolog Time}
Pericolo fallacia logica!

Per comodita' chiamo: \\
ammissibile(x) = f(x) \\
ottimale(x) = g(x) \\

\[
f(x) \land f(y) \Leftrightarrow cx < yb \lor cx = yb
\]
\[
f(x) \land g(x) \land f(y) \land g(y) \Leftrightarrow cx = yb
\]

\[ A = \{ f(x), f(y), \neg g(x), \neg g(y) \} \]
\[ B = \{ f(x), f(y), \neg g(x), g(y) \} \]
\[ C = \{ f(x), f(y), g(x), \neg g(y) \} \]

Definisco Modus Ponens = MP = $[ (p \rightarrow q) \land p]\vdash q$

\subparagraph{A}

Per A, applico $MP$:
\[
    (f(x) \land f(y) \Leftrightarrow cx < yb \lor cx = yb) \land (f(x) \land f(y) \land \neg g(x) \land \neg g(y)) \vdash cx < yb \lor cx = yb
\]
\[
    (f(x) \land g(x) \land f(y) \land g(y) \Leftrightarrow cx = yb) \land (f(x) \land f(y) \land \neg g(x) \land \neg g(y)) \vdash \neg (cx = yb)
\]

Quindi una semplice risoluzione:

\[
    ((cx < yb) \lor (cx = yb)) \land \neg (cx = yb) \vdash (cx < yb)
\]

\subparagraph{B}

Per A, applico $MP$:
\[
    (f(x) \land f(y) \Leftrightarrow cx < yb \lor cx = yb) \land (f(x) \land f(y) \land \neg g(x) \land g(y)) \vdash cx < yb \lor cx = yb
\]
\[
    (f(x) \land g(x) \land f(y) \land g(y) \Leftrightarrow cx = yb) \land (f(x) \land f(y) \land \neg g(x) \land g(y)) \vdash \neg (cx = yb)
\]

Quindi una semplice risoluzione:

\[
    ((cx < yb) \lor (cx = yb)) \land \neg (cx = yb) \vdash (cx < yb)
\]

\subparagraph{A}

Per A, applico $MP$:
\[
    (f(x) \land f(y) \Leftrightarrow cx < yb \lor cx = yb) \land (f(x) \land f(y) \land g(x) \land \neg g(y)) \vdash cx < yb \lor cx = yb
\]
\[
    (f(x) \land g(x) \land f(y) \land g(y) \Leftrightarrow cx = yb) \land (f(x) \land f(y) \land g(x) \land \neg g(y)) \vdash \neg (cx = yb)
\]

Quindi una semplice risoluzione:

\[
    ((cx < yb) \lor (cx = yb)) \land \neg (cx = yb) \vdash (cx < yb)
\]

\paragraph{Tasformazione Primale - Duale}

Per trovare il duale si consiglia di rifarsi alla seguente tabella:

\begin{center}
    \begin{tabular}{||c | c||}
        \hline
        Primale & Duale \\
        \hline
        \hline
        Min Z(x) = $c^Tx$ & Max W(y) = $b^Ty$ \\
        \hline
        $a^T_ix \geq b_i$ & $y_i \geq 0$ \\
        \hline
        $a^T_ix \leq b_i$ & $y_i \leq 0$ \\
        \hline
        $a^T_ix = b_i$ & $y_i$ libera \\
        \hline
        $x_j \geq 0$ & $y^TA_j \leq c_j$ \\
        \hline
        $x_j \leq 0$ & $y^TA_j \geq c_j$ \\
        \hline
        $x_j$ libera & $y^TA_j = c_j$ \\
        \hline
    \end{tabular}
\end{center}