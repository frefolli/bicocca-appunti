\chapter{Algoritmo del Simplesso}

\section{Teoria}

\paragraph{Introduzione}

Questo algoritmo di tipo greedy permette di risolvere problemi di Programmazione Lineare.
E' uno degli algoritmi piu' efficienti che si conoscano.
Il tempo nel caso medio e' $\Theta(n)$, lineare rispetto al numero di variabili.
Il tempo nel caso peggiore e' $O(e^n)$, esponenziale rispetto al numero di variabili.

\paragraph{Definizioni}

\subparagraph{Frontiera}
La \textbf{Frontiera del Vincolo} e' un vincolo della regione ammissibile che abbia $\leq oppure =$.

\subparagraph{Vertice}
Un \textbf{Vertice} e' l'intersezione di due \textbf{Frontiere di Vincolo}.
I \textbf{Vertici Ammissibili} sono quelli che stanno nella \textbf{Regione Ammissibile}.
Due \textbf{Vertici} si dicono adiacenti se condividono $n-1$ \textbf{Frontiere di Vincolo}.

\subparagraph{Spigolo}
Uno \textbf{Spigolo} e' il segmento che collega due \textbf{Vertici} adiacenti.
Gli \textbf{Spigoli Ammissibili} sono quelli che stanno nella \textbf{Regione Ammissibile}.

\subparagraph{Test di Ottimalita'}
Si consideri ogni problema PL tale da ammettere soluzioni ottimali, se una soluzione vertice non ammette soluzioni certice a lei adiacenti con valore della funzione obiettivo Z migliore, allora la soluzione in questione e' ottimale.

\paragraph{Algoritmo}

\subparagraph{Inizializzazione}
Scegliere il vertice $(0,0)$ come soluzione iniziale, (vantaggiosa senza complicazione) se questa fa parte della Regione Ammissibile.

\subparagraph{Passo}
Si valuta lo spostamento nei vertici ammissibili adiacenti.
Con il test di ottimalita' si valuta se ci si puo' fermare.
Ci si sposta nel vertice che garantisce il valore della funzione obiettivo migliore.

\paragraph{Concetti Chiave}

\subparagraph{1}
Il metodo del simplesso non esplora, ma ispeziona solo i vertici ammissibili adiacenti.
Per ogni problema PL trovare una soluzione richiede di trovare il vertice ammissibile migliore.
Si richiede che la Regione Ammissibile sia limitata.
Il numero di vertici sale esponenzialmente.

\subparagraph{2}
E' un algoritmo iterativo con due passi, Inizializzazione e Test di Ottimalita'.

\subparagraph{3}
Quando possibile l'inizializzazione a $(0,0)$ e' preferibile e "ottimale".
Si possono applicare algoritmi apposititi per garantire l'ammissibilita' della soluzione iniziale.

\subparagraph{4}
E' piu' vantaggioso ascoltare gli adiacenti che tentare di verificare vertici piu' lontani perche' in minore quantita'.

\subparagraph{5}
A partire dal vertice corrente compara i risultati ma non calcola ogni volta i valori della funzione, ma i tassi di miglioramento della funzione obiettivo.

\subparagraph{6}
E' assicurato che si ottiene ad ogni passo una soluzione migliore perche' si cerca il migliore tasso di crescita.

\section{Procedura Algebrica}

La forma standard di un problema PL comprende: opt = max, vincoli in $\geq$ e vincoli di non negativita'. \\
La forma aumentata consiste in: opt = max, vincoli in $=$ e vincoli di non negativita', introducendo nuove variabili. \\

La soluzione del problema PL in forma aumentanta (chiamata \textbf{soluzione aumentata}) giace sulla frontiera di vincolo.
Questo e' utile per utilizzare l'algoritmo del simplesso.

\paragraph{Forma Aumentata}

Si definisce modello in forma aumentata un modello di programmazione lineare formato da:

\begin{itemize}
	\item operazione $max \; Z$
	\item vincoli in forma $\Sigma _ {i = 1} ^ {m} a_{i,j} \cdot x_{i,j} = b_j$
	\item $\forall i \in [1,m] \mid x_i >= 0$
	\item $\forall j \in [1,n] \mid b_j >= 0$
\end{itemize}

E' possibile ricondursi alla forma aumentata a partire da quella standard, ma non e' strettamente necessario. \\
E' preferibile ricondursi alla forma aumentata direttamente se il modello e' in forma non standard e non e' vitale.

\paragraph{Ottenimento Forma Aumentata}

Si introduce una \textbf{variabile slack} per convertire un vincolo di disuguaglianza in uguaglianza.

\begin{align}
    \text{$x_1 \leq 4$} \\
    \text{$x_1 \geq 0$}
\end{align}

Si trasforma introducendo la variabile $x_s$.

\begin{align}
    \text{$x_1 + x_s = 4$} \\
    \text{$x_s \geq 0$} \\
    \text{$x_1 \geq 0$}
\end{align}

E' possibile riscrivere anche la funzione obiettivo in questo modo:

\begin{align}
    \text{$max \; Z = 3 \cdot x_1 + 5 \cdot x_2$}
\end{align}

Diventa:

\begin{align}
    \text{$max \; Z$} \\
    \text{$Z - 3 \cdot x_1 - 5 \cdot x_2 = 0$} \\
    \text{$x_s1, x_2, Z \geq 0$}
\end{align}


\paragraph{Variabili di Base}

Detto un modello con $z$ variabili ($x$ decisionali, $y$ slack) e $w$ equazioni. \\
Ho $g = z - grado(matrice \; vincoli)$ gradi di liberta' \\
Quindi fisso $g$ variabili uguali a zero per calcolare le altre. Queste $g$ variabili vengono dette \textbf{non di base}. Il resto vengono dette \textbf{di base}. \\
L'insieme delle variabili di base costituiscono una \textbf{base}. \\
Se le variabili di base soddisfano i requisiti di non negativita', la soluzione viene detta \textbf{soluzione di base ammissibile}.

Una soluzione di base e' una soluzione della forma aumentata dove le variabili possono essere di base o non di base. \\

Due soluzioni di base ammissibili sono adiacenti se condividono le stesse variaili non di base tranne una (non i valori, le variabili!). \\
Questo implica che una variabile non di base e una variabile di base si scambiano di posto.

\section{Algoritmo}

\paragraph{Inizializzazione}

Inizializzo le $g$ variabili decisionali (e eventualmente slack) a 0. \\
Secondo il \textbf{concetto chiave 3}. \\
Quindi calcolo la soluzione di base ottimale.

\paragraph{Test di ottimalita'}

Uso il test di ottimalita' per verificare se la soluzione di base e' ottimale.
Se la funzione obiettivo ha coefficienti positivi l'ottimizzazione non e' finita.

Ci si muove qunidi lungo uno degli spigoli, ovvero si scambia una variabile non di base con una di base.
Si usano i tassi di crescita per individuare qual'e' lo spostamento piu' vantaggioso. Secondo i \textbf{concetti chiave 5 e 6}.

Poi bisogna determinare la quantita' di spostamento. Si usa il test del rapporto minimo. \\
In pratica si calcola la dipendenza delle variabili di base rispetto alla variabile non di base entrante.
Si sceglie il minimo valore che manda a zero una variabile di base, quindi della variabile di base e' uscente.
Quindi si ricalcolano i valori delle altre variabili di base risolvendo i vincoli con i nuovi valori delle variabili non di base, della variabile entrante e di quella uscente. \\*
Questo si fa con la procedura di \textbf{eliminazione di Gauss-Jordan}. \\

Essendo la funzione obiettivo espressa tramite un vincolo, anch'essa viene modificata nella riassegnazione della soluzione di base. La variabile slack Z pero' rimane immutata. Z non e' una variabile di base. \\

\section{Forma Tabellare}

La forma migliore per rappresentare i vincoli della forma aumentata e' usare una matrice. \\*
Questa permette di rappresentare tutte le caratterische del problema.

Ho una tabella in cui ogni riga rappresenta un vincolo (esclusi quelli di non negativita') della forma aumentata. La prima riga e' Z  - f(x) =0 .

\paragraph{Risoluzione con Deficit di Base}

La base e' formato da $n$ variabili, dove $n$ e' il numero di equazioni nei vincoli del modello. \\*
Dette $k$ il numero di variabili di slack, un modello necessita' di $m = n - k$ variabili "artificiali" per creare la soluzione di base ammissibile, ovvero creare il tableau iniziale. \\

Se $m > 0$ bisogna creare $m$ variabili artificiali: $A_1, A_2 ... A_m$. \\*
Una per ogni vincolo che non contiene variabili di slack. \\

Le variabili artificiali vengono quindi sommate dentro i vincoli. Quindi si riempiono le righe della tabella. La prima riga e' particolare, viene  riempita con soli zero eccetto gli uni nelle colonne delle variabili artificiali. \\

Quindi alla prima riga della tabella si sottrae la somma delle righe contenenti variabili artificiali. \\

A questo punto si applica il normale algoritmo del simplesso. Nel momento in cui tutte le variabili artificiali saranno fuori dalla base, si procede a troncare la tabella, eliminando le colonne di tali variabili. Quindi si sostituisce la prima riga con una tradizionale $Z - f(x)$. Dopo la trasformazione, alla prima riga viene sottratta una combinazione lineare delle righe, quelle che avevano una variabile artificiale, che azzeri i coefficienti nella prima riga di tutte le variabili in base. \\

L'algoritmo del simplesso puo' riprendere normalmente. Questa operazione va effettuata nonappena si presenta l'occasione. \\

\paragraph{Passo 1}

Se la prima riga ha almeno un coefficiente negativo, posso ottimizzare ulteriorimente.
Quindi selezione la colonna con il coefficiente strettamente negativo piu' piccolo. \\*
Dico che e' la \textbf{colonna pivot}.

\paragraph{Passo 2}

Cerco le righe che hanno nella \textbf{colonna pivot} un coefficiente strettamente positivo.
Quindi, detti $n_i$ e $c_i$ il termine noto e il coefficiente di quella riga, cerco la riga tale per cui il rapporto $\frac {n_i} {c_i}$ piu' piccolo. \\*
Chiamo quella riga \textbf{riga pivot}. \\
Il valore nella cella che e' incrocio tra \textbf{riga pivot} e \textbf{colonna pivot} viene detto \textbf{valore pivot}.

\paragraph{Passo 3}

Moltiplico la \textbf{riga pivot} per $\frac {1} {valore \; pivot}$.

Quindi per ogni riga non pivot:
\begin{itemize}
    \item se il $c_i$ > 0, sottraggo alla riga: $|c_i| * (riga \; pivot)$.
    \item se il $c_i$ < 0, sommo alla riga: $|c_i| * (riga \; pivot)$.
\end{itemize}

\paragraph{4}

Torno a 1).
