\chapter{Combinatorial Testing}

\section{Dalla specifica ai test case}

\paragraph{Decomporre la specifica}

La specifica e' gonfia e confusionaria, occorre spezzettarla in caratteristiche individualmente testabili
Ci si concentra separatamente su come la specifica descrive il comportamento del sistema rispetto a input, parametri e risultati.
Quindi si applica \textbf{boundary testing} per ognuno di questi.

\paragraph{Aggregazione per Combinazione}

Si formano delle combinazioni che permettono di testare contemporaneamente diverse di queste condizioni.
Quindi si implementano queste combinazioni nei \textbf{text case}.

\section{Concetti Chiave}

\begin{itemize}
    \item Category-partition Testing (Decompose)
    \item Pairwise Testing (Aggregate)
    \item Catalog-based Testing (Decompose)
\end{itemize}

\paragraph{Category-partition Testing}

Identificazione manuale dei rappresentanti per i test objectives delle categorie che caratterizzano l'input space.
Quindi si generano automaticamnete tutte le combinazioni rilevanti ai fini del testing.

\paragraph{Pairwise Testing}

Testing sistematico delle interazioni tra gli attributi dell'input space.
Questo puo' essere fatto attraverso software specifici che permettono di generare quelle combinazioni sufficienti a testare tutte le interazioni a coppie.
Chiaramente e' possibile che le interazioni siano piu' complesse di semplici coppie. Si fare la stessa cosa con combinazioni di n-attributi contemporaneamente. Il concetto rimane lo stesso.

\paragraph{Catalog-based Testing}

Aggregazione o sintesi dell'esperienza dei test designers in un particolare dominio  per identificare i valori degli attributi.

\section{Altri Concetti Chiave}

\begin{itemize}
    \item inputs
    \item parametri
    \item risultati
    \item caratteristiche
    \item controllabilita'
\end{itemize}

\paragraph{inputs}

Gli input sono esplicitamente indicati dalla specifica.

\paragraph{parametri}

Altri fattori che possono essere variabili e che alterano il comportamento del software.

\paragraph{risultati}

Sono le condizioni misurabili che il programma resistituisce, sia come valori sia come comportamento.
Si vuole testare tutti i distiti tipi di risultati osservabili durante i test.

\paragraph{caratteristiche}

Sono caratteristiche intriseche di input, parametri e risultati.

\paragraph{controllabilita'}

Si definisce controllabile un qualcosa che puo' essere ottenuto durante l'esecuzione.
Non si e' interessati a testare valori che si sa con certezza che non possano essere ottenuti a runtime.

\paragraph{Vincoli}

Alcuni attributi possono porre vincoli ad altri, come la lunghezza di un array ai valori contenuti nell'array.
Questi vincoli sono utili per intuire quali combinazioni e' inutile / impossibile testare.
I vincoli possono introdurre categorie che rendono superfluo testare piu' di un rappresentante per categoria.