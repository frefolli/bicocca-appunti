\chapter{Esercizi di Testing Combinatorio}

\section{Esercizio 1}

\begin{verbatim}
int deleteIfEqual(String[] theSet, String toBeDeleted);
\end{verbatim}

La funzione \textbf{deleteIfEqual} rimuove gli elementi di un array di al massimo 100 elementi che sono uguali a \textbf{toBeDeleted}.
Se l'array ha piu' di 100 elementi ritorna -100. Se l'array e' vuoto ritorna -1. Altrimenti ritorna il numero di elementi eliminati.

\paragraph{Analisi delle Categorie}

\begin{itemize}
  \item lunghezza dell'array di input
  \item la quantita' di elementi eliminati
\end{itemize}

\paragraph{Obiettivi}

\begin{itemize}
  \item array con lunghezza (0, 1, $1 < x < 100$, 100, 101)
  \item quantita' di elementi eliminati (0, $0 < x \leq 100$)
\end{itemize}

\section{Esercizio 2}

\begin{verbatim}
double registratoreDiCassa(Prodotto[] carrello, PoliticaDiSconto pSconto);
\end{verbatim}

\paragraph{Analisi delle Categorie}

\begin{itemize}
  \item lunghezza del carrello
  \item numero di prodotti di tipo differente
  \item numero di prodotti per i quali sono presenti piu' prodotti dello stesso tipo
  \item numero di prodotti per un certo tipo
  \item politica con sconto senza soglia minima
  \item politica con sconto con soglia minima
  \item valore della soglia minima per lo sconto su un tipo di prodotto
\end{itemize}

\paragraph{Obiettivi}

\section{Esercizio 3}

\paragraph{1}

\begin{itemize}
  \item $t_1$ throws IllegalArgumentException
  \item $t_2$ returns 0
\end{itemize}

\paragraph{2}

\begin{itemize}
  \item $t_1 Rightarrow \frac 1 8\%$
  \item $t_1 \land t_2 Rightarrow \frac 5 8\%$
\end{itemize}

\section{Esercizio 4}

\paragraph{1}

basic condition coverage with $t_1 \land t_2 \land t_3 \land t_4 \Rightarrow \frac {10} {12}$

\section{Esercizio 5}

\paragraph{1}

basic condition coverage = $\frac {11} {14}$

\paragraph{2}

\begin{itemize}
  \item t5 = 3,{"", "", ""},2,0
  \item t6 = 2,{"", "", ""},0,0
  \item t7 = 1,{"", "", ""},2,0
\end{itemize}

\paragraph{3}

NO, copa

\section{Esercizio 6}

\begin{itemize}
  \item 1,0,100
  \item -1,-1,99
\end{itemize}
