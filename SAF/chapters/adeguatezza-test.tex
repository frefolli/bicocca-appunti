\usepackage{makecell}

\chapter{Test Adequacy}

\section{Terminologia}

\begin{center}
	\begin{tabular}{|| c c ||}
		\hline
		termine & descrizione \\
		\hline
		test obligation (= test objective) & \makecell{specifica di cosa \\ deve essere testato} \\
		test case & \makecell{un set di input e condizioni \\ appaiati con un risultato o un comportamento \\ che descrivono il fallimento o \\ il successo del test} \\
		test suite & una collezione di test case \\
		test catalogue & *da capire* \\
		\hline
	\end{tabular}
\end{center}

\section{Test Obligations}

\begin{center}
	\resizebox{\columnwidth}{!}{
	\begin{tabular}{|| c c ||}
		\hline
		tipo & descrizione \\
		\hline
			funzionale & si testa un comportamento della specifica \\
			strutturale & si testa il "comportamento" del codice, ovvero letteralmente i loop e i branch condizionali (= coverage) \\
			basata su modello & *da capire* \\
			basata su difetto & verificano la gestione (e quindi la presenza) di un difetto specifico \\
		\hline
	\end{tabular}}
\end{center}

\paragraph{Boundary Testing}

Selezionare i \textbf{test objectives} che si riferiscono a comportamenti importanti della specifica distinguendo con riguardo tra comportamento normale, eccezioni e casi limite.

\section{Adequacy Criterion}

E' un predicato che si dice soddisfatto o meno per una coppia (programma, \textbf{test suite}).
Quindi e' un insieme di \textbf{test obligations}.
Una test suite si dice che soddisfa un \textbf{adequacy criterion} se per ogni \textbf{test obligation} esiste almeno un \textbf{test case} nella suite che lo soddisfa e se tutti i \textbf{test cases} nella suite passano.

L'adequatezza di un test e' un problema indecidibile, ma ne e' misurabile invece l'inadequatezza, in base alla quale si ricava la lista dei test da modificare o sostituire.

\section{Complementarieta'}

Test funzionali e strutturali sono tra loro complementari, infatti alcune situazioni o alcune \textbf{failures} possono rilevate solo tramire una combinazione di essi.

\section{Testing Sistematico e Randomico}

\paragraph{Random}

Si usando input randomici. Questo permette di evitare il problema del bias del tester: il tester portrebbe subire le stesse sviste del programmatore. Ma non e' detto che manifesti i problemi: la probabilita' (essendo quasi uniforme) e' la stessa che selezioni i valori che non danno problemi, o addirittura molto ridotta nel caso in cui la probabilita' di manifestazione di un difetto sia non uniforme (il caso reale).

\paragraph{Systematic}

Si scelgono gli input che sono considerati come di maggior valore. Permette di controllare dei casi esplicitamente.
Il \textbf{testing funzionale} e' sistematico.

Tendenzialmente le \textbf{failures} sono sparse nello spazio degli input ma probabilmente dense in alcuni punti. Si cerca di trovare questi punti sfruttando la conoscenza pregressa del software.

Successive analisi sui test permettono di stabilire l'inadeguatezza dei test, quindi si ottengono in teoria delle partizioni sempre migliori.
