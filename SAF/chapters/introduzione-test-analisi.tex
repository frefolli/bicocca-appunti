\chapter{Introduzione ai Test e all'Analisi del Software}

\section{Testing Adequacy}

Il metodo naive per testare il software e' chiaramente il *Testing Esaustivo*.
Ovvero si mette il software in azione provando e controllando tutti i possibili input e tutte le possibili condizioni. 
Per adottare questo approccio naturalmente e' necessario che venga stesa una specifica per assicurarsi che il programma funzioni come ci si aspetta.

Mentre in teoria e' l'unico vero metodo per testare (quasi) completamente il codice, resta in pratica un metodo infallibile.
Questo spesso a causa del fatto che o lo spazio di input e' infinito o troppo elevanto, oppure la totalita' degli stati condizionali e' troppo varia.

La correttezza di un sistema quindi e' un problema indecidibile (vedi Halting Problem). Puo' essere ottenuta solo una ottimistica approssimazione di essa.

\section{Bugs}

*Bug* e' una parola utilizzata spesso a sproposito, in quanto e' nei fatti una parola ombrello per diversi tipi di problemi che agiscono su livelli diversi.

\begin{center}
    \begin{tabular}{|| c c ||}
        \hline
        nome & descrizione \\
        \hline
        failure (fallimento) & crash o comportamento scorretto del software \\
        fault (difetto) & problema nel codice, puo' causare una *failure* \\
        error (errore) & ragione o spiegazione del *fault* \\
        \hline
    \end{tabular}
\end{center}

Il *testing* e' l'attivita' che permette di rivelare i *faults* mostrando le *failures*.
Quindi il *testing* fornisce un stima (letteralmente *confidence*) della correttezza.

La percentuale di copertura del codice (*coverage*) pero' non fornisce di per se' motivo di sicurezza.
Infatti e' possibile (e nella pratica e' cosi') che il sottospazio degli input che manifestano *failure* sia molto piccolo rispetto allo spazio degli input.

Non potendo fare infiniti test si e' costretti a selezionare quelli giusti e piu' (anche solo potenzialmente) utili tra quelli possibili.
