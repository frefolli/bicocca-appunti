\chapter{Alberi Evolutivi e Filogenesi}

\section{Evoluzione}

L'Evoluzione e' l'effetto di una sequenza di mutazioni che a lungo termine porta a dei cambiamenti sostanziali e diffusi nella composizione del genoma di una specie.

Le cellule subiscono continue mutazioni durante la loro vita. Alcune fanno parte di un processo intrinseco di mutazione, altreno invece sono provocate da agenti esterni o dall'ambiente in cui ci si ritrova.

La teoria moderna dell'evoluzione e' basata su caratteri che vengono acquisiti una sola volta nell'albero evolutivo.

\section{Algoritmo di Gusfield}

La filogenesi e' il processo di ramificazione delle linee di discendenza nell'evoluzione della vita.

Sia $M$ la matrice $n \cdot n$ che ha come colonne i caratteri e come righe gli organismi La cella $M[i,j]$ indica con un numero $\{0,1\}$ se l'organismo i-esimo ha il carattere j-esimo.

Sia per esempio la seguente matrice:

\begin{center}
    \begin{tabular}{||c c c c c c||}
        \hline
        & A & J & H & L & V \\
        \hline
        Scorpione & 0 & 0 & 0 & 0 & 0 \\
        Anguilla & 0 & 0 & 0 & 0 & 1 \\
        Tonno & 0 & 1 & 0 & 0 & 1 \\
        Salamandra & 0 & 1 & 0 & 1 & 1 \\
        Tartaruga & 1 & 1 & 0 & 1 & 1 \\
        Leopardo & 1 & 1 & 1 & 1 & 1 \\
        \hline
    \end{tabular}
\end{center}

Abbiamo bisogno di un algoritmo che prenda in input una matrice binaria siffatta e restituisca in output l'albero che spiega M, se esiste. Usiamo per esempio l'algoritmo di Gusfield.

Si usa prima il Radix Sort sulle colonne in ordine decrescente (anche del numero di 1), quindi si passa a costruire l'albero una specie alla volta. Quindi si fa:

\begin{itemize}
    \item Considerando ogni colonna binaria come un numero binario si riordinano con il Radix Sort in ordine descrescente in modo da avere il numero piu' grande in colonna 1
    \item Per ogni riga p di $M_o$ si costruisce la stringa dei caratteri, in ordine crescente, che p possiede
    \item Si costruisce l'albero dei suffissi T per le $n$ stringhe cosi' costruite
    \item Si testa se T e' una filogenesi perfetta per M (ovvero se la spiega)
\end{itemize}

\paragraph{Caratteri e Stato}

Un carattere c e' acquisito quando lo stato di c passa da $0$ a $1$ in un arco.
Un carattere c e' perso quando lo statoc di c passa da $1$ a $0$ in un arco. (Mutazione ricorrente)

\paragraph{Modelli di evoluzione}

Ogni carattere c e' acquisito esattamente una volta nell'albero.

\begin{itemize}
    \item Filogenesi perfetta: nessuna mutazione ricorrente, nessuna perdita
    \item Dollo: mutazioni ricorrenti senza limiti, ma senza perdite
    \item Camin-Sokal: perdite senza limiti, ma senza mutazioni ricorrenti
\end{itemize}

\section{Approcci basati su parsimonia}

I problemi di parsimonia si dividono in due categorie: piccola parsimonia (topologia nota) e massima parsimonia (topologia ignota).

Per i problemi di piccola parsimonia si trattano due algoritmi: Fitch e Sankoff.

\paragraph{Istanza del problema}

Input:

\begin{itemize}
    \item Matrice M con n specie in S e m caratteri in C
    \item Albero T le cui foglie corrispondono alle specie di M
    \item Per ogni carattere un costo $w_c$ fra ogni coppia di stati
\end{itemize}

Output:

Per ogni carattere c una etichettatura $\lambda_c $ che assegna ad ogni nodo uno degli stati possibili per c

Funzione obiettivo:

$min \Sigma _ {c \in C} \Sigma _ {(x,y) \in archi} w_c(\lambda_c(x), \lambda_c(y))$

\section{Algoritmo di Sankoff}

Ogni carattere e' gestito separatamente. Si usa la programmazione dinamica.

Sia P un matrice con $P[x,z] = $ soluzione ottimale del sottoalbero T che ha radice x sotto la condizione che abbia etichetta z.

\paragraph{Caso base}

Se x e' una foglia con etichetta z allora $P[x,z] = 0$.

Se x e' una foglia con etichetta diversa da z allora $P[x,z] = +\infty$.

\paragraph{Caso ricorsivo}

Se x e' un nodo interno, detto F(x) = insieme di nodi figli di x allora:

$P[x,z] = \Sigma _ {f \in F(x)} min_s \{ w(z,s) + P[f,s] \}$.

La soluzione ottimale e' $min_s \{P[r,s]\}$, dove r e' la radice di T.

\section{Algoritmo di Fitch}

Solo per il caso non pesato di un albero T binario.

Con $S(x) = $ insieme degli stati ottimali per il nodo x.

Se x e' una foglia allora $S(x) = \lambda_c(x)$.

Se x e' un nodo interno, siano $a$ e $b$ si suoi figli in T.
Se $S(a) \cap S(b) = \emptyset$ allora $S(x) = S(a) \cup S(b)$.
Altrimenti $S(x) = S(a) \cap S(b)$.

\section{Approcci basati su distanza}

Sia $d : S x S \rightarrow N$. tale che:

\begin{align*}
    d(a,b) = 0 &\Leftrightarrow a = b &\forall a,b \in S \\
    d(a,b) &= d(b,a) & \forall a,b \in S \\
    d(a,b) &\leq d(a,c) + d(c,b) &\forall a,b,c \in S
\end{align*}

\section{UPGMA}

UPGMA sta per Unweighted Pair Group with Arithmetic Mean/

\begin{enumerate}
    \item Per ogni specie si crea un cluster $C_i$.
    \item Quindi si definisce la funzione di distanza $D : C x C \rightarrow N$ con $D(C_a,C_b) = \frac {\Sigma _ {i \in C_a} \Sigma _ {j \in C_b} d(i,j)} {|C_a| \cdot |C_b|}$.
    \item All'inizio $h(C_i) = 0 \forall i$.
    \item Quindi si fondono i due cluster $C_a,C_b$ piu' vicini, ovvero tali che $min \{ D(C_a,C_b) \forall a,b \}$.
    \item La fusione di $C_a,C_b$ la chiamo $C_x$.
    \item Per ogni cluster $C_y \ne C_x$, calcolo la distanza $D(C_x,C_y)$.
    \item Quindi $h(C_x) \gets \frac {D(C_a,C_b)} {2}$
    \item $(C_x,C_a)$ e' etichettato con $h(C_x) - h(C_a)$
    \item $(C_x,C_b)$ e' etichettato con $h(C_x) - h(C_b)$
\end{enumerate}

UPGMA produce una ultrametrica. Una ultrametrica e' uno spazio metrico (basato su distanza) in cui la diseguaglianza triangolare e' rinforzata. Ovvero in cui vale anche che $d(a,b) \leq max \{ d(a,c), d(c,b) \} \forall a,b,c \in S$.

\section{Modelli di Evoluzione}

TODO
