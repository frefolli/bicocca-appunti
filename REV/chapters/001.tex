\chapter{Sistemi Legacy}

\section{Situazione}

Un sistema legacy e' un sistema che ha ancora una criticita' dopo moltissimi anni, che possibilmente e' stato un software vincente ma che ora e' in decadenza.
Spesso usa tecnologie o approcci obsoleti che pero' sono ancora molto applicati nel mondo del software.

E' necessario prendere delle decisioni circa il suo futuro: lo cestino, lo riscrivo, lo miglioro ... 
E' spesso molto costoso sia continuare ad operarlo che cambiarlo.
Quindi e' molto importante appoggiarsi a tecniche di reverse engineering per ridurre lo sforzo di riscrittura mantenendo la piena funzionalita' del sistema.

Se la qualita' e il valore del software sono bassi allora il sistema puo' essere cestinato. Se invece l'importanza e' alta allora e' necessario re-ingegnerizzarlo, oppure rimpiazzarlo con un altro sistema esistente che assolve a tutte le funzioni del software dimissionario.


