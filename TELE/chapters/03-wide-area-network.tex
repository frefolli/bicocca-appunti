\chapter{Wireless Area Network}

E' una soluzione di rete dedicata piu' costosa di una generalizzata, utilizzata per collegare entita' di tipo business.
Si chiama \textit{wide} perche' e' generalmente geograficamente estesa.

\putimage{images/03-01.png}{WAN}{png:3-1}

\subsection{Physical WAN}

L'organizzazione che deve far uso di una rete dedicata possiede e gestisce fisicamente un'infrastruttura che collega piu' entita' o sottoentita'.
E' la soluzione piu' costosa in termini di risorse e denaro, ma da cui si ottiene in ritorno una larga banda e buonissime prestazioni.
E' possibile noleggiare pezzi di infrastruttura di ISP che non sono utilizzati (dark fibers).

\putimage{images/03-02.png}{P. WAN}{png:3-2}

\subsection{Leased WAN}

L'organizzazione acquista con un contratto il servizio di circuito al gestore della rete con alcuni tipi di garanzie.
Costa meno ma si ha meno controllo sull'infrastruttura.

\putimage{images/03-03.png}{L. WAN}{png:3-3}

\subsection{Multiprotocol Label Switching WAN}

E' una soluzione meno costosa che permette ad un'organizzazione di acquistare con un contratto un servizio di mesh tra alcune entita'.

\putimage{images/03-04.png}{M.L.S. WAN}{png:3-4}

\section{Multiprotocol Label Switching}

MPLS da la possibilita' di creare delle linee virtuali tra Provider Edge Router per connettere le sottoreti mesh.
Le linee possono essere dinamicamente gestite ed ottimizzate (flessibilita').

\putimage{images/03-05.png}{MPLS}{png:3-5}

\subsection{Label Swapping Forwarding}

Un pacchetto IP viene incapsulato in un pacchetto MPLS, e questo header (\textit{label}) viene utilizzata per fare il routing virtuale all'interno del circuito.
Gli indirizzi MPLS sono locali alla rete MPLS, e i pacchetti possono essere incapsulati piu' volte.

\putimage{images/03-06.png}{MPLS Packet}{png:3-6}

Durante il routing le label vengono sostituite al transito per accomodare il flusso di dati.

\putimage{images/03-07.png}{MPLS Routing}{png:3-7}

Il Label Edge Router aggancia o disaggancia l'header MPLS rilasciando il pacchatto IP nella destinazione.

\putimage{images/03-08.png}{Label Edge Router}{png:3-8}

\subsection{Confronto rispetto a Destination-Oriented}

In IP il forwarding non dipende dalla sorgente del flusso ma solo dalla destinazione, mentre in MPLS il focus e' il circuito dedicato ad una certa sorgente che viaggia per una certa destinazione.

\putimage{images/03-09.png}{IP}{png:3-9}

\putimage{images/03-10.png}{MPLS}{png:3-10}
