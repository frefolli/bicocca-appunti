\chapter{Wireless Area Network}

E' una soluzione di rete dedicata piu' costosa di una generalizzata, utilizzata per collegare entita' di tipo business.
Si chiama \textit{wide} perche' e' generalmente geograficamente estesa.

\putimage{images/03-01.png}{WAN}{png:3-1}

\subsection{Physical WAN}

L'organizzazione che deve far uso di una rete dedicata possiede e gestisce fisicamente un'infrastruttura che collega piu' entita' o sottoentita'.
E' la soluzione piu' costosa in termini di risorse e denaro, ma da cui si ottiene in ritorno una larga banda e buonissime prestazioni.
E' possibile noleggiare pezzi di infrastruttura di ISP che non sono utilizzati (dark fibers).

\putimage{images/03-02.png}{P. WAN}{png:3-2}

\subsection{Leased WAN}

L'organizzazione acquista con un contratto il servizio di circuito al gestore della rete con alcuni tipi di garanzie.
Costa meno ma si ha meno controllo sull'infrastruttura.

\putimage{images/03-03.png}{L. WAN}{png:3-3}

\subsection{Multiprotocol Label Switching WAN}

E' una soluzione meno costosa che permette ad un'organizzazione di acquistare con un contratto un servizio di mesh tra alcune entita'.

\putimage{images/03-04.png}{M.L.S. WAN}{png:3-4}

\section{Multiprotocol Label Switching}

MPLS da la possibilita' di creare delle linee virtuali tra Provider Edge Router per connettere le sottoreti mesh.
Le linee possono essere dinamicamente gestite ed ottimizzate (flessibilita').

\putimage{images/03-05.png}{MPLS}{png:3-5}

\subsection{Label Swapping Forwarding}

Un pacchetto IP viene incapsulato in un pacchetto MPLS, e questo header (\textit{label}) viene utilizzata per fare il routing virtuale all'interno del circuito.
Gli indirizzi MPLS sono locali alla rete MPLS, e i pacchetti possono essere incapsulati piu' volte.

\putimage{images/03-06.png}{MPLS Packet}{png:3-6}

Durante il routing le label vengono sostituite al transito per accomodare il flusso di dati.

\putimage{images/03-07.png}{MPLS Routing}{png:3-7}

Il Label Edge Router aggancia o disaggancia l'header MPLS rilasciando il pacchatto IP nella destinazione.

\putimage{images/03-08.png}{Label Edge Router}{png:3-8}

\subsection{Confronto rispetto a Destination-Oriented}

In IP il forwarding non dipende dalla sorgente del flusso ma solo dalla destinazione, mentre in MPLS il focus e' il circuito dedicato ad una certa sorgente che viaggia per una certa destinazione.

\putimage{images/03-09.png}{IP}{png:3-9}

\putimage{images/03-10.png}{MPLS}{png:3-10}

\subsection{Creazione delle Rotte}

La creazione delle rotte in MPLS puo' avvenire o manualmente o in modo automatico tramite un layer di controllo simile per certi versi a quello del livello di Rete dello stack ISO/OSI.

\putimage{images/03-11.png}{MPLS Control Plane}{png:3-11}

Il routing dei pacchetti di controllo avviene normalmente come in IP.

\subsection{Traffic Engineering Database}

E' un database che contiene informazioni circa la topologia dell'infrastruttura, i dati di carico, le limitazioni di banda ... etc e i dati amministrativi con le configurazioni dei clienti. E' utilizzato in MPLS per stabilire le rotte in maniera statica ("offline") o dinamica ("online").

\putimage{images/03-12.png}{MPLS + TED}{png:3-12}

\subsection{Protocolli di Segnalazione}

Un meccanismo di segnalazione e stabilimento delle rotte e' necessario per coordinare i nodi nella distribuzione delle label, riservare/liberare risorse e prevenire i loop.

\subsubsection{Label Distribution Protocol}

Questo e' un protocollo molto basilare che segue il routing dettato dai protocolli IGP.
Non e' possibile specificare le rotte da seguire (ne' fa uso del TED) e funziona "hop-by-hop" ovvero ogni nodo del percorso agisce autonomamente nella distribuzione delle label.
Ogni nodo invia la richiesta al nodo successivo, il quale risponde con la label da utilizzare per l'instradamento nel percorso "riservato".

\putimage{images/03-13.png}{LDP /1}{png:3-13}
\putimage{images/03-14.png}{LDP /2}{png:3-14}

\subsubsection{Constrained LDP Routing}

In questa variante di LDP abbiamo la possibilita' di specificare esplicitamente le rotte da instaurare nodo per nodo.
Anche qua pero' il funzionamento e' "hop-by-hop".

\putimage{images/03-15.png}{C. LDP R.}{png:3-15}


\subsubsection{Resource Reservation Protocol}

La differenza fondamentale rispetto agli altri due e' che in questo caso non sono i singoli nodi a stabilire le label: Il nodo sorgente invia una richiesta di path (con percorso esplicito) al destinatario, il quale procede lui a calcolare le label da distribuire e poi le assegna tramite una risposta che fluisce fino al mittente.

Un'altra particolarita' di questo protocollo e' che puo' fare uso del TED (in questo caso si parla di RSVP-TE).
In questo modo si puo' distribuire il carico del flusso in modo equo o intelligente lungo la topologia della rete.


\putimage{images/03-16.png}{RSVP}{png:3-16}

Nel caso di esempio IP avrebbe mandato tutti i flussi nel percorso piu' veloce, con evidente overload rispetto alla capacita' del canale (quindi avremmo avuto perdite o rallentamenti). Invece RSVP-TE divide il flusso in base allo stato della rete.

\putimage{images/03-17.png}{RSVP-TE /1}{png:3-17}

RSVP-TE puo' anche decidere di fare \textit{Protection Switching}, ovvero riservare una porzione di rete alle emergenze, in modo da essere pronto a switchare l'assegnamento delle rotte per perdere meno dati possibile. Il problema di questo schema e' che occorre bilanciare la quantita' di risorse riservate nella rete per evitare di avere o un throughput troppo basso oppure una bassa capacita' di recupero (esistono schemi 1-N cosi' come 1-1).

Questo tipo di recovery e' molto piu' veloce rispetto a quello dei protocolli IP perche' coinvolge solo una ristretta cerchia di nodi che devono cambiare una label in modo dinamico, non e' richiesta la riconfigurazione di tutta la rete ...

\putimage{images/03-18.png}{RSVP-TE /2}{png:3-18}
