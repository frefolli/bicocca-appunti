\chapter{Reti Complesse}

\section{Rete Causale}

$N = (B, E, F)$ e' una rete causale sse:
\begin{itemize}
    \item $\forall b \ in B$, $|\cdot b| \leq 1 \land |b \cdot| \leq 1$ (quindi non ci sono conflitti)
    \item $\forall x,y \in B \cup E \; : \; (x, y) \in F^+ \Rightarrow (y, x) \notin F^+$ (quindi non ci sono cicli)
    \item $\forall e \in E$, $\{ x \in B \cup E \; | \; (x, e) \in F \}$ e\ un insieme finito (la rete puo' essere infinita)
\end{itemize}

Ad una rete causale corrisponde un ordine parziale $(X, \leq) = (B \cup E, F^*)$, quindi:

\begin{itemize}
    \item $x,y \in X$
    \item $x \leq y$ $\Leftrightarrow$ x causa y
    \item $x \; li \; y$, $x \leq y \lor y \leq x$, causalmente dipendenti
    \item $x \; co \; y$, $\neg (x < y) \land \neg (y < x)$, causalmente indipendenti
    \item le relazioni $co$, $li$, sono simmetriche, riflessive ma \textbf{non} transitive
\end{itemize}

Definite queste due relazioni si definiscono i sequenti sottoinsiemi:

\begin{itemize}
    \item $C \subseteq X$ e' detto \textbf{co-set} sse $\forall x,y \in C$, $x \; co \; y$
    \item $C$ e' detto \textbf{taglio} sse co-set massimale ($\forall x \in (X \ C)$, $\exists c \in C \; : \; c \; li \; x$)
    \item $C$ e' detto \textbf{B-taglio} se $C \supseteq B$
    \item $L \subseteq X$ e' detto \textbf{li-set} sse $\forall x,y \in L$, $x \; li \; y$
    \item $L$ e' detto \textbf{linea} sse li-set massimale ($\forall x \in (X \ L)$, $\exists l \in L \; : \; l \; co \; x$)
\end{itemize}

\subsection{K-Density}

Una Rete Causale $N$ e' detta \textbf{K-densa} sse $\forall l \in Linee(N), \forall c \in Tagli(N) \; | \; |l \cap c| = 1$.

NB: Se N e' finita allora e' anche densa.

\section{Processo Non Sequenziale}

\section{Processo Ramificato}
