\chapter{CCS e LTS}

\section{Programmazione Sequenziale}

\begin{itemize}
  \item Semantica operazionale: data una Macchina Astratta si descrivono i passi di computazione come funzione $M : I \rightarrow O$.
  \item Semantica denotazionale: dato un Programma, assegno una funzione $f : I \rightarrow O$ (= lambda calcolo).
  \item Semantica assiomatica: specificare le proprieta' di un programma come funzione $P : I \rightarrow O$, dove i dati in input vengono specificati come i valori che gli input possono assumere in termini di condizioni, e vice versa gli output.
\end{itemize}

Caratteristiche della programmazione sequenziale:
\begin{itemize}
  \item Terminazione
  \item Composizionalita': composizione funzionale f,g: $h : D(f) \rightarrow C(g) = f * g$, con $C(f) \equiv D(g)$
\end{itemize}

Notazione `I P O`, ex: $\{x>0\}$ P $\{x<0\}$

\subsection{Con programmi concorrenti}

\begin{itemize}
  \item P1: $x = 2$, allora $\{x = V\}$ P1 $\{x = 2\}$.
  \item P2: $x = 3$, allora $\{x = V\}$ P2 $\{x = 3\}$.
  \item P3: $x += 1$, allora $\{x = V\}$ P3 $\{x = V\}$.
\end{itemize}

Come si nota, in presenza di concorrenza non c'e' determinismo e quindi composizionalita' semplice (come nei programmi sequenziali).

\begin{itemize}
  \item Esecuzione di $P1 \rightarrow P2$. $\{x = V\}$ P1 P2 $\{x = 3\}$.
  \item Se si mettono in parallelo: $\{x = V\}$ P1 \&\& P2 $\{x = 2 \lor x = 3\}$.
  \item Con esecuzione in parallelo di P1,P2,P3: $\{x = V\}$ P2 \&\& P3 $\{x = 2 \lor x = 3 \lor x = 4\}$.
\end{itemize}

\section{Labelled Transition System (LTS)}

Sistema di Transizioni Etichettate scritto $(S, AcT, T)$ dove:
\begin{itemize}
  \item $S$ e' insieme di stati
  \item $Act$ e' insieme di nomi di azioni
  \item $T$ e' la funzione di transizione denotata $T: S \times AcT \rightarrow S$, dove:
    \begin{itemize}
      \item $(s, a, s') \in T$
      \item $s \xrightarrow[]{a} s'$
      \item $(s) \xrightarrow[]{a} (s')$
    \end{itemize}
  \item formalmente, $s \xrightarrow[]{w} s'$ sse:
  \begin{itemize}
    \item o) $w = \epsilon \land s = s'$
    \item o) $w = a*x \land s \xrightarrow[]{a} s'' \xrightarrow[]{x} s'$
  \end{itemize}
\end{itemize}

\section{Calculus of Communicating Systems (CCS)}

\begin{itemize}
  \item Lambda calcolo di processi concorrenti
  \item Communicating Sequential Processes.
\end{itemize}

\paragraph{definizione}
Un set di programmi e' un insieme di processi che vengono composti con l'operatore di esecuzione parallela e tramite la sincronizzazione l'interazione con l'ambiente.

CCS e' un sistema di calcolo interpretato come LTS che viene denotato con la seguente espressione BNF:

$
P ::= 0 \,\,\,
  | \,\,\, a. \, P_{1} \,\,\,
  | \,\,\, A\,\,\,
  | \,\,\, P_{1} + P_{2} \,\,\,
  | \,\,\, P_{1} | P_{2} \,\,\,
  | \,\,\, P_{1} [b/a] \,\,\,
  | \,\,\, P_{1} {\backslash }a \,\,\,
$

Che significa rispettivamente:
\begin{itemize}
  \item il processo inattivo generico $0$
  \item $P$ esegue $a$ e diventa $P_1$
  \item definendo $A \stackrel{def}{\equiv} P_1$ si usa $A$ come identificatore
  \item $P$ diventa $P_1$ **o** $P_2$
  \item $P_1$ e $P_2$ coesistono
  \item a $P_1$ si rinomina l'azione $a \rightarrow b$
  \item a $P_1$ si rimuove l'azione $a$
\end{itemize}

Due processi P1 e P2 sono detti equivalenti non in termini funzionali, ma "all'osservazione", cioe' se un processo esterno non e' in grado di distinguere tra i processi P1 e P2 interagendo con essi. (Liskov dei poveri). La Bisimulazione e' tecnica di verifica per identificare se due processi sono dissimili.

Si denota un sistema CCS come tupla $(K, Act, T)$, dove
\begin{itemize}
  \item $K$, insieme di nomi di Processi
  \item $A$, insieme di nomi di azioni
  \item $\overline A$, insieme di conomi di azioni, (= $\overline a \in \overline A$ e' il reciproco di $a \in A$, es: SYN e ACK)
  \item $Act = A \cup B \cup \{t\}$, con $t \not \in A$
  \item $(A \cup B)$ sono le azioni osservabili
  \item $t$ e' azione di handshake tra due processi (azione che non aspetta nulla indre', ndr)
\end{itemize}

Sia $P$ un processo $P \in K$, ricordando la notazione LTS:
\begin{itemize}
  \item es: $(P = a . \ P)$ := P esegue $a$ e poi si comporta come $P$ (ricorsione)
  \item $(P1) \xrightarrow[]{a} (P2)$ denota sincronizzazione e mutamento
  \item $(P1) \xrightarrow[]{a} (P1)$ denota ricorsione di processo
\end{itemize}

\subsection{${B^1}$}

\begin{itemize}
  \item ${B^1}_0 = in \; . \; {B^1}_1$
  \item ${B^1}_1 = \overline {out} \; . \; {B^1}_0$
\end{itemize}

\putimage{images/1.png}{diagram}{png:1}

\subsection{${B^2}$}

\begin{itemize}
  \item ${B^2}_0 = in \; . \; {B^2}_1$
  \item ${B^2}_1 = \overline {out} \; . \; {B^2}_0 \; + \; in \; . \; {B^2}_2$
  \item ${B^2}_2 = \overline {out} \; . \; {B^2}_1$
\end{itemize}

\putimage{images/2.png}{diagram}{png:2}

\subsection{L'esecuzione parallela ${B^1}_0 | {B^1}_0$ e' bisimile a ${B^2}_0$}

\putimage{images/3.png}{diagram}{png:3}
