\chapter{Reti ad Alto Livello}

Le reti elementari sono belle ma nel momento in cui io dovessi arricchire il protocollo che rappresentano come potrei aggiungere dettagli senza modificare la struttura di una rete?
Semplice: reti piu' complesse!

\section{Reti di Posti e Transizioni}

$\Sigma  = (S, T, F, K, W, M_0)$ si dice sistema di Posti e Transizioni (sistema $P/T$) sse:
\begin{itemize}
    \item (S, T, F) e' una rete
    \item $K : S \rightarrow N^+ \cup \{\infty\}$ e' la funzione capacita' dei posti
    \item $W : F \rightarrow N$ e' la funzione peso degli archi
    \item $M_0 : S \rightarrow N : \forall s \in S \;\; M_0(s) \leq K(s)$ e' la marcatura iniziale
\end{itemize}

\paragraph{regola di scatto}

\begin{itemize}
    \item $M[t$ sse $\forall s \in S \;\; M(s) \geq W(s, t) \land M(s) + W(t, s) \leq K(s)$
    \item $M[t > M'$ sse $M'(s) = M(s) - W(s, t) + W(t, s)$
\end{itemize}

L'insieme delle marcature raggiungibili di $\Sigma$ e' $[M_0$:
\begin{itemize}
    \item $M_0 \in [M_0 >$
    \item se $M \in [M_0 > \land \exists t \in T : M[t > M' allora M' \in [M_0 >$
\end{itemize}

Valgono tutte le considerazioni delle Reti di petri su conflitto, concorrenza e indipendenza.

\paragraph{Grafo delle marcature raggiungibili}

Sia $\Sigma = (S, T, F, K, W, M_0)$ un sistema $P/T$. $RG(\Sigma) = ([M_0>, U_\Sigma, A, M_0)$ e' il suo grafo dove: \\

$A = \{ (M, U, M') \; : \; M,M' \in [M_0> \land U \in U_\Sigma \land M[U>M' \}$.

Se $U$ e' una singola transizione (insieme singoletto), allora si ha il grafo di raggiungibilita' sequenziale $SRG(\Sigma)$.

Tensio': La \textbf{Diamond Property} non e' piu' valida in generale perche' qui si ammettono i self loop.

$\Sigma$ e' detto limitato sse $\exists n \in N$, tale che $\forall s \in S$, $\forall M \in [M_0>$, $M(s) \leq N$.
$\Sigma$ e' limitato $\Leftrightarrow$ $[M_0>$ e' un insieme finito (il grafo delle marcature e' finito).

\putimagebig{images/12.png}{example}{png:12}

\section{Contatti}

Un sistema $P/T$ $\Sigma = (S, T, F, K, W, M_0)$ e' detto \textbf{senza contatti} sse:
\begin{itemize}
  \item $\forall M \in [M_0>$, $\forall t \in T$, $\forall s \in S$
  \item $M(s) \leq W(s, t) \Rightarrow M(s) + W(t, s) \leq K(s)$
\end{itemize}

Per rendere un sistema senza contatti si puo' complementare i posti come nel caso delle reti di petri.
La conseguenza di questa condizione e' che l'abilitazione di una transizione $t \in T$ in $M \in [M_0>$ non dipende piu' dalla capacita' dei posti, ma solo dalla marcatura del posto e dal peso degli archi.

\section{Reti Marcate}

Un sistema $P/T$ $\Sigma = (S, T, F, K, W, M_0)$ e' detto \textbf{rete marcata} sse:

\begin{itemize}
  \item $\forall s \in S$, $M_0(s) \in N \land K(s) = \infty$
  \item $\forall t \in T$, $W(s, t) \leq 1 \land W(t, s) \leq 1$
\end{itemize}

Ovvero nel computo dell'abilitazione di una transizione, il peso degli archi e la capacita' dei posti sono ridondanti ovvero ininfluenti.

\paragraph{Sicurezza}

Una Rete Marcata e' detta sicura ss: $\forall M \in [M_0>$, $\forall s \in S$, $M(s) \leq 1$.

Sorpresa delle sorprese, una rete marcata \textbf{sicura} e \textbf{pura} e' anche una rete di petri \textbf{pura}.

\section{Matrice di Incidenza}

Sia $\Sigma = (S, T, F, K, W, M_0)$ un sistema P/T tale che $\forall s \in S$, $K(s) = \infty$ e $N = (S, T, F)$ sia una pura.

Il sistema puo' essere rappresentato con la \textbf{matrice di incidenza} $\underline N \; : \; S \times T \rightarrow N$
