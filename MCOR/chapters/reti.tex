\chapter{Reti}

\section{Rete}

$N = (B, E, F)$ \`e una rete sse:
\begin{itemize}
  \item $B$ \`e l'insieme di condizioni (stati locali)
  \item $E$ \`e l'insieme di eventi
  \item tali che $B \cap E = \emptyset$ e $B \cup E \neq \emptyset$
  \item $F \subseteq (B \times E) \cup (E \times B)$ relazione di flusso rappresenta da $\rightarrow$
\end{itemize}

La relazione di flusso definisce pre-elementi ($\bullet x$) e post-elementi ($x \bullet$). \\

Un \textbf{caso} \`e un'insieme di condizioni $c \subseteq B$ che rappresenta lo stato globale del sistema.

\paragraph{regola di scatto}

L'evento $e$ \`e abilitato in $c$ ($c[e>$) sse: $\bullet e \subseteq c$ and $e \bullet \cap c = \emptyset$.
Quando $e$ scatta (sse \`e abilitato), si genera un caso $c' = (c - \bullet e) \cup e \bullet$. \\

Una rete si dice \textbf{semplice} se $\bullet x = \bullet y \land x \bullet = y \bullet \Rightarrow x = y$.\\

Una rete si dice \textbf{pura} se $\forall e \in E : \bullet e \cap e \bullet = \emptyset$.

Sia $U \subseteq E$
\begin{itemize}
  \item due eventi $e_1,e_2 \in U$ sono indipendenti se $(\bullet e_1 \cup e_1 \bullet) \cap (\bullet e_2 \cup e_2 \bullet) = \emptyset$.
  \item $U$ \`e un insieme di eventi indipendenti se tutti gli eventi sono a due a due indipendenti
  \item $U$ \`e un passo abilitato se $\forall e \in U$, $c[e>$.
  \item $U$ \`e un passo da $c_1$ a $c_2$ ($c_1[U>c_2$) sse: $c_2 = (c_1 - \bullet U) \cup U \bullet$
\end{itemize}

\section{Sistema Elementare}

E' detto sistema elementare $\Sigma = (B, E, F, c_{in})$ sse:
\begin{itemize}
  \item $N = (B, E, F)$ \`e una rete e il caso iniziale \`e $c_{in} \subseteq B$.
\end{itemize}

L'insieme dei casi raggiungibili del sistema elementare \`e definito dal piu' piccolo sottoinsieme $C_\Sigma \subseteq 2^B$ tale che:
\begin{itemize}
  \item $c_{in} \in C_\Sigma$
  \item detti $c \in C_\Sigma, U \subseteq E, c' \subseteq B$, se $c[U>c'$ allora $c' \in C_\Sigma$
\end{itemize}

Del resto, \`e detto insieme dei passi il corrispondente insieme $U_\Sigma \subseteq 2^E$ tale che \\ $\forall U \subseteq E \; | \; \exists c,c' \; : \; c[U>c'$.

Di conseguenza il grafo dei casi sequenziale \`e il sistema a transizioni etichettate LTS $CG_\Sigma = (C_\Sigma, U_\Sigma, A, c_{in})$ dove:
\begin{itemize}
  \item $C_\Sigma$ \`e l'insieme dei nodi del grafo
  \item $U_\Sigma$ \`e l'alfabeto
  \item $A$ \`e l'insieme degli archi etichettati con le parole $p \in U$
  \item $A = \{ (c, U, c') | c,c' \in C_\Sigma, U \in U_\Sigma, c[U>c'\}$
\end{itemize}

\section{Diamond Property}

\paragraph{Anche detto "Teorema dei Due Passi"}

In quel tempo il figlio dell'uomo si stava dilettando con i sistemi elementari, al che ne prese uno, lo spezzo', ne diede un po' ai suoi discepoli e disse:

\begin{itemize}
  \item Sia un sistema elementare $\Sigma = (B, E, F, c_{in})$
  \item Sia $U \subseteq E$ un sottoinsieme siano $U_1,U_2$ una sua partizione ($U_1,U_2 \neq \emptyset$, $U_1 \cup U_2 = U \land U_1 \cap U_2 = \emptyset$)
  \item Siano $C,D \subseteq B$ due casi in $\Sigma$
\end{itemize}

Se $C[U>D$ allora esiste un caso intermedio $E \subseteq B$ tale che siano anche $C[U_1>E$ e $E[U_2>D$. \\
Vice versa se $C[U_1>E$ e $E[U_2>D$, allora $C[U>D$.

\section{E le sue consequenze}

Per la \textbf{Diamond Property} il \textbf{Grafo dei Casi Sequenziale} e il relativo \textbf{Sistema Elementare} sono tra solo semanticamente equivalenti e percio' si possono ricavare l'uno dall'altro.

Dato questo principio base, \`e immediato che l'equivalenza di due Sistemi Elementari puo' essere data dall'isomorfismo tra i due relativi Grafi dei Casi Sequenziali.

Tale isomorfismo tra $S_0$ e $S_1$ , $I = (a, b)$, \`e composto da due mappature biunivoche tali che:
\begin{itemize}
  \item $a: S_0 \rightarrow S_1$
  \item $b : E_0 \rightarrow E_1$
  \item $a(s^0_0) = s^1_0$
  \item $\forall s,s' \in S; \forall e \in E; (s, e, s') \in T_0 \Leftrightarrow (a(s), b(e), a(s')) \in T_1$
\end{itemize}

\section{Sono state un disastro per la razza umana}

Dato un sistema etichettato $A = (S, E, T, s_0)$, esiste un sistema elementare $\Sigma = (B, E, F, c_0)$ tale che il grafo dei casi $CG_\Sigma$ sia isomorfo ad $A$ (e in tal costruire $\Sigma$)?

Bella domanda, comunque \`e stato risolto dalla \textbf{Teoria delle Regioni}.

\section{Contatto}

Sia un sistema elementare $\Sigma = (B, E, F, c_0)$, si definisce contatto un caso $c \subseteq B$ tale che: \\
$\bullet e \subseteq c \land c \cap e \bullet \neq \emptyset$.

Cio\`e un evento potrebbe anche scattare se non fosse che alcune post-condizioni non sono inattive.

Un sistema \textbf{senza contatti} \`e un sistema in cui $\forall e \in E, c \in C_\Sigma \; | \; \bullet e \subseteq c \rightarrow c \cap e \bullet = \emptyset$.

E' possibile trasformare un sistema elementare \textbf{con contatti} in uno \textbf{senza contatti} aggiungendo il complementare di ogni condizione facendo rimanere isomorfo il grafo dei casi sequenziale.
Ovvero: $\forall b \in B\; \exists b' \in B' \supseteq B \; \forall e \in b \bullet \rightarrow b' \in \bullet e$.

\putimage{images/6.png}{example}{png:6}

\section{Dipendenza}

Due eventi $e_1,e_2$ si dicono sequenziali sse $c[e_1> \land \neg c[e_2> \land c[e_1e_2 >$, ovvero se $c[e_1 > c[e_2 >$, ovvero se c'\`e dipendenza causale fra i due.

Vice versa due eventi si dicono concorrenti se $c[e_1> \land c[e_2> \land (\bullet e_1 \cup e_1 \bullet) \cap (\bullet e_2 \cup e_2 \bullet) = \emptyset$.
Ovvero se sono entrambi abilitati ed indipendenti.

\section{Conflitto}

Due eventi si dicono in \textbf{conflitto} sse $c[e_1> \land c[e_2> \land \bullet e_1 \cap e_2 \bullet \neq \emptyset$.

Importante notare che in molti casi non \`e possibile stabilire se si \`e risolto un conflitto o no. C'\`e molta confusione visto che il sistema \`e visto dall'alto senza vedere i passi intermedi che compongono il passaggio da un caso all'altro.

\section{Genesi}

Data una rete $N = (B, E, F)$ \`e possibile generare delle sottoreti.

Sia $N' = (B', E', F')$ una rete.

Essa si dice genericamente sottorete di N sse:
\begin{itemize}
  \item $B' \subseteq B$
  \item $E' \subseteq E$
  \item $F' = F \cap (B' x E') \cup (E' x B')$
\end{itemize}

Quindi N' si dice sottorete generata da B sse:
\begin{itemize}
  \item $B' \subseteq B$
  \item $E' = \bullet B \cup B \bullet$
  \item $F' = F \cap (B' x E') \cup (E' x B')$
\end{itemize}

Similarmente N' si dice sottorete generata da E sse:
\begin{itemize}
  \item $B' = \bullet E \cup E \bullet$
  \item $E' \subseteq E$
  \item $F' = F \cap (B' x E') \cup (E' x B')$
\end{itemize}

\section{Composizione}

Non ho la minima idea del se l'abbiano fatto o meno, comunque giusto per completezza ...

Due Reti sono componibili in tre modi:
\begin{itemize}
  \item sincrono
  \item asincrono
  \item misto
\end{itemize}

\putimagebig{images/7.png}{sincrono}{png:7}
\putimagebig{images/8.png}{asincrono}{png:8}
