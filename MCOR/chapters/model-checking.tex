\chapter{Model Checking}

\section{Reticoli}

Un \textbf{reticolo} \`e un insieme ordinato parzialmente ordinato $R = (L, \leq)$ tale che $\forall x,y \in L$ esistono $x \land y$ e $x \lor y$.
Un reticolo si dice completo se $\lor B$ e $\land B$ esistono per ogni $B \subseteq L$.

\section{Monotonia}

Siano $X, Y$ due insiemi parzialmente ordinati, data una funzione $f : X \rightarrow Y$, essa si dice monotona sse $\forall a, b \in X$, $a \geq b \Rightarrow f(a) \geq f(b)$ e $a \leq b \Rightarrow f(a) \leq f(b)$.

\section{Punti Fissi}

Data una funzione monadica $f : X \rightarrow X$, $x \in X$ \`e un punto fisso sse $f(x) = x$.

\section{Teorema di Knaster Tarski}

Sia $R = (L, \leq)$ un reticolo completo, e $f : L \rightarrow L$ una funzione monotona. Allora $f$ ha un minimo e un massimo punto fisso.

\section{Teorema di Kleene}

Sia una funzione $f : 2^A \rightarrow 2^A$, essa si dice continua se $f(\cup X_i) = \cup F(X_i)$.

Allora posso trovare:

\begin{itemize}
  \item il massimo punto fisso calcolando $f(A), f(f(A)), f(f(f(A))) ...$
  \item il minimo punto fisso calcolando $f(\emptyset), f(f(\emptyset)), f(f(f(\emptyset))) ...$
\end{itemize}

\section{Algoritmo per CTL}

\putimagebig{images/15.png}{Esempio}{png:15}

Siano $M = (Q, T, I)$ un modello di Kripke e $\alpha$ una formula. Si definisce $[[\alpha]]$ l'estensione di $alpha$, dove $[[\alpha]] = \{q \in Q \; | \; M,q \models \alpha\}$.

Ora sia $\alpha \equiv AF\beta$. A questa formula si associa una funzione $f_\alpha : 2^Q \rightarrow 2^Q$ tale che, per ogni $H \subseteq Q$:

\begin{itemize}
  \item $f_\alpha(H) = [[\beta]] \cup \{p \in Q \; | \; \forall (p, q) \in T \; q \in H\}$
  \item ovvero ad ogni passo seleziono solo gli stati tali per cui posso andare solo in H e "li aggiungo"
  \item $f_a(\emptyset) = \beta$ trivialmente
  \item Il minimo punto fisso di $f_a$ \`e $[[\alpha]]$
\end{itemize}

Vice versa sia ora $\alpha \equiv EG\beta$. A questa formula si associa una funzione $g_\alpha : 2^Q \rightarrow 2^Q$ tale che, per ogni $H \subseteq Q$:

\begin{itemize}
  \item $g_\alpha(H) = [[\beta]] \cap \{p \in Q \; | \; \exists (p, q) \in T \; q \in H\}$
  \item ovvero ad ogni passo seleziono solo gli stati tali per cui posso andare in H e "rimuovo" gli altri
  \item $g_a(Q) = \beta$ trivialmente
  \item Il massimo punto fisso di $g_a$ \`e $[[\alpha]]$
\end{itemize}

\section{Automi di Buchi}

Gli automi di Buchi sono automi che riconoscono parole infinite su un alfabeto finito $\Sigma$.
Invece che avere stati finali abbiamo stati "accettanti", e il criterio di accettazione per questo genere di automi e' di conseguenza: "l'automa accetta se durante la computazione passa infinite volte per almeno uno degli stati accettanti".

\section{Algoritmo per LTL}

Per calcolare la validit\`a di una formula LTL su un Modello di Kripke dobbiamo usare il seguente metodo.

\begin{itemize}
  \item Si costruisce l'automa di Buchi $B_{\neg \alpha}$
  \item Si trasforma M in automa etichettato da insiemi di proposizioni atomiche (ogni arco entrante ha come label le proposizioni atomiche di uno stato)
  \item Si calcola il prodotto sincrono dei due automi $PS$
  \item Si verifica se $L(PS) = \emptyset$ (decidibile), allora $M,q_0 \models \alpha$
\end{itemize}

Questa cosa "funziona" solo perch\`e con la composizione completamente sincrona si fa in modo che i due automi eseguano le stesse istruzioni somehow. Non l'ho capita allora ma non l'ha nemmeno spiegato in realt\`a, l'ha dato per scontato.

\section{Fairness}

Un ultimo concetto da presentare \`e la nozione di Fairness su sistemi elementari. Un'esecuzione si dice \textbf{unfair} se un evento \`e abilitato ma non scatta mai. Questa eventualit\`a esiste (e per certi versi rappresenta i limiti espressivi di un modello sequenziale, ma sono parole in libert\`a), per questa ragione si vuole limitare la valutazione di una formula alle esecuzioni \textbf{fair}.

Esistono due tipi di fairness:

\begin{itemize}
  \item Fairness debole: $\forall i : (FGen_i) \Rightarrow (F(ex_i))$
  \item Fairness forte: $\forall i : (GFen_i) \Rightarrow (GF(ex_i))$ 
\end{itemize}
