\chapter{Processi}

\section{Rete Causale}

$N = (B, E, F)$ \`e una rete causale sse:
\begin{itemize}
    \item $\forall b \in B$, $|\bullet b| \leq 1 \land |b \bullet| \leq 1$ (quindi non ci sono conflitti)
    \item $\forall x,y \in B \cup E \; : \; (x, y) \in F^+ \Rightarrow (y, x) \notin F^+$ (quindi non ci sono cicli)
    \item $\forall e \in E$, $\{ x \in B \cup E \; | \; (x, e) \in F \}$ e\ un insieme finito (la rete puo' essere infinita)
\end{itemize}

Ad una rete causale corrisponde un ordine parziale $(X, \leq) = (B \cup E, F^*)$, quindi:

\begin{itemize}
    \item $x,y \in X$
    \item $x \leq y$ $\Leftrightarrow$ x causa y
    \item $x \; li \; y$, $x \leq y \lor y \leq x$, causalmente dipendenti
    \item $x \; co \; y$, $\neg (x < y) \land \neg (y < x)$, causalmente indipendenti
    \item le relazioni $co$, $li$, sono simmetriche, riflessive ma \textbf{non} transitive
\end{itemize}

Definite queste due relazioni si definiscono i sequenti sottoinsiemi:

\begin{itemize}
    \item $C \subseteq X$ \`e detto \textbf{co-set} sse $\forall x,y \in C$, $x \; co \; y$
    \item $C$ \`e detto \textbf{taglio} sse co-set massimale ($\forall x \in (X \ C)$, $\exists c \in C \; : \; c \; li \; x$)
    \item $C$ \`e detto \textbf{B-taglio} se $C \subseteq B$
    \item $L \subseteq X$ \`e detto \textbf{li-set} sse $\forall x,y \in L$, $x \; li \; y$
    \item $L$ \`e detto \textbf{linea} sse li-set massimale ($\forall x \in (X \ L)$, $\exists l \in L \; : \; l \; co \; x$)
\end{itemize}

\subsection{K-Density}

Una Rete Causale $N$ \`e detta \textbf{K-densa} sse $\forall l \in Linee(N), \forall c \in Tagli(N) \; | \; |l \cap c| = 1$.

NB: Se N \`e finita allora \`e anche densa.

\section{Intermezzo}

Questo esempio di Sistema Elementare $\Sigma$ verra' usato per illustrare i Processi non Sequenziali e Ramificati.

\putimagebig{images/11.png}{example}{png:11}

\section{Processo Non Sequenziale}

Dato un sistema elementare senza contatti e finito $\Sigma = (S, T, F, c_{in})$, sia $N = (B, E, F)$ una \textbf{rete causale}. \\

$P = <N, \phi : B \cup E \rightarrow S \cup T>$ \`e detto \textbf{Processo non Sequenziale} sse:

\begin{itemize}
    \item $\phi(B) \subseteq S, \phi(E) \subseteq T$
    \item $\forall x,y \in B \cup E \; : \; \phi(x) = \phi(y) \Rightarrow x \; li \; y$
    \item $\forall e \ in E \; : \; \phi(\bullet e) = \bullet \phi(e) \land  \phi(e \bullet) = \phi(e) \bullet$
    \item $Min(N) = \{ x \in B \cup E \} \; | \; \nexists y \; : \; (y, x) \in F \}$
    \item $Min(N)$ de facto \`e il set dei minimali della rete causale
    \item ecco, $\phi(Min(N)) = c_{in}$, (con un po' di immaginazione il concetto \`e quello)
\end{itemize}

\putimagebig{images/9.png}{example}{png:9}

\section{K-Density 2.0}

Sia $<N , \phi>$ un processo non sequenziale di un sistema elementare $\Sigma$ finito e senza contatti;

\paragraph{Teorema}

$N = (B, E, F)$ \`e K-Densa, e $\forall K \subseteq B$, $K$ \`e B-Taglio di $N$ tale che K \`e finito e $\exists c \in C_\Sigma \; : \; \phi(K) = c$.

\section{Rete di Occorrenze}

$N = (B, E, F)$ \`e detta \textbf{Rete di Occorrenze} sse:

\begin{itemize}
    \item $\forall b \in B$, $|\bullet b| \leq 1$ (due eventi non hanno una stessa post-condizione, conflitti solo in avanti)
    \item $\forall x,y \in B \cup E \; : \; (x, y) \in F^+ \Rightarrow (y, x) \notin F^+$ (quindi non ci sono cicli)
    \item $\forall e \in E$, $\{ x \in B \cup E \; | \; (x, e) \in F \}$ e\ un insieme finito (la rete puo' essere infinita)
\end{itemize}

L'ordine parziale $(X, \leq) = (B \cup E, F^*)$ persiste.

\section{Conflitti}

Sia definita la relazione $\# : B \cup E \rightarrow B \cup E$ simmentrica ma non riflessiva n\`e transitiva:

\begin{itemize}
    \item $x \# y$, sse $\exists e_1,e_2 \in E \; : \; \bullet e_1 \land \bullet e_2 \neq \emptyset \land e_1 \leq x \land e_2 \leq y$.
    \item Ovvero, se esistono due eventi con precondizioni in comune tali che $x$ e $y$ dipendono da essi
\end{itemize}

\section{Processo Ramificato}

Dato un sistema elementare senza contatti e finito $\Sigma = (S, T, F, c_{in})$, sia $N = (B, E, F)$ una \textbf{rete di occorrenze}. \\

$P = <N, \phi : B \cup E \rightarrow S \cup T>$ \`e detto \textbf{Processo Ramificato} sse:

\begin{itemize}
    \item $\phi(B) \subseteq S, \phi(E) \subseteq T$
    \item $\forall x,y \in E \; : \; \phi(x) = \phi(y) \land \bullet x = \bullet y \Rightarrow x \; = \; y$
    \item $\phi$ \`e una biiezione su $\phi(\bullet e) \Leftrightarrow \bullet \phi(e)$ 
    \item $\phi$ \`e una biiezione su $\phi(e \bullet) \Leftrightarrow \phi(e) \bullet$ 
    \item $\phi$ \`e una biiezione su $\phi(Min(N)) \Leftrightarrow c_{in}$ 
\end{itemize}

\putimagebig{images/10.png}{example}{png:10}

\section{Prefissi}

Dato un sistema elementare $\Sigma$ e due suoi processi ramificati $P_1 = (N_1, \phi_1), P_2 = (N_2, \phi_2)$, $P_1$ si dice prefisso di $P_2$ sse:

$N_1$ \`e sottorete di $N_2$ e $\phi_{2|N_1} = \phi_1$

\section{Unfolding}

Un sistema elementare $\Sigma$ ammette un solo massimale rispetto alla relazione di prefisso tra processi ramificati. Questo processo \`e chiamato \textbf{unfolding} ($= Unf(\Sigma)$).
Ogni processo non sequenziale di $\Sigma$ \`e un prefisso di $Unf(\Sigma)$.

Da prendere con le pinze, ma a quanto pare un processo sequenziale \`e un processo ramificato tale che la rete sia una rete causale (senza conflitti).
Boh si chiama anche corsa (run).
