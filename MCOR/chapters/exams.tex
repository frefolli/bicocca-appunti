\chapter{Esami}

\section{Indicazioni sullo Scritto}

\begin{itemize}
  \item fare attenzione alle conseguenze logice nella dimostrazione di correttezza
  \item fare attenzione ad indicare le derivazioni corrette e a partire dalla post-condizione per ricavare la precondizione (mai vice versa, perch\`e sarebbe cannato)
  \item scrivere l'esercizio sulla bisimulazione solo dopo aver identificato la strategia vincente per attaccante o difensore
  \item se usi la notazione sbagliata per qualcosa "fa niente" ma te lo fa notare lo stesso
  \item la dimostrazione di correttezza e la bisimulazione sono molto importanti a livello di punteggio, e personalmente da molto valore a queste
  \item assicurarsi di aver considerato tutte le possibili mosse per il difensore prima di fare qualcosa
  \item al $90\%$ i sistemi CCS non sono bisimili, ma pu\`o capitare che lo siano. Ad ogni modo a volte nello scritto ti dice direttamente che non lo sono o vice versa.
\end{itemize}

\section{Indicazioni sull'Orale}

\begin{itemize}
  \item mi ha chiesto rispettivamente
  \item \begin{itemize}
    \item Cos'\`e una dimostrazione nella logica di Hoare, di fare un esempio di derivazione e cose cos\'i
    \item Di spiegare come funziona il gioco della bisimulazione (proprio in parole molto semplici e astratte, cio\`e l'attaccante fa una mossa e il difensore risponde, le regole di base e chi vince e quando vince)
    \item Quando due eventi si dicono indipendenti e che tipo di informazioni ne ricaviamo (passo abilitato, diamond property per il grafo dei casi)
    \item Cos\`e $CTL^\ast$ e perch\`e non si usa
  \end{itemize}
\end{itemize}

In generale ho visto che \`e abbastanza buono con l'orale, non tanto per i voti quanto per il fatto che prova ad ogni costo a farti ragionare e cerca di aiutarti a capire dove hai sbagliato spiegandoti per filo e per segno.

\section{Esercizi da portare all'orale}

\subsection{Dimostrazione del Teorema di Knaster Tarski per il massimo punto fisso}

\begin{itemize}
    \item Costruiamo l’insieme $Z = \{ T \subseteq A \; | \; T \subseteq f(T) \}$.
    \item Poniamo $M = \cup Z$.
    \item $\forall T \in Z \; | \; T \subseteq M$, Quindi $T \subseteq f(T) \subseteq f(M)$.
    \item $\forall T \in Z \; | \; T \subseteq f(M)$, Quindi $\cup Z = M \subseteq f(M)$.
    \item Quindi $M \in Z$.
    \item Visto che $f$ \'e monotona: $M \subseteq f(M) \; \rightarrow \; f(M) \subseteq f(f(M))$.
    \item Il che significa che $f(M) \in Z$, ovvero che $f(M) \subseteq M$.
    \item Quindi $M$ \`e il massimo punto fisso del Reticolo su $f$.
\end{itemize}

\subsection{Formula $f$ di espansione della formula $c \equiv AU(\alpha, \beta)$}

\begin{itemize}
    \item $c \equiv A \; \alpha \; U \; \beta$
    \item $f(H) = [[b]] \cup (\{q \in Q \; | \; \forall(q,p) \in T \; | \; p \in H\} \cap [[a]])$
    \item $f(\emptyset) = [[b]]$
    \item $min \; f \equiv H \subseteq ([[a]] \cup [[b]])$
\end{itemize}
