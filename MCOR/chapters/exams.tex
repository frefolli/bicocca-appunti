\chapter{Esami}

\section{Indicazioni sullo Scritto}

\section{Indicazioni sull'Orale}

\section{Esercizi da portare all'orale}

\subsection{Dimostrazione del Teorema di Knaster Tarski per il massimo punto fisso}

\begin{itemize}
    \item Costruiamo l’insieme $Z = \{ T \subseteq A \; | \; T \subseteq f(T) \}$.
    \item Poniamo $M = \cup Z$.
    \item $\forall T \in Z \; | \; T \subseteq M$, Quindi $T \subseteq f(T) \subseteq f(M)$.
    \item $\forall T \in Z \; | \; T \subseteq f(M)$, Quindi $\cup Z = M \subseteq f(M)$.
    \item Quindi $M \in Z$.
    \item Visto che $f$ \'e monotona: $M \subseteq f(M) \; \rightarrow \; f(M) \subseteq f(f(M))$.
    \item Il che significa che $f(M) \in Z$, ovvero che $f(M) \subseteq M$.
    \item Quindi $M$ \`e il massimo punto fisso del Reticolo su $f$.
\end{itemize}

\subsection{Formula $f$ di espansione della formula $c \equiv AU(\alpha, \beta)$}

\begin{itemize}
    \item $c \equiv A \; \alpha \; U \; \beta$
    \item $f(H) = [[b]] \cup (\{q \in Q \; | \; \forall(q,p) \in T \; | \; p \in H\} \cap [[a]])$
    \item $f(\emptyset) = [[b]]$
    \item $min \; f \equiv H \subseteq ([[a]] \cup [[b]])$
\end{itemize}
