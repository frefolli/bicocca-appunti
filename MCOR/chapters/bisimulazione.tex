\chapter{Bisimulazione}

\begin{itemize}
  \item $S = (P_1 | B_1 | C_1) \setminus \{dep, est\}$
  \item $P_1 = prod \; . \; P_2$
  \item $P_2 = \overline {dep} \; . \; P_1$
  \item $C_1 = estr \; . \; C_2$
  \item $C_2 = cons \; . \; C_1$
  \item $B_1 = dep \; . \; B_2$
  \item $B_2 = \overline {estr} \; . \; B_1$
\end{itemize}

$
(P_1 | B_1 | C_1)
\xrightarrow[]{prod} (P_2 | B_1 | C_1)
\xrightarrow[]{\overline {dep} / dep} (P_1 | B_2 | C_1)
\xrightarrow[]{prod} (P_2 | B_2 | C_1)
\xrightarrow[]{\overline {estr} / estr} (P_1 | B_1 | C_2)
\xrightarrow[]{prod} (P_2 | B_1 | C_2)
\xrightarrow[]{cons} (P_2 | B_1 | C_1)
$

\section{Nozione di equivalenza}

Nel caso di programmi sequenziali \`e sufficiente una congruenza funzionale $H: L(I, O) \rightarrow L(I, O)$.
Invece nei programmi CCS non sequenziali, si deve tener conto dell'interazione dei moduli con l'environment.
Serve una relazione di equivalenza $R : C \rightarrow C$ che permetta di valutare l'interscambiabilita' di moduli.

\subsection{$S_1 \sim S_2$, $S_1$ isomorfo a $S_2$}

Detto ${LTS}_1 = (Q_1, E_1, T_1, q_{0_1})$ e ${LTS}_2 = (Q_2, E_2, T_2, q_{0_2})$, allora

\begin{itemize}
  \item $\alpha : Q_1 \rightarrow Q_2$
  \item $\beta : E_1 \rightarrow E_2$
  \item $\alpha(q_{0_1}) = q_{0_2}$
\end{itemize}

$(q_1,a,q_1) \in T_1 \rightarrow (\alpha(q_1),\beta(a),\alpha(q_1)) \in T_2$

\begin{itemize}
  \item si dice traccia una sequenza $s \in Act^\ast$ di azioni percorribili dalla radice (ovvero dal processo considerato).
  \item l'insieme delle tracce di un processo \`e composto da tutte le possibili sequenze di azioni e di sincronizzazioni a partire da esso.
\end{itemize}

Due processi $P_1$ e $P_2$ sse $Tracce(P_1) \equiv Tracce(P_1)$

\section{CCS di un Dispensatore di Bevande}

$Uni = (M | LP) \setminus \{coin, \; caffe\}$

\begin{itemize}
  \item $M_1 = \overline {coin} \; . \; (caffe \; . \; M_1 + \overline {coin} \; . \; tea \; . \; M_1)$
  \item $M_2 = \overline {coin} \; . \; caffe \; . \; M_2 + \overline {coin} \; . \; \overline {coin} \; . \; tea \; . \; M_2$
\end{itemize}

\subsection{Schema di $M_1$}

\putimage{images/4.png}{diagram}{png:4}

\subsection{Schema di $M_2$}

\putimage{images/5.png}{diagram}{png:5}

\subsection{Comparazione}

Il sistema di transizioni di $M_1$ e $M_2$ non sono isomorfi, ovviamente.

tracce di $M_1$:
\begin{itemize}
  \item coin caffe
  \item coin coin tea
\end{itemize}

tracce di $M_2$
\begin{itemize}
  \item coin caffe
  \item coin coin tea
\end{itemize}

Le tracce sono uguali, ma il comportamento di queste CCS \`e potenzialmente differente perch\`e non c'\`e isomorfismo. Si dicono equivalenti perch\`e eseguono le stesse sequenze, ma non c'\`e garanzia dell'effettiva esecuzione di $M_2$ perch\`e \`e non deterministica (i.e. \`e concorrenti sull'azione $\overline {coin}$).

\section{Equivalenza ad Osservazione}

$M_1$ e $M_2$ sono intercambiabili sse un osservatore esterno non \`e in grado di distinguere i due attori distinti.

Transizioni:
\begin{itemize}
  \item $p \xRightarrow[]{a} p' \equiv p \xrightarrow[]{t^\ast} q  \xrightarrow[]{a} q  \xrightarrow[]{t^\ast} p'$ : transizione debole, si puo' usare delle \textbf{t-mosse} per percorrere l'azione a
  \item $p \xrightarrow[]{a} p'$ : transizione forte, non si possono usare delle \textbf{t-mosse} per percorrere l'azione a
  \item $p \xrightarrow[]{t} p'$ : \textbf{t-mossa}, l'avversario puo' fare una \textbf{t-mossa} o stare fermo
\end{itemize}

  L'attaccante puo' sempre decidrere se usare \textbf{P} o \textbf{Q} come mossa.

\subsection{Bisimulazione debole}

La relazione $R : P \times P$,  \`e una relazione di equivalenza (di bisimulazione forte) sse

$\forall p,q \in P \;\; | \;\; p,q \in P$ 

\begin{itemize}
  \item $\forall \alpha \in Act = A \cup \overline A \cup \{t\}, \;\; se \; p \xrightarrow[]{a} p' \; | \; \exists q' \in Q \; | \; q \xRightarrow[]{a} q'$
  \item $\forall \alpha \in Act = A \cup \overline A \cup \{t\}, \;\; se \; q \xrightarrow[]{a} q' \; | \; \exists p' \in P \; | \; p \xRightarrow[]{a} p'$
\end{itemize}

$p \xleftrightarrow[]{bis} q$ sse $\exists R$ relazione di Bisimulazione t.c. $p \; R \; q$.

\subsection{Bisimulazione forte}

La relazione $R : P \times P$,  \`e una relazione di equivalenza (di bisimulazione forte) sse

$\forall p,q \in P \;\; | \;\; p,q \in P$ 

\begin{itemize}
  \item $\forall \alpha \in Act = A \cup \overline A \cup \{t\}, \;\; se \; p \xrightarrow[]{a} p' \; | \; \exists q' \in Q \; | \; q \xrightarrow[]{a} q'$
  \item $\forall \alpha \in Act = A \cup \overline A \cup \{t\}, \;\; se \; q \xrightarrow[]{a} q' \; | \; \exists p' \in P \; | \; p \xrightarrow[]{a} p'$
\end{itemize}

$p \xleftrightarrow[]{bis} q$ sse $\exists R$ relazione di Bisimulazione t.c. $p \; R \; q$.

\section{Bisimulazione debole come simulazione avversaria}

\begin{itemize}
  \item Attaccante: cerca di dimostrare $p \not \equiv_{BIS} q$
  \item Difensore: cerca di dimostrare $p \equiv_{BIS} q$
\end{itemize}

\subsection{Una simulazione G(p, q)}

\`e composta di varie \textbf{run}. Ogni \textbf{run} \`e una sequenza finita o infinita di configurazioni (p0,q0), (p1,q1), ... (pi,qi)
In ogni \textbf{step} di \textbf{run}, si passa da una configurazione all'altra (parte l'attaccante).

\begin{itemize}
  \item si sceglie uno dei due processi ed esegue una azione \textbf{a}, $p \xrightarrow[]{a} p'$, si trasforma l'altro processo con l'azione \textbf{a} verificando che il processo $q'$ \`e bisimile a $p'$.
  \item \`e possibile che uno dei due attori non possa muovere, in quel caso vince l'altro
  \item se la \textbf{run} \`e finita, allora vince il difensore
  \item importante: il difensore puo' usare la \textbf{t} come un ponte jolly.
\end{itemize}

per ogni simulazione, solo uno dei due attori ha una strategia vincente che gli da la possibilita' di vincere ogni partita

\begin{itemize}
  \item $P_1$ = t . a . NIL + b . NIL
  \item $P_2$ = t (a . NIL + b . NIL)
\end{itemize}

attaccante con $ t \; . \; b$ su $P_2$

