\chapter{20/03}

\section{1}

Si consideri la seguente base di dati:

\begin{itemize}
  \item Production(SerialNumber, PartType, Model, Quantity, Machine)
  \item Pickup(SerialNumber, Lot, Client, SalesPerson, Amount)
  \item Client(ClientID, Name, City, Address)
  \item SalesPerson(SalesID, Name, City, Address)
\end{itemize}

Inoltre:
\begin{itemize}
  \item Si assumano 4 production centers (Dublin, San Jose, Zurichm Taiwan), e tre punti vendita (San Jose, Zurich, Taiwan).
  \item Ogni production center e' responsabile per una parte: Cable (Dublin), CPU (San Jose), Keyboard (Zurich), Screen (Taiwan).
  \item Si supponga che i clienti di X siano serviti solo dalle persone del reparto vendite di X. (assumi che Zurich serva anche Dublin).
  \item Assumi che ogni area geografica abbia il suo database.
\end{itemize}

Crea una frammentazione orizzontale per le tabelle.

\putimage{images/e01-01.png}{}{png:e1-1}

\subsection{Soluzione}

Se ho gia' creato un frammento potrei riutilizzarlo in un'altra query ma qui assumiamo che ogni query veda solo lo schema globale.

Frammento Production:
\begin{itemize}
  \item Production\_1 = $\sigma _ {PartType = CPU} (Production)$
  \item Production\_2 = $\sigma _ {PartType = KEYBOARD} (Production)$
  \item Production\_3 = $\sigma _ {PartType = SCREEN} (Production)$
  \item Production\_4 = $\sigma _ {PartType = CABLE} (Production)$
\end{itemize}

Frammento Pickup:
\begin{itemize}
  \item Pickup\_1 = $\pi _ {*Pickup} (Pickup \triangleright \triangleleft _ {Pr.SerialNumber = Pi.SerialNumber} (\sigma _ {PartType = CPU} (Production)))$
  \item Pickup\_2 = $\pi _ {*Pickup} (Pickup \triangleright \triangleleft _ {Pr.SerialNumber = Pi.SerialNumber} (\sigma _ {PartType = KEYBOARD} (Production)))$
  \item Pickup\_3 = $\pi _ {*Pickup} (Pickup \triangleright \triangleleft _ {Pr.SerialNumber = Pi.SerialNumber} (\sigma _ {PartType = SCREEN} (Production)))$
  \item Pickup\_4 = $\pi _ {*Pickup} (Pickup \triangleright \triangleleft _ {Pr.SerialNumber = Pi.SerialNumber} (\sigma _ {PartType = CABLE} (Production)))$
\end{itemize}

Frammento Client:
\begin{itemize}
  \item Client\_1 = $\sigma _ {City = San Jose} (Client)$
  \item Client\_2 = $\sigma _ {City = Zurich or City = Dublin} (Client)$
  \item Client\_3 = $\sigma _ {City = Taiwan} (Client)$
\end{itemize}


Per qualche ragione non esistendo una sede di Dublin ci sono comunque SalesPerson con City = Dublin, quindi le devo assegnare a Zurich. Frammento SalesPerson:
\begin{itemize}
  \item SalesPerson\_1 = $\sigma _ {City = San Jose} (SalesPerson)$
  \item SalesPerson\_2 = $\sigma _ {City = Zurich or City = Dublin} (SalesPerson)$
  \item SalesPerson\_3 = $\sigma _ {City = Taiwan} (SalesPerson)$
\end{itemize}

\section{2}

\begin{itemize}
  \item 1) Determine the available quantity of the product 77y6878
  \item 2) Determine the client IDs who have bought a lot from the retailer Wong, who has an office in Taiwan
  \item 3) Determine the machines used for the production of the parts type Keyboard sold to the client Brown (only one is present)
  \item 4) Modify the address of the retailer Brown (only one is present), who is moving form \textit{27 Church St.} in Dublin to \textit{43 Park Hoi St.} in Taiwan
  \item 5) Calculate the sum of amounts of the orders received in San Josè, Zurich and Taiwan (note that the aggregate functions are also distributable)
\end{itemize}

\subsection{Soluzione}

\paragraph{1}

\begin{itemize}
  \item SELECT Quantity FROM Production WHERE SerialNumber = 77y6878
  \item (SELECT Quantity FROM Production\_1 WHERE SerialNumber = 77y6878) UNION (SELECT Quantity FROM Production\_2 WHERE SerialNumber = 77y6878) UNION (SELECT Quantity FROM Production\_3 WHERE SerialNumber = 77y6878) UNION (SELECT Quantity FROM Production\_4 WHERE SerialNumber = 77y6878)
  \item (SELECT Quantity FROM Production\_1@Server1 WHERE SerialNumber = 77y6878) UNION (SELECT Quantity FROM Production\_2@Server2 WHERE SerialNumber = 77y6878) UNION (SELECT Quantity FROM Production\_3@Server3 WHERE SerialNumber = 77y6878) UNION (SELECT Quantity FROM Production\_4@Server4 WHERE SerialNumber = 77y6878)
\end{itemize}

\paragraph{2}
\begin{itemize}
  \item SELECT UNIQUE Client FROM Pickup\_3 JOIN SalesPerson\_3 ON Pickup\_3.SalesPerson = SalesPerson\_3.SalesID WHERE SalesPerson\_3.Name = Wong GROUP BY (Client) HAVING COUNT(*) > 10
\end{itemize}

\paragraph{3}

\paragraph{4}

\paragraph{5}
