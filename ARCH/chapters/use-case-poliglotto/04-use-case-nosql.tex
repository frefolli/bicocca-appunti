\chapter{Use Case NOSQL}

\section{Use Case - 1}

L'obiettivo e' costruire un DB per memorizzare le risposte di un API in ambito crypto: date le API response si vuole costruire delle tabelle per gestirne i dati.

\putimage{images/04-01.png}{Esempio di Risposta}{png:4-1}

Si puo' pensare di costruire una tabella per ogni struttura di risposta, ma c'e' un problema, i dati possono essere strutturati:

\putimage{images/04-02.png}{Esempio di Schema Strutturato}{png:4-2}

Abbiamo bisogno di un modo per rappresentare un documento multi livello o strutturato senza esplosione delle tabelle, non possiamo creare una tabella per ogni possibile struttura diversa o per ogni array.

\putimage{images/04-03.png}{Esplosione delle Entita'}{png:4-3}

Qui si evidenzia il problma fondamentale delle applicazioni che lavorano su basi di dati: il collante tra il DB e l'applicazione, la configurazione.
Esiste ORM come tanti altri paradigmi ma generalmente serve molta configurazione per mappare i dati da e al DB.

\putimage{images/04-04.png}{Il Collante}{png:4-4}

\section{Use Case - 2}

Si vuole creare un nuovo social network perche' non ne esistono gia' troppi. Serve una basi di dati \textbf{relazionale} per memorizzare gli utenti e le relazioni \textit{segue} e \textit{seguito da}.

\putimage{images/04-05.png}{Schema del DB}{png:4-5}

A volte il problema non e' solo schematizzare i dati, ma anche interagirci ... gia' una query per trovare gli \textit{amici degli amici} e' "complessa", figuriamoci se dovessi andare avanti con gli \textit{amici degli amici degli amici degli amici} (se pensi che sia improbabile non hai lavorato sui piani di manutenzione e le configurazioni dei livelli).
Il sistema potrebbe anche non reggere fisicamente a certi tipi di operazioni.

\putimage{images/04-06.png}{Query di "Amici di Amici"}{png:4-6}

\section{NoSQL}

E' un insieme di modelli di rappresentazione dati + software di gestione che sperimentano soluzioni alternative a SQL.
Sono schemaless (senza schema), i dati sono lo schema (durante l'inserimento la memorizzazione implica la sua struttura).

Il vantaggio e' che i cambiamenti non sono onerosi, ma il problema e' che non si sa cosa contenga un database.
Lo schema sono i dati stessi, e' necessario leggerli, non esiste un modo per sapere la loro struttura.

C'e' l'assunzione di mondo aperto (TODO).
Piu' semplice e' il modello piu' scala bene ed e' flessibile.

\putimage{images/04-07.png}{Modelli NoSQL}{png:4-7}

\subsection{Key-Value}

Un esempio e' Redis. E' il modello piu' semplice che si possa immaginare: una tabella chiave valore.
Spesso e' tutto in memory.

\subsection{Wide Column}

Un esempio e' Cassandra. Ogni riga ha dei campi e qualcosa di diverso, non specifico uno schema ma solo i dati che conosco su una certa riga.

\putimage{images/04-08.png}{Esempio}{png:4-8}

\subsection{Document Based}

Un esempio e' CouchDB. E' un archivio di documenti (JSON per esempio). I documenti sono indirizzati nel DB tramite una chiave unica e si possono effettuare query di ricerca su essi.

\putimage{images/04-09.png}{Esempio}{png:4-9}

\subsection{Graph Based}

Un esempio e' Neo4j.

\putimage{images/04-10.png}{Esempio}{png:4-10}

\subsection{Comfronto}

Ogni modello ha i suoi pregi e i suoi difetti, per esempio un Grafo non puo' essere distribuito in modo orizzontale su piu' macchine mentre un Key Value non ha problemi.

\putimage{images/04-11.png}{Complessita}{png:4-11}

\putimage{images/04-12.png}{Dimensione e Gomplessita}{png:4-12}

\section{Connettere i Dati}

\subsection{Model Data}

Il problema rimane come mettere insieme dati legati fra loro.

\putimage{images/04-13.png}{Esempio Relazionale}{png:4-13}

Nel modello documentale i dati possono essere innestati, anche quelli delle relazioni delle "entita'":

\putimage{images/04-14.png}{Esempio JSON}{png:4-14}

Il problema di un approccio del genere e' che l'unico modo per accedere ai dati delle relazioni sono costretto a passare per i singoli documenti.
L'esempio di prima assume che si acceda sempre alle auto solo dopo alle persone.

E' necessario progettare la base di dati in base al problema che si deve risolvere e per come li deve processare l'applicazione. (<<A basi di dati si insegnava il contrario, ci scusiamo>> - Maurino).

\subsection{Model Comparison}

Questi sono diversi modelli per rappresentare gli stessi dati. Non esiste un modello giusto o sbagliato, ne esiste solo un piu' corretto in base all'ambito e alle risorse del problema (e' appaltato al progettista l'onere della scelta).

\putimage{images/04-15.png}{Esempio Relazionale}{png:4-15}

\putimage{images/04-16.png}{Esempi NoSQL}{png:4-16}

Diventa molto divertente lavorare con modelli ML su strutture a grafo o a documento. Il problema e' avere i dati nella forma d'uso, altrimenti e' necessario trasformarli o estrarre banche dati che possono introdurre errori (oversampling se porto i dati dell'entita' nella relazione per esempio).

\section{Key Value}

Il modello e' molto semplice, associo ad una chiave un valore (generalmente blob) accanto a cui posso salvare anche informazioni sul tipo del dato.

\putimage{images/04-17.png}{Modello Key Value}{png:4-17}

E' generalmente molto efficiente grazie all'uso delle HashTable e si presta alla distribuzione. Le operazioni sono le solite: aggiunta, cancellazione, ricerca.

\subsection{Redis}

Redis accomoda diversi tipi di dati: hashset, liste, code ... cosi' come i dati "primitivi" (stringhe, interi).

\putimage{images/04-18.png}{Esempio di Record}{png:4-18}

\putimage{images/04-19.png}{Modello di Redis}{png:4-19}

\subsection{Wide Column}

\subsection{BigTable}

\subsection{HBase}

\subsection{Cassandra}
