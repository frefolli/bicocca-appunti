\chapter{Introduzione}

Il corso approfondisce le architetture dati nel senso di DBMS:

\begin{itemize}
  \item Progettare l'architettura dati piu' adatta valudando degli specifici use case
  \item Comprendere i meccanismi di gestione moderni dei DBMS
  \item I Modelli di rappresentazione dei dati alternativi ai modelli relazionali
  \item DBMS principali e i loro principi
\end{itemize}

In particolare durante il corso si approfondiranno i seguenti use case, che sono articolati in parti nel documento presente:

\begin{itemize}
  \item \textbf{Ambito bancario}: si vuole realizzare una architettura dati per consentire le normali operazioni bancarie come i bonifici
  \item \textbf{Machine learning}: si vuole realizzare una pipeline per la costruzione dei dataset di training per addestrare modelli di Machine/deep learning e le relative pipeline per l’applicazione dei modelli addestrati sui dati in esecuzione
  \item \textbf{Modelli e Architetture non relazionali}: si vuole gestire dati che non sono facilmente modellabili nel modello relazionale o che non possono essere gestiti dai sistemi distribuiti relazionali
  \item \textbf{Cloud}: Si vuole utilizzare sistemi di gestione dei dati in cloud
  \item \textbf{AI generativa per la gestione dei dati}: si vuole capire come utilizzare l'AI generativa nei sistemi di gestione dei dati
\end{itemize}

L'esame e' uno scritto con Domande a risposte aperte sulla teoria/esercizi di progettazione di architetture o sistemi di gestione dati/esercizi sui linguaggi di interrogazioni NoSQL.

In alternativa è possibile realizzare un elaborato che approfondisca un particolare tema / use cases mostrato a lezione, da concordare con il docente.
