\chapter{Use Case Distribuito Relazionale}

\section{Introduzione}

L'architettura dati deve essere in grado di gestire le operazioni bancarie, quindi transazioni e bonifici tra piu' banche.

Il workload e' misto, si avranno milioni di letture e scritture ogni giorno, di carattere semplice. Il volume di dati e' grande e il sistema e' eterogeneo.
Infatti non solo le banche possono usare sistemi diversi, ma quando avvengono acquisizioni e fusioni spesso e volentieri accade che i sistemi delle sottobanche siano mantenuti e messi in parallelo.

Le metriche di qualita' sono il numero di operazioni che e' in grado di soddisfare e la latenza di esse.

\paragraph{Un Sistema Centralizzato va bene?}

Necessariamente no, non solo perche' le banche sono piu' di una e potenzialmente distanti tra loro ma anche perche' il volume di operazioni e' tale da richiedere un sistema distribuito.

\paragraph{Sistemi Distribuiti}

Nella progettazione di un sistema distribuito il problema principale e' rappresentato dal capire cosa e come distribuire le risorse e l'elaborazione. Nella progettazione di architetture dati ci si pongono gli stessi questi.
Infatti l'elaborazione puo' essere su piu' nodi cosi' come lo storage dei dati (DDBMS). La distribuzione in un livello e' comunque ortogonale e trasparente agli altri livelli (Layered Architecture).

Particolari sono i casi di merger \& acquisition perche' richiedono un problema tecnologico ulteriore: capire come unire logiche applicative e sistemi di archiviazione.

\section{Un Esempio}

Si consideri la base di dati di una azienda con progetti e vendite. Questa ha piu' sedi: Milano, Piacenza, Bologna.

\putimage{images/01-01.png}{}{png:1-1}

Una base di dati centralizzata avra' dei nodi elaborativi si' anche a Piacenza e a Bologna, ma l'archiviazione risiedera' solo a Milano.
Questo introduce diversi problemi, primo di tutti la rete: infatti una congestione della rete puo' rendere impossibile alle altre due sedi di acquisire i dati necessari all'elaborazione.

\putimage{images/01-02.png}{}{png:1-2}

Quindi e' utile dividere il database in frammenti per poter avere i dati vicini il piu' possibile al luogo di utilizzo.
Dopo di che possiamo anche immaginare di mantenere delle copie di alcuni frammenti anche su altri nodi se sono necessari all'elaborazione oppure per mantenere un po' di ridondanza.

\putimage{images/01-03.png}{}{png:1-3}

\section{Modelli Architetturali per DBMS}

\paragraph{Un DBMS} Distribuito Eterogeneo Autonomo e' una federazione di DBMS che collaborano nel fornire servizi di accesso ai dati con livelli di trasparenza definiti.

\putimage{images/01-04.png}{}{png:1-4}

\paragraph{Per trasparenza} si intende la proprieta' di nascondere le diversita' tra basi di dati nei nodi del sistema riguardo alla distribuzione, all'eterogeneita' e all'autonomia dei nodi.

\subsection{Autonomia}

\paragraph{Per autonomia} si intende il grado di indipendenza dei nodi che puo' assumere varie forme:

\begin{itemize}
  \item \textbf{Di progetto}: ogni nodo adotta un proprio modello dei dati e sistema di gestione delle transazioni
  \item \textbf{Di condivisione}: ogni nodo sceglie la porzione di dati che intende condividere con altri nodi
  \item \textbf{Di esecuzione}: ogni nodo decide in che modo eseguire le transazioni che gli vengono sottoposte
\end{itemize}

\paragraph{In DBMS integrati} i dati sono logicamente centralizzati e abbiamo un unico data manager che esegue le transazioni applicative ed orchestra i data manager locali.

\paragraph{Si parla di DBMS semi autonomi} quando ogni data manager e' autonomo ma partecipa a transazioni globali, una parte dei dati e' condivisa e si richiedono modifiche archietturali per formare una federazione.

\paragraph{Invece esistono DBMS totalmente autonomi} dove ognuno lavora in completa autonomia ed e' inconsapevole dell'esistenza di altri nodi.

\subsection{Distribuzione}

\paragraph{La Distribuzione} fa riferimento alla distribuzione dei dati, dove possiamo avere architetture client/server con i dati nel server o distribuzioni peer-to-peer in cui non c'e' distinzione tra client/server e tutti i nodi del sistema sono DBMS equipollenti.

\subsection{Eterogeneita'}

\paragraph{Questa} riguarda o il modello dei dati (nominalmente relazionale o no), il linguaggio di query, il sistema delle transazioni o lo schema concettuale/logico della base di dati.

\section{Progettazione di un DBMS Autonomo (DBMSA)}

Noi mettiamo il focus sullo schema "Shared Nothing", ovvero i nodi sono completamente slegati da legami infrastrutturali come possono essere invece lo schema Oracle RAC o lo Shared-Everything.

\putimage{images/01-05.png}{Architettura Funzionale di un DBMS}{png:1-5}

Per ciascuna funzione di un DBMS nei suoi nodi possiamo avere una gestione centralizzata/gerarchica o distribuita con assegnazione statica o dinamica dei ruoli (importante per la ripresa da shock).

Nella progettazione di un DBMS Distribuito si introduce la fase di progettazione della distribuzione successiva alla progettazione concettuale e le progettazioni logico/fisica diventano locali ai nodi.

\paragraph{Per quanto riguarda Portabilita' e Interoperabilita'} dei DBMS DEA queste sono facilitate dalla standardizzazione di protocolli e linguaggi ma comunque possono avere dei problemi (si pensi ai diversi dialetti adottati da alcuni DBMS ...).

\subsection{Vantaggi}

Disporre di nodi di DBMS locali permette di avvicinare i dati alle applicazioni rendendo i tempi di accesso tendenzialmente inferiori e la modularita' che ne consegue da la possibilita' di effetturare modifiche a basso costo (locali ai nodi).
Inoltre e' resistente ai guasti in quando solo il nodo locale rotto sara' incapacitato, non tutto l'apparato.

La distribuzione dei dati generalmente riguarda la distizione tra \textbf{Utenti Locali} e \textbf{Utenti Globali} e puo' essere incrementale e progressiva, molto flessibile.
Inoltre la possibilita' di avere ridondanza riduce i danni da guasto nei singoli nodi ("fail soft") dalno la possibilita' di riassegnare le rotte (operazione comunque non priva di costi, tanto che alcune aziende preferiscono pagare le penali che adottare alcuni sistemi di ripresa).

Infine per quanto riguarda le \textbf{Prestazioni} la possibilita' di effettuare operazioni locali introduce la parallelizzazione di esse per quanto possibile ma anche la necessita' di coordinamento tra i nodi per effettuare transazioni globali (con potenzialmente aumento del traffico in rete).

\section{Funzionalita' Specifiche dei DDBMS rispetto ai DBMS Centralizzati}

Ogni server mantiene la capacita' di gestire applicazioni in modo indipendente ma coopera con gli altri server, e queste interazioni sono un carico aggiuntivo sul sistema complessivo.

La rete e' l'elemento critico, quindi e' necessario distribuire i dati in modo che la maggior parte delle transazioni sia locale o si eviti l'invio di dati tra nodi.

Rispetto ai DBMS Centralizzati abbiamo sia la trasmissione di query che di dati del DB, che di dati di controllo. Inoltre ci sono in piu' da gestire la frammentazione, la replicazione in modo da garantire la trasparenza ai nodi.

Il query processor deve pianificare una strategia globale accanto a quelle locali e bisogna controllare la concorrenza durante le transazioni per evitare operazioni non consistenti. Inoltre si devono prevedere strategie di ripristino dei singoli nodi come parte della gestione di guasti globali.

\subsection{Frammentazione, Replicazione, Trasparenza}

\paragraph{La frammentazione} e' la possibilita' di allocare porzioni diverse del DB su nodi diversi

Questa puo' essere sia verticale (sulle colonne), che orizzonale (sulle righe).

Chiaramente durante la frammentazione e' necessario garantire la \textbf{Completezza} dei dati, la loro \textbf{Ricostruibilita'} e la \textbf{Disgiunzione} (ogni elemento e' in un solo frammento a meno di Replicazione).

\paragraph{La replicazione} e' la possibilita' di allocare le stesse porzioni di DB su nodi diversi

\paragraph{La trasparenza} rende possibile al sistema di accedere ai dati senza sapere dove sono allocati. Questa in particolare e' una caratteristica che di solito si ottiene solo pagando una certa cifra (altrimenti solitamente bisogna indirizzare tutto a mano).

Questa e' solitamente di due tipi:

\begin{itemize}
  \item transparenza logica: indipendenza dell'applicazione da modifiche dello schema logico.
  \item trasparenza fisica: indipendenza dell'applicazione da modifiche allo schema fisico.
\end{itemize}

A sua volta la trasparenza si divide in:

\begin{itemize}
  \item trasparenza di frammentazione: il sistema non sa che i dati sono frammentati.
  \item trasparenza di allocazione: sa che e' frammentato ma non sa dove sono allocati.
  \item trasparenza di linguaggio: e' possibile usare lo stesso linguaggio di query (SQL standard) a prescindere dalle interfacce dei nodi singoli.
\end{itemize}
