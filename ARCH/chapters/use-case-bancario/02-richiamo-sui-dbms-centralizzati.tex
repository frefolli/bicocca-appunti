\chapter{Richiamo sui DBMS Centralizzati}

\section{DBMS}

E' un sistema in grado di gestire collezioni di dati che siano di grandi dimensioni, persistenti, condivise ed affidabili (resistenti ai guasti).

\subsection{Caratteristiche}

Abbiamo un unico schema logico, una unica base di dati e un unico schema fisico, quindi nessuna forma di eterogeneita' contettuale o fisica, e nessuna forma di distribuzione.

Inoltre abbiamo un unico linguaggio di interrogazione, un sistema di gestione e di accesso, una modalita' di ripristino e un amministratore dei dati.

La persistenza richiede una gestione in memoria secondaria sogisticata in modo da archiviare ed accedere in modo efficiente una grande mole di dati.

Inoltre richiede l'uso condiviso di risorse tra piu' applicazioni, e quindi la gestione della concorrenza di operazioni sulla stessa base di dati tramite meccanismi di autorizzazione e di controllo della concorrenza che garantiscano si' la sicurezza ma anche una ragionevole efficienza.

Le interrogazioni devono essere ottimizzate il piu' possibile e i dati devono essere conservati anche in presenza di malfunzionamenti.

\putimage{images/02-01.png}{Schema di un DBMS}{png:2-1}

\section{Ottimizzazione delle Interrogazioni}

Si trasforma il query tree (parsed SQL) in un query plan logico contenente operazioni di algebra relazionale sulle tabelle logiche dello schema e lo ottimizza rispetto al costo.

\putimage{images/02-02.png}{Query Tree/Plan}{png:2-2}

Visto che un query tree puo' essere trasformato in piu' Query Plan logici possibili, si sceglie quello computazionalmente migliore.
Ad esempio, in genere conviene che le selezioni siano anticipate rispetto ai join, perche' in questo modo i join operano su tabelle di minore dimensione.

La trasformazione e' effettuata usando proprieta' algebriche e stime dei costi delle operazioni accessibile tramite una tabella con delle statistiche.
In generale il problema di ottimizzazione ha complessita' esponenziale, quindi si introducono delle approssimazioni (si ottimizza finche' si riesce in base a delle euristiche).

\putimage{images/02-03.png}{Schema di Ottimizzazione}{png:2-3}

\section{Transazioni}

Una transazione e' una serie di istruzioni di lettura e scrittura sulla base di dati (eventualmente in un linguaggio di programmazione) il cui esito puo' essere \textbf{commit work} o \textbf{rollback work}; ovvero memorizza quanto fatto o dimentica tutto e vai avanti, non sono ammesse soluzioni intermedie.
Un sistema transazionale (OLTP) e' in grado di definire ed eseguire le transazioni per conto di un certo numero di applicazioni concorrenti.

\subsection{Proprieta' ACID}

\paragraph{Atomicita'}: una transazione e' una unita' atomica di elaborazione, non puo' lasciare la base i dati in uno stato intermedio.

\paragraph{Consistenza}: la transazione rispetta i vincoli di integrita', ovvero se lo stato iniziale e' corretto anche lo stato finale sara' corretto.

\paragraph{Isolamento}: la transazione non risente degli effetti delle altre transazioni concorrenti (determinismo?).

\paragraph{Durata} (Persistenza): un commit prsiste i dati anche in caso di quasti (necessario che il DBMS garantisca l'affidabilita').

Le anomalie che possono avvenire durante piu' transazioni concorrenti:

\begin{itemize}
  \item \textbf{update loss}: una scrittura e' persa per colpa di una sovrascrittura
  \item \textbf{dirty read}: una lettura di un dato scritto e non confermato (commit)
  \item \textbf{unrepeatable read}: due letture di seguito restituiscono valori differenti
  \item \textbf{ghost update}: \textit{non l'ho capito in realta'}
\end{itemize}

\putimage{images/02-04.png}{Foto di Maurino non spiegata, penso che sia un update loss (ha perso il posto in aereo)}{png:2-4}

\subsection{Gestione della Concorrenza}

Una sequenza di esecuzione di un insieme di transazioni e' detta \textbf{schedule}. Essa si dice seriale se una transazione termina prima che la successiva inizi.
Una schedule e' serializzabile se l'esito della sua esecuzione e' lo stesso indipendentemente dall'ordine di esecuzione seriale delle transazioni in essa contenute.

Per controllare la concorrenza e' necessaria la presenza di uno scheduler a cui si richiedono e si rilasciano i \textbf{lock}, le cui assegnazioni sono salvate nella lock table.
Si aggiunge anche la regola \textbf{Two Phase Locking} (2PL), ovvero in ogni tutte le richieste di lock precedono tutti gli unlock.
